\documentclass[12pt, catalan]{article}

\usepackage[utf8]{inputenc}
\usepackage[T1]{fontenc}
\usepackage{babel}
\usepackage{lmodern}
\usepackage{geometry}
\usepackage{hyperref}
\usepackage[dvipsnames]{xcolor}
\usepackage[bf,sf,small,pagestyles]{titlesec}
\usepackage{titling}
\usepackage[font={footnotesize, sf}, labelfont=bf]{caption} 
\usepackage{siunitx}
\usepackage{graphicx}
\usepackage{booktabs}
\usepackage{amsmath,amssymb}
\usepackage[sort]{cleveref}
\usepackage{enumitem}

% Configuració dels marges
\geometry{
	a4paper,
	right = 2.5cm,
	left = 2.5cm,
	bottom = 3cm,
	top = 3cm
}

% Configuració dels links i referències
\hypersetup{
	colorlinks,
	linkcolor = {red!50!blue},
	linktoc = page
}

\crefname{figure}{figura}{figures}
\crefname{table}{taula}{taules}
\numberwithin{table}{section}
\numberwithin{figure}{section}
\numberwithin{equation}{section}

% Directori per les figures
\graphicspath{{./figs/}}

% Format de la pàgina
\newpagestyle{pagina}{
	\headrule
	\sethead*{\sffamily {\bfseries Entrega 2}}{}{\theauthor}
	\footrule
	\setfoot*{}{}{\sffamily \thepage}
}
\renewpagestyle{plain}{
	\footrule
	\setfoot*{}{}{\sffamily \thepage}
}
\pagestyle{pagina}

% Nous comands
\newcommand{\cond}{\, \vert \,}
\DeclareMathOperator{\DGamma}{Gamma}
\DeclareMathOperator{\E}{\mathbb{E}}
\DeclareMathOperator{\Var}{Var}

\title{\sffamily {\bfseries Entrega 2:} Comparació d'estimadors}
\author{\sffamily Andreu Arderiu, Arnau Mas, Alejandro Plaza}
\date{\sffamily 25 d'abril de 2019}

\begin{document}

\maketitle

\section{Introducció}
La funció de densitat d'una variable aleatòria \( X \) que segueix la distribució gamma és
\begin{equation*}
	f(x \cond \alpha, \nu) = \frac{\alpha^\nu x^{\nu - 1}}{\Gamma(\nu)} e^{-\alpha x}
\end{equation*}
on \( \alpha \) s'anomena el paràmetre de ràtio (\emph{rate} en anglès) i \( \nu \) rep el nom de paràmetre de forma (\emph{shape} en anglès). El nostre objectiu és produir estimadors per aquests dos paràmetres mitjançant els mètodes de màxima versemblança i dels moments i comparar-los.  

\subsection{Mètode dels moments}
La funció generatriu de moments d'una variable aleatòria de distribució gamma, \( X \sim \DGamma(\alpha, \nu) \) és
\begin{equation*}
	M_X(t) = \alpha^\nu (\alpha - t)^{\nu - 1},
\end{equation*}
i les successives derivades són
\begin{gather*}
	M'_X(t) = \nu \alpha^\nu (\alpha - t)^{-\nu - 1} \\
	M''_X(t) = \nu(\nu + 1) \alpha^\nu (\alpha - t)^{-\nu - 2}, 
\end{gather*}
Per tant, per la definició de la funció generatriu tenim
\begin{equation} \label{eq:moments}
	\begin{gathered}
		\E[X] = M'_X(0) = \frac{\nu}{\alpha} \\
		\Var[X] = \E[X^2] - \E[X]^2 = M''_X(0) - M'_X(0)^2 = \frac{\nu(\nu + 1)}{\alpha^2} - \left( \frac{\nu}{\alpha} \right)^2 = \frac{\nu}{\alpha^2}. 
	\end{gathered}
\end{equation}
Si invertim l'\cref{eq:moments} obtenim
\begin{equation} \label{eq:moments invertit}
	\begin{gathered}
		\alpha = \frac{\E[X]}{\Var[X]} \\
		\nu = \frac{\E[X]^2}{\Var[X]}
	\end{gathered}
\end{equation}
Per tant els estimadors per als paràmetres que obtenim amb el mètode dels moments són
\begin{equation} \label{eq:estimadors moments}
\begin{gathered}
	\hat{\alpha}_\text{Mom}(X) = \frac{\bar{X}}{S^2} \\
	\hat{\nu}_\text{Mom}(X) = \frac{\bar{X}^2}{S^2} \\
\end{gathered}
\end{equation}
on \( \bar{X} \) i \( S^2 \) són la mitjana i variància mostrals.   


\end{document}
