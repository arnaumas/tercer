\documentclass[12pt,twoside]{report}

\usepackage[utf8]{inputenc}
\usepackage[T1]{fontenc}
\usepackage[catalan]{babel}
\usepackage{lmodern}
\usepackage{geometry}
\usepackage[dvipsnames]{xcolor}
\usepackage[bf,sf,small,pagestyles]{titlesec}
\usepackage{titling}
\usepackage[font={footnotesize, sf}, labelfont=bf]{caption} 
\usepackage{siunitx}
\usepackage{graphicx}
\usepackage{booktabs}
\usepackage{amsmath}
\usepackage{amssymb}
\usepackage{amsthm}
\usepackage{thmtools}
\usepackage{hyperref}
\usepackage[catalan,sort]{cleveref}
\usepackage[shortlabels]{enumitem}

\geometry{
	a4paper,
	right = 3cm,
	left = 3cm,
	bottom = 3cm,
	top = 3cm
}

\hypersetup{
	colorlinks,
	linkcolor = {red!50!blue},
	linktoc = page
}

\crefname{figure}{figura}{figures}
\crefname{table}{taula}{taules}
\numberwithin{table}{section}
\numberwithin{equation}{section}
\numberwithin{figure}{section}

\declaretheoremstyle[spaceabove=6pt, spacebelow=6pt, headfont=\bfseries, notefont=\normalfont, notebraces={(}{)}, qed=\qedtriangle]{definicio}
\declaretheoremstyle[spaceabove=6pt, spacebelow=6pt, headfont=\bfseries, notefont=\normalfont, notebraces={(}{)}, qed=\qedtriangledown]{exemple}

\declaretheorem[name=Teorema, refname={teorema,teoremes}, Refname={Teorema,Teoremes}, numberwithin=chapter]{teo}
\declaretheorem[name=Proposició, refname={proposició,proposicions}, Refname={Proposició,Proposicions}, numberlike=teo]{prop}
\declaretheorem[name=Definició, style=definicio, refname={definició,definicions}, Refname={Definició,Definicions}, numberwithin=chapter]{defn}
\declaretheorem[name=Exemple, style=exemple, refname={exemple,exemples}, Refname={Exemple,Exemples}, numberwithin=chapter]{exe}

\graphicspath{{./figs/}}

\newlist{punts}{enumerate}{1}
\setlist[punts,1]{label=\textup{(}{\itshape \roman*}\textup{)}, wide}

% Unitats
\sisetup{
	inter-unit-product = \ensuremath{ \cdot },
	allow-number-unit-breaks = true,
	detect-family = true,
	list-final-separator = { i },
	list-units = single
}

\newcommand{\N}{\ensuremath{\mathbb{N}}}
\newcommand{\Z}{\ensuremath{\mathbb{Z}}}
\newcommand{\Q}{\ensuremath{\mathbb{Q}}}
\newcommand{\R}{\ensuremath{\mathbb{R}}}
\newcommand{\C}{\ensuremath{\mathbb{C}}}
\newcommand{\Cu}{\ensuremath{\mathbb{C}^\times}}
\newcommand{\qedtriangle}{\ensuremath{\triangle}}
\newcommand{\qedtriangledown}{\ensuremath{\bigtriangledown}}
\renewcommand{\vec}[1]{\mathbf{#1}}
\newcommand{\abs}[1]{\left\lvert #1 \right\rvert}
\newcommand{\norm}[1]{\left\lVert #1 \right\rVert}
\let\Im\relax
\let\Re\relax
\DeclareMathOperator{\Im}{Im}
\DeclareMathOperator{\Re}{Re}
\DeclareMathOperator{\Arg}{Arg}
\DeclareMathOperator{\Log}{Log}
\DeclareMathOperator{\Hol}{Hol}
\newcommand{\parbreak}{
	\begin{center}
		--- $\ast$ ---
	\end{center} 
}
\makeatletter
\newcommand*{\defeq}{\mathrel{\rlap{%
    \raisebox{0.3ex}{$\m@th\cdot$}}%
  \raisebox{-0.3ex}{$\m@th\cdot$}}%
	=
}
\makeatother

\newpagestyle{pagina}[\sffamily \footnotesize]{
	\headrule
	\sethead{{\bfseries Capítol \thechapter.} \chaptertitle}{}{\ifthesection{{\bfseries \thesection} \sectiontitle}{}}
	\footrule
	\setfoot*{}{}{\thepage}
}
\renewpagestyle{plain}[\sffamily \footnotesize]{
	\footrule
	\setfoot*{}{}{\thepage}
}
\assignpagestyle{\chapter}{plain}
\pagestyle{pagina}

\titleformat{\chapter}[block]{\sffamily \bfseries \Huge}{\filleft \large Capítol \Huge \thechapter\\}{0pt}{\Huge \titlerule[1pt] \vspace{1ex} \filleft}

\title{Anàlisi Complexa i de Fourier}
\author{Arnau Mas}
\date{2019}

\begin{document}
\maketitle

\chapter{Propietats fonamentals del pla complex}

\section{Argument}
Per tot \( z \in \Cu \), el complex \( \frac{z}{\abs{z}} \) té modul 1. Per tant podem escriure
\begin{equation*}
	\frac{z}{\abs{z}} = \cos{\theta} + i \sin{\theta}.
\end{equation*}
Aquest \( \theta \), però, està definit mòdul \( 2\pi \), ja que tant el cosinus com el sinus són \( 2\pi \)-periòdics. Aquesta manca d'unicitat és una subtilesa important de l'análisi complexa. Escriurem \( \Arg z \) per denotar el conjunt de tots els possibles arguments de \( z \). Si \( \theta \) és un argument de \( z \) aleshores és clar que \( \Arg z = \theta + 2\pi\Z \), ja que dos angles tenen el mateix sinus i cosinus si i només si difereixen en un múltiple enter de \( 2\pi \). Per tant, el conjunt d'arguments de qualsevol \( z \in \Cu \) conté un únic element en l'interval \( (-\pi, \pi] \)\footnotemark[1], que rep el nom d'\emph{argument principal}. El denotem \( \arg z \). Aquesta tria defineix l'aplicació
\begin{align*}
	\arg \colon \Cu & \longrightarrow (-\pi, \pi] \\
	z & \longmapsto \arg{z}.
\end{align*}
L'aplicació argument, però, no és contínua al semieix negatiu. Si ens hi apropem amb complexos de part real positiva aleshores el límit és \( \pi \), mentre que si ho fem amb complexos de part real negativa el límit és \( -\pi \). Aquesta discontinuïtat és una altra important subtilesa de l'análisi complexa, responsable de la dificultat de definir conceptes com ara el logaritme complex. 

\footnotetext[1]{De fet, el conjunt d'arguments conté un únic element de qualsevol interval semiobert de longitud \( 2\pi \). La tria de \( (-\pi, \pi] \) per a l'argument principal és arbitrària. També és molt comú considerar l'argument principal com aquell que està entre 0 i \( 2\pi \).} 

Introduïm la següent notació:
\begin{equation*}
	e^{i\theta} \defeq \cos{\theta} + i\sin{\theta}.
\end{equation*}
Ara per ara, aquesta notació no és res més que una abreviació que, a priori, no té cap relació amb l'exponencial. Molt aviat veurem, però, que li podem donar sentit matemàtic de diverses maneres i totes consistents entre si. Introduïm ara una sèrie de propietats de l'argument. Donat \( z \in \Cu \) l'escrivim en \emph{forma polar} com \( z = \abs{z}e^{i\theta} \) on \( \theta \) és un argument de \( z \).

\begin{prop}[name=Propietats de l'argument]\label{prop:propietats de l'argument}
Per a qualssevol \( z, w \in \Cu \) es té
	\begin{punts}
	\item \( \Arg \bar{z} = \Arg z^{-1} = - \Arg{z}, \) 
	\item \( \Arg{(zw)} = \Arg z + \Arg w. \)
	\end{punts}
\end{prop}

\begin{proof}
	\begin{punts}
	\item	Si \( z = \abs{z}(\cos{\theta} + i \sin{\theta}) \) aleshores \[ \bar{z} = \abs{z}(\cos{\theta} - i\sin{\theta}) = \abs{\bar{z}}(\cos{(-\theta)} + i \sin{(-\theta)}), \] per tant \( -\theta \) és un argument de \( \bar{z} \), per tant \( \Arg{\bar{z}} = -\theta + 2\pi\Z = -\theta - 2\pi\Z = - \Arg{z}. \)

	\item Per tot \( \theta, \phi \in \R \) es verifica \( e^{i\theta}e^{i\phi} = e^{i(\theta + \phi)} \). En efecte
		\begin{align*}
			e^{i\theta}e^{i\phi} & = (\cos{\theta} + i \sin{\theta})(\cos{\phi} + i \sin{\phi}) \\
													 & = \cos{\theta}\cos{\phi} - \sin{\theta}\sin{\phi} + i(\cos{\theta}\sin{\phi} + \sin{\theta}\cos{\phi}) \\
													 & = \cos{(\theta + \phi)} + i \sin{(\theta + \phi)} = e^{i(\theta + \phi)}.
		\end{align*}
		Per tant, si \( z = \abs{z}e^{i\theta} \) i \( w = \abs{w}e^{i\phi} \) aleshores \( zw = \abs{z}\abs{w} e^{i\theta} e^{i\phi} = \abs{zw} e^{i(\theta + \phi)}. \). Per tant \( \Arg{(zw)} = (\theta + \phi) + 2\pi\Z = \Arg{z} + \Arg{w} \). \qedhere
	\end{punts}
\end{proof}

Cal remarcar que les anteriors propietats en general no són certes per als arguments principals. Sense anar més lluny, \( -i \) té argument principal \( \frac{-\pi}{2} \) però \( (-i)^2 = -1 \) té argument principal \( \pi \), que no és \( -\frac{\pi}{2} - \frac{\pi}{2} \). Quan fem la tria de l'argument principal guanyem en especificitat però perdem propietats.

\section{Potències i arrels de nombres complexos}
Donat un nombre complex \( z = a+bi \) podem calcular-ne potències naturals fent servir la fórmula del binomi i que \( i^2 = -1 \). També podem calcular-ne les potències negatives ja que \( z^{-1} = \frac{\bar{z}}{\abs{z}^2} \). Per exemple, \[ (a + bi)^2 = a^2 - b^2 - 2iab. \] Aquests càlculs, però, són complicats de visualitzar si els escrivim en forma binomial. El seu significat és més clar si els pensem en forma polar. Tenim
\begin{equation*}
	z^n = \left(\abs{z} e^{i\theta}\right)^n = \abs{z}^n (e^{i\theta})^n.
\end{equation*}
I per la \cref{prop:propietats de l'argument}, \( (e^{i\theta})^n = e^{in\theta} \). Per tant \( z^n \) és un nombre que té módul \( \abs{z}^n \) i argument (generalment no principal) \( n\theta \). 

Més interessant és el càlcul d'arrels \( n \)-èssimes. Sabem que a \( \R \) no tots els nombres tenen arrels. En el cas de les arrels quadrades, els nombres negatius no tenen arrel quadrada real, mentre que els reals positius en tenen dues que difereixen en un signe. Com sabem, el patró general és que tot nombre real té una única arrel senar, mentre que només els nombres reals positius tenen arrels parelles, i en tenen dues. Veurem a continuació com aquesta aparent asimetria esdevé un patró regular quan ens mirem les coses a \( \C \). 

Fem pas a pas el cas de l'arr

\chapter{Funcions holomorfes}
El principal objecte d'estudi de l'anàlisi complexa són les funcions complexes, és a dir, funcions de la forma \( f \colon \Omega \subseteq \C \to \C \). En els aspectes més senzills, les propietats de les funcions complexes són semblants a les de les funcions reals. Però ben aviat veurem que la noció de derivabilitat a \( \C \) és molt més restrictiva que a \( \R \) i dóna lloc a una classe de funcions amb molt més bones propietats que les funcions derivables reals.

\section{Continuïtat al pla complex}
El contingut d'aquesta secció és gairebé cerimonial, en el sentit de que la noció de continuïtat a \( \C \) no difereix de la que tenim a \( \R^n \). Així, si el lector ja ha vist anàlisi real, no trobarà cap novetat. No és sorprenent que això sigui així, ja que \( \C \) és homeomorf a \( \R^2 \) i per tant la continuïtat en un és la mateixa que en l'altre. 

\begin{defn}[Continuïtat]
	Diem que una funció \( f \colon \Omega \subseteq \C \to \C \) és \emph{contínua} al punt \( z \in \Omega \) si per tot \( \epsilon > 0 \) existeix un \( \delta > 0 \) tal que si \( \abs{z - w} < \delta \) aleshores \( \abs{f(z) - f(w)} < \epsilon \). I direm que \( f \) és contínua a \( \Omega \) si és contínua a tot \( z \in \Omega \). 
\end{defn}

\section{Derivabilitat al pla complex}
Com que \( \C \) és un cos, la definició de derivada és exactament la mateixa que en el cas real. 
\begin{defn}[Derivabilitat complexa]\label{def:funcio holomorfa}
	Direm que una funció \( f \colon \Omega \to \C \) és \emph{derivable} al punt \( z \in \Omega \) si existeix
	\begin{equation*}
		\lim_{h \to 0}{\frac{f(z + h) - f(h)}{h}}.
	\end{equation*}
	Diem que \( f \) és derivable a \( \Omega \) si és derivable per tot \( z \in \Omega \). En aquest cas definim la seva funció derivada,
	\begin{align*}
		f' \colon \Omega & \longrightarrow \C \\
		z & \longmapsto \lim_{h \to 0}{\frac{f(z + h)}{h}}. \qedhere
	\end{align*}
\end{defn}
Com que tota funció de variable complexa es pot pensar com una funció de \( \R^2 \), i viceversa, ens podem preguntar per la relació entre la definició de derivada que acabem de donar i la definició de diferenciabilitat a \( \R^2 \). El següent exemple i\l.lustra que la primera és més forta que la segona.

\begin{exe}[La conjugació no és una funció holomorfa]
	Considerem la funció conjugació, \( f(z) = \bar{z} \). Si La pensem com a funció de \( \R^2 \) aleshores és \( f(x,y) = (x, -y) \), que és lineal i per tant diferenciable i de fet de classe \( C^{\infty} \).
	
	Provem de calcular-ne la derivada complexa al zero. Podem fer-ho apropant-nos per l'eix real:
	\begin{equation*}
		\lim_{t \to 0}{\frac{f(t) - f(0)}{t}} = \lim_{t \to 0}{\frac{t - 0}{t}} = 1.
	\end{equation*}
	Però si ens apropem per l'eix imaginari:
	\begin{equation*}
		\lim_{t \to 0}{\frac{f(it) - f(0)}{it}} = \lim_{t \to 0}{\frac{-it - 0}{it}} = -1.
	\end{equation*}
	Així doncs \( f \) no pot ser derivable a l'origen.
\end{exe}

El següent resultat clarifica la relació entre aquestes dues nocions de derivabilitat.
\begin{prop}
	Una funció \( f \colon \Omega \to \C \) és \( \C \)-derivable en un punt \( z \in \Omega \) (derivable en el sentit de la \cref{def:funcio holomorfa}) si i només si la seva diferencial (pensada com a funció de \( \R^2 \)) en aquest punt és de la forma 
	\begin{equation*}
		df(z) = \begin{pmatrix}
			a & -b \\
			b & a
		\end{pmatrix}.
	\end{equation*}
\end{prop}

\begin{proof}
	Suposem que \( f \) és \( \C \)-derivable a \( z \). Una manera equivalent d'escriure-ho és que
	\begin{equation*}
		f(z + h) = f(z) + f'(z)h + o(h).
	\end{equation*}
	Posem que \( f'(z) = a + bi \) i \( h = h_1 + h_2i \) i \( x + yi \). Aleshores tenim
	\begin{align*}
		f(x + h_1 + i(y + h_2)) & = f(x + iy) + (a + bi)(h_i + h_2i) + o(h) \\
														& = f(x + iy) + (ah_1 - bh_2 + i(bh_1 + ah_2)) + o(h) \tag{\( \ast \)}
	\end{align*}
	Si escrivim \( (\ast) \) com una igualtat a \( \R^2 \) tenim
	\begin{align*}
		f(x + h_1, y + h_2) & = f(x,y) + (ah_1 - bh_2, bh_1 + ah_2) + o(\norm{h}) \\
												& = f(x,y) + \begin{pmatrix} a & -b \\ b & a \end{pmatrix} \begin{pmatrix}								h_1 \\ h_2 \end{pmatrix} + o(\norm{h}),
	\end{align*}
	on \( o(\norm{h}) \) es la notació \( o \)-petita de \( \R^2 \). El que hem escrit és cert perquè si un quocient de nombres complexos té límit zero, també el té el quocient dels seus mòduls. Per tant, per la definició de la diferencial,
	\begin{equation*}
		df(z) = \begin{pmatrix}
			a & -b \\
			b & a
		\end{pmatrix},
	\end{equation*}
	com volíem. 	

	Recíprocament, si \( df(z) \) té la forma en qüestió es demostra que aleshores \( f'(z) = a + bi \), essencialment desfent els passos anteriors.
\end{proof}

Com que el concepte de derivabilitat complexa és prou diferent del de diferenciabilitat a \( \R^2 \), s'introdueix un nom diferen. D'ara en endavant, direm que una funció és \emph{holomorfa} si és \( \C \)-derivable. I denotarem per \( \Hol{(\Omega)} \) el conjunt de funcions holomorfes sobre \( \Omega \). Finalment, direm que una funció holomorfa a \( \C \) és \emph{entera}.

La derivada complexa també satisfà les propietats que hom esperaria.
\begin{prop}[Propietats de la derivada]
	Si \( f \) i \( g \) són holomorfes a \( \Omega \subseteq \C \) aleshores
	\begin{punts}
	\item \( f + g \) és holomorfa a \( \Omega \) i \( (f + g)' = f' + g' \),
	\item \( fg \) és holomorfa a \( \Omega \) i \( (fg)' = f'g + fg' \),
	\item	si \( g \) no s'anu\l.la a \( \Omega \) aleshores \( \frac{f}{g} \) és holomorfa a \( \Omega \) i \( \frac{f'g - fg'}{g^2} \).
	\end{punts}
	Finalment si \( f \colon U \to V \) i \( g \colon V \to \C \) són holomorfes aleshores 
	\begin{punts}[resume]
	\item \( g \circ f \) és holomorfa a \( U \) i \( (f \circ g)' = (g' \circ f)f' \).
	\end{punts}
\end{prop}

\end{document}
