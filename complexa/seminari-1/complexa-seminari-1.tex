\documentclass[12pt]{article}

\usepackage[utf8]{inputenc}
\usepackage[T1]{fontenc}
\usepackage[catalan]{babel}
\usepackage{lmodern}
\usepackage{geometry}
\usepackage{hyperref}
\usepackage[dvipsnames]{xcolor}
\usepackage[bf,sf,small,pagestyles]{titlesec}
\usepackage{titling}
\usepackage[font={footnotesize, sf}, labelfont=bf]{caption} 
\usepackage{siunitx}
\usepackage{graphicx}
\usepackage{booktabs}
\usepackage{amsmath,amssymb}
\usepackage[catalan,sort]{cleveref}
\usepackage{enumitem}

\geometry{
	a4paper,
	right = 2.5cm,
	left = 2.5cm,
	bottom = 3cm,
	top = 3cm
}

\hypersetup{
	colorlinks,
	linkcolor = {red!50!blue},
	linktoc = page
}

\crefname{figure}{figura}{figures}
\crefname{table}{taula}{taules}
\numberwithin{table}{section}
\numberwithin{figure}{section}
\numberwithin{equation}{section}

\graphicspath{{./figs/}}

% Unitats
\sisetup{
	inter-unit-product = \ensuremath{ \cdot },
	allow-number-unit-breaks = true,
	detect-family = true,
	list-final-separator = { i },
	list-units = single
}

\newcommand{\Z}{\mathbb{Z}}
\renewcommand{\vec}[1]{\mathbf{#1}}
\newcommand{\N}{\mathbb{N}}
\newcommand{\R}{\mathbb{R}}
\newcommand{\C}{\mathbb{C}}
\newcommand{\Ry}{\mathit{Ry}}
\let\Im\relax
\let\Re\relax
\DeclareMathOperator{\Im}{Im}
\DeclareMathOperator{\Re}{Re}
\newcommand{\abs}[1]{\lvert #1 \rvert}
\newcommand{\ket}[1]{\vert {#1} \rangle}
\newcommand{\bra}[1]{\langle #1 \vert}
\newcommand{\braket}[2]{\langle {#1} \vert {#2} \rangle}
\newcommand{\parbreak}{
	\begin{center}
		--- $\ast$ ---
	\end{center} 
}
\makeatletter
\newcommand*{\defeq}{\mathrel{\rlap{%
    \raisebox{0.3ex}{$\m@th\cdot$}}%
  \raisebox{-0.3ex}{$\m@th\cdot$}}%
	=
}
\makeatother

\newpagestyle{pagina}{
	\headrule
	\sethead*{\sffamily {\bfseries Seminari 1:} Representació conforme i homografies}{}{\theauthor}
	\footrule
	\setfoot*{}{}{\sffamily \thepage}
}
\renewpagestyle{plain}{
	\footrule
	\setfoot*{}{}{\sffamily \thepage}
}
\pagestyle{pagina}

\title{\sffamily {\bfseries Seminari 1:} Representació conforme i homografies}
\author{\sffamily Arnau Mas --- 77633181Q}
\date{\sffamily 21 de març de 2019}

\begin{document}
\maketitle

\section*{Problema 4}
Sigui \( \Pi^+ = \{ z \in \C \mid \Im{z} > 0 \} \) el semiplà superior i \( U = \C - [-1, 1] \). Volem trobar una representació conforme de \( \Pi^+ \) sobre \( U \), és a dir, una bijecció conforme \( T \colon \Pi^+ \to U \). 

En primer lloc observem que l'aplicació 
\begin{align*}
	f \colon \Pi^+ & \longrightarrow \C - [0, \infty) \\
	z & \longmapsto z^2
\end{align*}
és una bijecció. En efecte, si \( w \in \C - [0, \infty) \) aleshores sempre hi ha \( z \in \Pi^+ \) tal que \( z^2 = w \). Si triem a \( \C - [0, \infty) \) una determinació de l'argument entre 0 i \( 2\pi \) ---de manera que \( -1 \) té argument \( \pi \), \( i \) té argument \( \frac{\pi}{2} \), etc.--- aleshores podem escriure
\begin{equation*}
	w = \abs{w} e^{i\theta}
\end{equation*}
amb \( \theta \in (0, 2\pi) \), ja que un nombre complex té argument 0 o \( 2\pi \) si i només si és real positiu. Aleshores
\begin{equation*}
	z = \sqrt{\abs{w}}e^{i\frac{\theta}{2}} \in \Pi^+,
\end{equation*}
ja que \( \frac{\theta}{2} \in (0, \pi) \). I aleshores és clar que \( z^2 = w \). Per tant \( f \) és exhaustiva.

D'altra banda, si \( z_1, z_2 \in \Pi^+ \) satisfan \( z_1^2 = z_2^2 \) aleshores o bé \( z_1 = z_2 \) o bé \( z_1 = -z_2 \). Aixó és perquè tot nombre complex té només dues arrels quadrades, les quals tenen arguments que estan separats \( \pi \), és a dir, l'una és l'oposat de l'altra. Però no pot ser que \( z_1 = -z_2 \), ja que, en general, si \( z \in \Pi^+ \) aleshores \( -z \in \Pi^+ \), perquè \( \Re{-z} = -\Re{z} \). Així doncs ha de ser \( z_1 = z_2 \) i per tant \( f \) és injectiva.    

És clar que \( f \) és holomorfa ---de fet és entera--- amb \( f'(z) = 2z \). Veiem doncs que \( f' \) només s'anu\l.la a 0, però \( 0 \in \Pi^+ \). Per tant, com que és holomorfa amb derivada no nu\l.la a tot \( \Pi^+ \), \( f \) és una aplicació conforme.  

Ara només hem de trobar una representació conforme de \( \C - [0, \infty) \) a \( U \). Considerem l'homografia \( T \colon \C_\infty \to \C_\infty \) tal que 
\begin{gather*}
	T(0) = -1, \\
	T(1) = 0, \\
	T(\infty) = 1,
\end{gather*}
sent \( \C_\infty \) l'esfera de Riemann. Aleshores \( T \) està donada per
\begin{equation*}
	T(z) = \frac{z - 1}{z + 1}.
\end{equation*}
És clar que la imatge de \( [0, \infty) \) per \( T \) és \( [-1, 1] \). Per tant la imatge de \( \C - [0, \infty) \) per \( T \) és \( \C - [-1, 1] = U \). Així doncs la composició \( T \circ f \) és una representació conforme de \( \Pi^+ \) sobre \( U \). Explícitament
\begin{equation*}
	(T \circ f)(z) = \frac{z^2 - 1}{z^2 + 1}.
\end{equation*}

\end{document}
