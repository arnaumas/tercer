\documentclass[12pt]{article}

\usepackage[utf8]{inputenc}
\usepackage[T1]{fontenc}
\usepackage[spanish]{babel}
\usepackage{cmbright}
\usepackage{lmodern}
\usepackage{geometry}
\usepackage{tikz}
\usetikzlibrary{positioning,calc,math,arrows.meta,decorations.markings,intersections}
\usepackage{hyperref}
\usepackage[bf,sf,pagestyles]{titlesec}
\usepackage{titling}
\usepackage[runin]{abstract}
\usepackage[font={footnotesize, sf}, labelfont=bf]{caption} 
\usepackage{siunitx}
\usepackage{graphicx}
\usepackage{booktabs}
\usepackage{amsmath,amssymb}
\usepackage[catalan,sort]{cleveref}
\usepackage{enumitem}

\geometry{
	a4paper,
	right = 2.5cm,
	left = 2.5cm,
	bottom = 3cm,
	top = 3cm
}

\newcommand{\sfbright}{\fontfamily{cmbr}\selectfont}
\renewcommand{\familydefault}{\rmdefault}
\renewcommand{\sfdefault}{cmbr}


\hypersetup{
	colorlinks,
	linkcolor = {red!50!blue},
	citecolor = {red!50!blue},
	linktoc = page
}

\numberwithin{table}{section}
\numberwithin{figure}{section}
\numberwithin{equation}{section}

\graphicspath{{./figs/}}

% Unitats
\sisetup{
	inter-unit-product = \ensuremath{ \, },
	allow-number-unit-breaks = true,
	math-celsius = {}^{\circ}\kern-\scriptspace C,
	detect-family = true,
	list-final-separator = { y },
	list-pair-separator = { y },
	list-units = single,
	separate-uncertainty = true
}

\newcommand{\Z}{\mathbb{Z}}
\newcommand{\N}{\mathbb{N}}
\newcommand{\R}{\mathbb{R}}
\newcommand{\Ry}{\mathit{Ry}}
\newcommand{\conv}[2]{\filldraw[fill = white!50!-red, fill opacity = 0.5, draw = -red!] (#1,0) ellipse [x radius = 0.1, y radius = #2];}
\newcommand{\data}[3]{\SI{#1 \pm #2}{#3}}
\newcommand{\unc}[2]{\ensuremath{{}\pm \SI{#1}{#2}}}
\DeclareMathOperator{\gr}{gr}
\newcommand{\abs}[1]{\left\lvert #1 \right\rvert}
\newcommand{\inn}[2]{\left\langle #1 , #2 \right\rangle}
\newcommand{\parbreak}{
	\begin{center}
		--- $\ast$ ---
	\end{center} 
}
\makeatletter
\newcommand*{\defeq}{\mathrel{\rlap{%
			\raisebox{0.3ex}{$\m@th\cdot$}}%
		\raisebox{-0.3ex}{$\m@th\cdot$}}%
	=
}
\makeatother

\newpagestyle{pagina}{
	\headrule
	\sethead*{\sffamily \bfseries Práctica 3}{}{\theauthor}
	\footrule
	\setfoot*{}{}{\sffamily \thepage}
}
\renewpagestyle{plain}{
	\footrule
	\setfoot*{}{}{\sffamily \thepage}
}
\pagestyle{pagina}

\title{\sffamily {\bfseries Práctica 2:} Óptica geométrica. Sistemas ópticos }
\author{\sffamily B2 2: Arnau Mas, Alejandro Plaza}
\date{\sffamily 6 de junio de 2019}

\begin{document}
\maketitle
\renewcommand{\abstractname}{\bfseries Resumen:}
\begin{abstract}
	Hola hola
\end{abstract}

\section{Introducción y objetivos}

\section{Telescopio astronómico}

\begin{figure}[htb]
	\centering \footnotesize \sffamily 
	\newcommand{\setcoords}{
		\coordinate (focus) at (5,0);
		\coordinate (obj) at (1,0);
		\coordinate (ocu) at (8,0);
	}
	\begin{tikzpicture}
		% Definim les posicions de les lents
		\tikzmath{
			\xoc = 8;
			\yoc = 2;
			\xob = 1;
			\yob = 2;
		}
		\setcoords

		% Eix òptic
		\draw[dashdotted, color = gray] (0,0) -- (12,0);

		\setcoords
		\tikzset{raig/.style = {
				color = red!30!white,
				decoration = {markings, mark = between positions 0.25 and 0.75 step 0.25 with {\arrow{>}}},
				postaction = decorate
			}
		};
		\draw[raig] (obj) ++ (0,-\yob) -- (ocu |- 10,1.5) -- (10,1.5) ;
		\draw[raig] (obj) ++ (0,\yob) -- (ocu |- 10,-1.5) -- (10,-1.5) ;


		% Objectiu
		\conv{\xob}{\yob}
		\draw[-|, thick, color = olive] (\xob,\yob + 0.5) -- (\xob,\yob);
		\draw[-|, thick, color = olive] (\xob,-\yob - 0.5) -- (\xob,-\yob);
		\node [above,gray] at (\xob,\yob + 0.5) {Objetivo};
		\node [below, olive, align = center] at (\xob,-\yob - 0.5) {DO\\PE};

		% Ocular
		\conv{\xoc}{\yoc}
		\draw[-|, thick, color = olive] (\xoc,\yoc + 0.5) -- (\xoc,\yoc);
		\draw[-|, thick, color = olive] (\xoc,-\yoc - 0.5) -- (\xoc,-\yoc);
		\node [above,gray] at (\xoc,\yoc + 0.5) {Ocular};
		\node [below, olive, align = center] at (\xoc,-\yoc - 0.5) {DC\\LS};

		% Distàncies
		\tikzset{dist/.style = {arrows = {[ line width = 0pt 1.5]}, color = gray, thin}};
		\begin{scope}[yshift = 20pt]
			\setcoords
			\draw[color = gray, thin, densely dashed] (obj) -- ++(0,-20pt);
			\draw[|<->|, dist] (obj) -- node [above] {\( d = \data{53}{3}{cm} \)} (ocu);
			\draw[color = gray, thin, densely dashed] (ocu) -- ++(0,-20pt);
		\end{scope}
		\begin{scope}[yshift = -20pt]
			\setcoords
			\draw[|<->|, dist] (obj) -- node [below] {\( f'_{\text{obj}} = \SI{50}{cm} \)} (focus);
			\draw[<->|, dist] (focus) -- node [below] {\( f_{\text{oc}} = \SI{5}{cm} \)} (ocu);
			\draw[<->|, dist] (ocu) -- node [below] {\( l = \SI{3.7}{cm} \)} +(2,0);
			\draw[color = gray, thin, densely dashed] (focus) -- ++(0,20pt);
		\end{scope}
		\begin{scope}[xshift = 2.2cm]
			\setcoords
			\draw[|<->|, dist] (ocu) ++ (0,1.5) -- node [below, rotate = 90] {\( \Phi_{\text{PS}} = \SI{4}{mm} \)} +(0,-3);
		\end{scope}
		\begin{scope}[xshift = -0.2cm]
			\setcoords
			\draw[|<->|, dist] (obj) ++ (0,2) -- node [above, rotate = 90] {\( \Phi_{\text{PE}} = \SI{4.3}{cm} \)} +(0,-4);
		\end{scope}
		\begin{scope}[xshift = 3.2cm]
			\setcoords
			\draw[|<->|, dist] (ocu) ++ (0,1.75) -- node [below, rotate = 90] {\( \Phi_{\text{Ojo}} = \SI{6}{mm} \)} +(0,-3.5);
		\end{scope}

		% Pupil·la de sortida
		\begin{scope}[xshift = 2cm]
			\setcoords
			\draw[-|, densely dashed, thick, opacity = 0.7, color = olive] (ocu) ++ (0,2) -- +(0,-0.5);
			\draw[-|, densely dashed, thick, opacity = 0.7, color = olive] (ocu) [yscale = -1] ++ (0,2) node[below, olive, opacity = 1] {PS} -- +(0,-0.5);
		\end{scope}

		% Finestra d'entrada
		\draw[olive, <-] (-0.5, 2) -- node [above] {LE} (0.5,2);

		% Ull
		\begin{scope}[xshift = 3cm]
			\setcoords
			\draw[-|, thick, color = olive] (ocu) ++ (0,2.25) node[above, gray] {Ojo} -- +(0,-0.5);
			\draw[-|, thick, color = olive] (ocu) [yscale = -1] ++ (0,2.25) -- +(0,-0.5);
		\end{scope}


	\end{tikzpicture}
	\caption{Diagrama del telescopio astronómico}
	\label{fig:telescopio astronómico}
\end{figure}

En la figura 

\section{Telescopio terrestre}

\begin{figure}[htb]
	\centering \footnotesize \sffamily 
	
	\begin{tikzpicture}
		\tikzmath{
			% Posicions de les lents
			\xoc = 9;
			\xin = 5.5;
			\xob = 1; 
			\y = 2;
			% Focals
			\foc = 1.5;
			\fob = 2.5; 
			\sin = 2;
		}

		\coordinate (focus1) at (\xob + \fob, 0);
		\coordinate (focus2) at (\xoc - \foc, 0);
		\coordinate (ocu) at (\xoc, 0);
		\coordinate (obj) at (\xob, 0);
		\coordinate (inv) at (\xin, 0);

		% Eix òptic
		\draw[dashdotted, color = gray, name path = eix] (0,0) -- (12,0);	

		\tikzset{raig/.style = {
				color = red!30!white,
				decoration = {markings, mark = between positions 0.25 and 0.75 step 0.25 with {\arrow{>}}},
				postaction = decorate
			}
		};

		% Objectiu
		\conv{\xob}{\y}
		\draw [-|, thick, color = olive] (\xob,\y + 0.5) -- (\xob,\y);
		\draw[-|, thick, color = olive] (\xob,-\y - 0.5) -- (\xob,-\y);
		\node [above,gray] at (\xob,\y + 0.5) {Objetivo};
		\node [below, olive, align = center] at (\xob,-\y - 0.5) {DO\\PE};

		% Ocular
		\conv{\xoc}{\y}
		\draw[-|, thick, color = olive] (\xoc,\y + 0.5) -- (\xoc,\y);
		\draw[-|, thick, color = olive] (\xoc,-\y - 0.5) -- (\xoc,-\y);
		\node [above,gray] at (\xoc,\y + 0.5) {Ocular};
		\node [below, olive, align = center] at (\xoc,-\y - 0.5) {DC\\LS};

		% Lent inversora
		\conv{\xin}{\y}
		\node [above,gray] at (\xin,\y + 0.5) {Lente inversora};

		% Distàncies
		\tikzset{dist/.style = {
				arrows = {[ line width = 0pt 1.5]}, 
				color = gray, thin
			}
		}
		% Focal ocular
		\draw [dist, |<->|] let \p{ocu} = (ocu), \p{focus2} = (focus2) in {
				[yshift = 40pt] (\p{ocu}) -- node[above, gray!50!black] {\( f_{\text{oc}} = \SI{5}{cm} \)} (\p{focus2})
			}; 
		% Focal objectiu
		\draw [dist, |<->|] let \p{obj} = (obj), \p{focus1} = (focus1) in {
			[yshift = 40pt] (\p{obj}) -- node[above, gray!50!black] {\( f'_{\text{obj}} = \SI{20}{cm} \)} (\p{focus1})
		};
		% Lent inversora
		\draw [dist, |<->|] let \p{inv} = (inv), \p{focus1} = (focus1) in {
			[yshift = 40pt] (\p{focus1}) -- node[above, gray!50!black] {\( s' = \SI{10}{cm} \)} (\p{inv})
		};
		\draw [dist, |<->|] let \p{inv} = (inv), \p{focus2} = (focus2) in {
			[yshift = 40pt] (\p{inv}) -- node[above, gray!50!black] {\( s = \SI{10}{cm} \)} (\p{focus2})
		};
		\draw [dist, dashed] (focus1) ++(0,40pt) -- ++(0,-80pt);
		\draw [dist, dashed] (focus2) ++(0,40pt) -- ++(0,-80pt);
		% Distància entre lents
		\draw [dist, <->|] let \p{ocu} = (ocu), \p{obj} = (obj) in {
			[yshift = -40pt] (\p{ocu}) -- node[below, gray!50!black] {\( d = \SI{10}{cm} \)} (\p{obj})
		};
		% Pupil·la de sortida
		\draw [dist, <->|] let \p{ocu} = (ocu) in {
			[yshift = -40pt] (\p{ocu}) -- node[below, gray!50!black] {\( l = \SI{11.7}{cm} \)} ++ (2,0) 
		};

		% Diàmetre pupil·la entrada
		\draw [dist, |<->|] let \p{obj} = (obj) in {
			[xshift = -0.2cm] (\p{obj}) ++(0,2) -- node[above, gray!50!black, rotate = 90] {\( \Phi_{\text{PE}} = \SI{4}{cm} \)} ++(0,-4) 
		};

		% Diàmetre pupil·la sortida
		\draw [dist, |<->|] let \p{ocu} = (ocu) in {
			[xshift = 0.2cm] (\p{ocu}) ++(2,1.6) -- node[below, gray!50!black, rotate = 90] {\( \Phi_{\text{PS}} = \SI{0.9}{cm} \)} ++(0,-3.2) 
		};

		% Pupil·la de sortida
		\draw[|-, densely dashed, thick, opacity = 0.7, color = olive] (ocu) ++ (2,1.6) -- +(0,0.5);
		\draw[|-, densely dashed, thick, opacity = 0.7, color = olive] (ocu) ++ (2,-1.6)-- +(0,-0.5) node[below, olive, opacity = 1] {PS};

		% Finestra d'entrada
		\draw[olive, <-] (-0.5, 2) -- node [above] {LE} (0.5,2);
	\end{tikzpicture}

	\caption{Diagrama del telescopio terrestre con lente inversora sin ojo}
	\label{fig:telescopio terrestre}
\end{figure}

\begin{figure}[htb]
	\centering \footnotesize \sffamily 
	
	\begin{tikzpicture}
		\tikzmath{
			% Posicions de les lents
			\xoc = 9;
			\xin = 5.5;
			\xob = 1; 
			\y = 2;
			% Focals
			\foc = 1.5;
			\fob = 2.5; 
			\sin = 2;
		}

		\coordinate (focus1) at (\xob + \fob, 0);
		\coordinate (focus2) at (\xoc - \foc, 0);
		\coordinate (ocu) at (\xoc, 0);
		\coordinate (obj) at (\xob, 0);
		\coordinate (inv) at (\xin, 0);

		% Eix òptic
		\draw[dashdotted, color = gray, name path = eix] (0,0) -- (12,0);	

		\tikzset{raig/.style = {
				color = red!30!white,
				decoration = {markings, mark = between positions 0.25 and 0.75 step 0.25 with {\arrow{>}}},
				postaction = decorate
			}
		};

		% Objectiu
		\conv{\xob}{\y}
		\draw [-|, thick, color = olive] (\xob,\y + 0.5) -- (\xob,\y);
		\draw[-|, thick, color = olive] (\xob,-\y - 0.5) -- (\xob,-\y);
		\node [above,gray] at (\xob,\y + 0.5) {Objetivo};
		\node [below, olive, align = center] at (\xob,-\y - 0.5) {PE};

		% Pupil·la d'entrada
		\draw [-|, densely dashed, thick, opacity = 0.7, color = olive] (\xob,\y - 0.5) -- (\xob,\y - 1);
		\draw[-|, densely dashed, thick, opacity = 0.7, color = olive] (\xob,-\y + 0.5) -- (\xob,-\y + 1);


		% Ocular
		\conv{\xoc}{\y}
		\draw[-|, thick, color = olive] (\xoc,\y + 0.5) -- (\xoc,\y);
		\draw[-|, thick, color = olive] (\xoc,-\y - 0.5) -- (\xoc,-\y);
		\node [above,gray] at (\xoc,\y + 0.5) {Ocular};
		\node [below, olive, align = center] at (\xoc,-\y - 0.5) {DC\\LS};

		% Lent inversora
		\conv{\xin}{\y}
		\node [above,gray] at (\xin,\y + 0.5) {Lente inversora};

		% Distàncies
		\tikzset{dist/.style = {
				arrows = {[ line width = 0pt 1.5]}, 
				color = gray, thin
			}
		}
		% Focal ocular
		\draw [dist, |<->|] let \p{ocu} = (ocu), \p{focus2} = (focus2) in {
				[yshift = 40pt] (\p{ocu}) -- node[above, gray!50!black] {\( f_{\text{oc}} = \SI{5}{cm} \)} (\p{focus2})
			}; 
		% Focal objectiu
		\draw [dist, |<->|] let \p{obj} = (obj), \p{focus1} = (focus1) in {
			[yshift = 40pt] (\p{obj}) -- node[above, gray!50!black] {\( f'_{\text{obj}} = \SI{20}{cm} \)} (\p{focus1})
		};
		% Lent inversora
		\draw [dist, |<->|] let \p{inv} = (inv), \p{focus1} = (focus1) in {
			[yshift = 40pt] (\p{focus1}) -- node[above, gray!50!black] {\( s' = \SI{10}{cm} \)} (\p{inv})
		};
		\draw [dist, |<->|] let \p{inv} = (inv), \p{focus2} = (focus2) in {
			[yshift = 40pt] (\p{inv}) -- node[above, gray!50!black] {\( s = \SI{10}{cm} \)} (\p{focus2})
		};
		\draw [dist, dashed] (focus1) ++(0,40pt) -- ++(0,-80pt);
		\draw [dist, dashed] (focus2) ++(0,40pt) -- ++(0,-80pt);
		% Distància entre lents
		\draw [dist, <->|] let \p{ocu} = (ocu), \p{obj} = (obj) in {
			[yshift = -40pt] (\p{ocu}) -- node[below, gray!50!black] {\( d = \SI{10}{cm} \)} (\p{obj})
		};
		% Pupil·la de sortida
		\draw [dist, <->|] let \p{ocu} = (ocu) in {
			[yshift = -40pt] (\p{ocu}) -- node[below, gray!50!black] {\( l = \SI{11.7}{cm} \)} ++ (2,0) 
		};
		
		% Diàmetre ull
		\draw [dist, |<->|] let \p{ocu} = (ocu) in {
			[xshift = 0.2cm] (\p{ocu}) ++(2,0.8) -- node[below, gray!50!black, rotate = 90] {\( \Phi_{\text{Ojo}} = \SI{6}{mm} \)} ++(0,-1.6) 
		};

		% Ull 
		\draw[|-, thick, color = olive] (ocu) ++ (2,0.8) -- +(0,0.5);
		\draw[|-, thick, color = olive] (ocu) ++ (2,-0.8)-- +(0,-0.5) node[below = 20pt, olive, opacity = 1, align = center] {DO\\PS};
		\node [above,gray] at (\xoc + 2,\y + 0.5) {Ojo};

		% Finestra d'entrada
		\draw[olive, <-] (-0.5, 2) -- node [above] {LE} (0.5,2);
	\end{tikzpicture}

	\caption{Diagrama del telescopio terrestre con lente inversora con ojo}
	\label{fig:telescopio terrestre ojo}
\end{figure}

\section{Telescopio de Galileo}

\begin{figure}[htb]
	\centering \footnotesize \sffamily 
	
	\begin{tikzpicture}
		\tikzmath{
			% Posicions de les lents
			\xoc = 9;
			\xob = 1; 
			\y = 2;
			% Focals
			\foc = 2;
			\fob = 10; 
			% Posició de la pupil·la de sortida
			\xso = 1/(1/\foc + 1/\fob);
			\yso = \xso*\y / (\xoc - \xob); 
		}

		\coordinate (focus) at (\xoc + \foc, 0);
		\coordinate (ocu) at (\xoc, 0);
		\coordinate (obj) at (\xob, 0);

		% Eix òptic
		\draw[dashdotted, color = gray, name path = eix] (0,0) -- (12,0);


		\tikzset{raig/.style = {
				color = red!30!white,
				decoration = {markings, mark = between positions 0.25 and 0.75 step 0.25 with {\arrow{>}}},
				postaction = decorate
			}
		};

	\draw[raig, dashed, name path = focal1] (\xoc - \foc, 0) -- (\xoc, \y);
	\draw[raig, dashed, name path = focal2] (\xoc - \foc, 0) -- (\xoc, -\y);
	\path[name path = central1] (\xob, \y) -- (\xoc, 0);
	\path[name path = central2] (\xob, -\y) -- (\xoc, 0);
	\path[name intersections = {of = central1 and focal1, by = PS1}];
	\path[name intersections = {of = central2 and focal2, by = PS2}];
	\draw[raig] (\xob, \y) -- (\xoc, \y) -- ($ (PS1)!1.7!(\xoc, \y) $);
	\draw[raig] (\xob, -\y) -- (\xoc, -\y) -- ($ (PS2)!1.7!(\xoc, -\y) $);

		% Objectiu
		\conv{\xob}{\y}
	\draw [-|, thick, color = olive] (\xob,\y + 0.5) -- (\xob,\y);
	\draw[-|, thick, color = olive] (\xob,-\y - 0.5) -- (\xob,-\y);
	\node [above,gray] at (\xob,\y + 0.5) {Objetivo};
	\node [below, olive, align = center] at (\xob,-\y - 0.5) {DO\\PE};

	% Ocular
	\filldraw[fill = white!50!-red, fill opacity = 0.5, draw = -red!] (\xoc + 0.15,2) arc [start angle = 90, end angle = 270, x radius = 0.1, y radius = \y ] -- ++(-0.3,0) arc [start angle = 270, end angle = 90, x radius = -0.1, y radius = \y ] -- cycle;
	\draw[-|, thick, color = olive] (\xoc,\y + 0.5) -- (\xoc,\y);
	\draw[-|, thick, color = olive] (\xoc,-\y - 0.5) -- (\xoc,-\y);
	\node [above,gray] at (\xoc,\y + 0.5) {Ocular};
	\node [below, olive, align = center] at (\xoc,-\y - 0.5) {DC\\LS};

	% Pupil·la de sortida
	\draw[|-, densely dashed, thick, opacity = 0.7, color = olive] (PS1) -- +(0,0.5);
	\draw[|-, densely dashed, thick, opacity = 0.7, color = olive] (PS2) -- +(0,-0.5) node[below, olive, opacity = 1] {PS};

	% Distàncies
	\tikzset{dist/.style = {
			arrows = {[ line width = 0pt 1.5]}, 
			color = gray, thin
	}};
	% Focal ocular
	\draw [dist, |<->|] let \p{ocu} = (ocu), \p{focus} = (focus) in {
		[yshift = 40pt] (\p{ocu}) -- node[above, gray!50!black] {\( f_{\text{oc}} = \SI{10}{cm} \)} (\p{focus})
	};
	% Focal objectiu
	\draw [dist, <->|] let \p{ocu} = (ocu), \p{obj} = (obj) in {
		[yshift = 40pt] (\p{ocu}) -- node[above, gray!50!black] {\( d = \SI{10}{cm} \)} (\p{obj})
	};
	\draw [dist, dashed] (focus) ++(0,40pt) -- ++(0,-80pt);
	% Distància entre lents
	\draw [dist, |<->|] let \p{obj} = (obj), \p{focus} = (focus) in {
		[yshift = -40pt] (\p{obj}) -- node[below, gray!50!black] {\( f'_{\text{obj}} = \SI{50}{cm} \)} (\p{focus})
	};
	% Diàmetre pupil·la entrada
	\draw [dist, |<->|] let \p{obj} = (obj) in {
		[xshift = -0.2cm] (\p{obj}) ++(0,2) -- node[above, gray!50!black, rotate = 90] {\( \Phi_{\text{PE}} = \SI{4.5}{cm} \)} ++(0,-4) 
	};


	% Finestra d'entrada
	\draw[olive, <-] (-0.5, 2) -- node [above] {LE} (0.5,2);
\end{tikzpicture}

	\caption{Diagrama del telescopio de Galileo sin ojo}
	\label{fig:telescopio galileo}
\end{figure}

\begin{figure}[htb]
	\centering \footnotesize \sffamily 
	
	\begin{tikzpicture}
		\tikzmath{
			% Posicions de les lents
			\xoc = 9;
			\xob = 1; 
			\y = 2;
			% Focals
			\foc = 2;
			\fob = 10; 
			% Posició de la pupil·la de sortida
			\xso = 1/(1/\foc + 1/\fob);
			\yso = \xso*\y / (\xoc - \xob); 
		}

		\coordinate (focus) at (\xoc + \foc, 0);
		\coordinate (ocu) at (\xoc, 0);
		\coordinate (obj) at (\xob, 0);

		% Eix òptic
		\draw[dashdotted, color = gray, name path = eix] (0,0) -- (12,0);


		\tikzset{raig/.style = {
				color = red!30!white,
				decoration = {markings, mark = between positions 0.25 and 0.75 step 0.25 with {\arrow{>}}},
				postaction = decorate
			}
		};

	\draw[raig, dashed, name path = focal1] (\xoc - \foc, 0) -- (\xoc, \y);
	\draw[raig, dashed, name path = focal2] (\xoc - \foc, 0) -- (\xoc, -\y);
	\path[name path = central1] (\xob, \y) -- (\xoc, 0);
	\path[name path = central2] (\xob, -\y) -- (\xoc, 0);
	\path[name intersections = {of = central1 and focal1, by = PS1}];
	\path[name intersections = {of = central2 and focal2, by = PS2}];
	\draw[raig] (\xob, \y) -- (\xoc, \y) -- ($ (PS1)!1.7!(\xoc, \y) $);
	\draw[raig] (\xob, -\y) -- (\xoc, -\y) -- ($ (PS2)!1.7!(\xoc, -\y) $);

		% Objectiu
		\conv{\xob}{\y};
	\draw [-|, thick, color = olive] (\xob,\y + 0.5) -- (\xob,\y);
	\draw[-|, thick, color = olive] (\xob,-\y - 0.5) -- (\xob,-\y);
	\node [above,gray] at (\xob,\y + 0.5) {Objetivo};
	\node [below, olive, align = center] at (\xob,-\y - 0.5) {DC\\LE};

	% Ocular
	\filldraw[fill = white!50!-red, fill opacity = 0.5, draw = -red!] (\xoc + 0.15,2) arc [start angle = 90, end angle = 270, x radius = 0.1, y radius = \y ] -- ++(-0.3,0) arc [start angle = 270, end angle = 90, x radius = -0.1, y radius = \y ] -- cycle;
	\draw[-|, thick, color = olive] (\xoc,\y + 0.5) -- (\xoc,\y);
	\draw[-|, thick, color = olive] (\xoc,-\y - 0.5) -- (\xoc,-\y);
	\node [above,gray] at (\xoc,\y + 0.5) {Ocular};

	% Pupil·la de sortida
	\draw[|-, densely dashed, thick, opacity = 0.7, color = olive] (PS1) -- +(0,0.5);
	\draw[|-, densely dashed, thick, opacity = 0.7, color = olive] (PS2) -- +(0,-0.5) node[below, olive, opacity = 1] {LS};
	% Ull
	\draw[|-, thick, color = olive] (\xoc + 0.5, 1) -- +(0,0.5);
	\draw[|-, thick, color = olive] (\xoc + 0.5, -1) -- ++(0,-0.5) node[below, gray] {Ojo};
	\node[below, olive, align = center] at (\xoc + 0.5, -2) {DO\\PS};

	% Distàncies
	\tikzset{dist/.style = {
			arrows = {[ line width = 0pt 1.5]}, 
			color = gray, thin
	}};
	% Focal ocular
	\draw [dist, |<->|] let \p{ocu} = (ocu), \p{focus} = (focus) in {
		[yshift = 40pt] (\p{ocu}) -- node[above, gray!50!black] {\( f_{\text{oc}} = \SI{10}{cm} \)} (\p{focus})
	};
	% Focal objectiu
	\draw [dist, <->|] let \p{ocu} = (ocu), \p{obj} = (obj) in {
		[yshift = 40pt] (\p{ocu}) -- node[above, gray!50!black] {\( d = \SI{10}{cm} \)} (\p{obj})
	};
	\draw [dist, dashed] (focus) ++(0,40pt) -- ++(0,-80pt);
	% Distància entre lents
	\draw [dist, |<->|] let \p{obj} = (obj), \p{focus} = (focus) in {
		[yshift = -40pt] (\p{obj}) -- node[below, gray!50!black] {\( f'_{\text{obj}} = \SI{50}{cm} \)} (\p{focus})
	};
	% Diàmetre pupil·la entrada
	\draw [dist, |<->|] let \p{obj} = (obj) in {
		[xshift = -0.2cm] (\p{obj}) ++(0,2) -- node[above, gray!50!black, rotate = 90] {\( \Phi_{\text{PE}} = \SI{4.5}{cm} \)} ++(0,-4) 
	};
	% Diàmetre de l'ull
	\draw [dist, |<->|] let \p{ocu} = (ocu) in {
		[xshift = 0.7cm] (\p{ocu}) ++(0,1) -- node[below, gray!50!black, rotate = 90] {\( \Phi_{\text{Ojo}} = \SI{6}{mm} \)} ++(0,-2) 
	};

	% Finestra d'entrada
	\draw[olive, <-] (-0.5, 2) -- node [above] {PE} (0.5,2);
\end{tikzpicture}

	\caption{Diagrama del telescopio de Galileo con ojo}
	\label{fig:telescopio galileo ojo}
\end{figure}

\end{document}
