\documentclass[12pt]{article}

\usepackage[utf8]{inputenc}
\usepackage[T1]{fontenc}
\usepackage[spanish]{babel}
\usepackage{cmbright}
\usepackage{lmodern}
\usepackage{geometry}
\usepackage{pgfplots}
\usepackage{tikz}
\usetikzlibrary{positioning,calc,math,arrows.meta,decorations.markings,intersections}
\usepackage{hyperref}
\usepackage[bf,sf,pagestyles]{titlesec}
\usepackage{titling}
\usepackage[runin]{abstract}
\usepackage[font={footnotesize, sf}, labelfont=bf]{caption} 
\usepackage{siunitx}
\usepackage{graphicx}
\usepackage{booktabs}
\usepackage{amsmath,amssymb}
\usepackage[spanish,sort]{cleveref}
\usepackage{enumitem}

\geometry{
	a4paper,
	right = 2.5cm,
	left = 2.5cm,
	bottom = 3cm,
	top = 3cm
}

\newcommand{\sfbright}{\fontfamily{cmbr}\selectfont}
\renewcommand{\familydefault}{\rmdefault}
\renewcommand{\sfdefault}{cmbr}
\renewcommand{\arraystretch}{1.4}

\hypersetup{
	colorlinks,
	linkcolor = {red!50!blue},
	citecolor = {red!50!blue},
	linktoc = page
}

\numberwithin{table}{section}
\numberwithin{figure}{section}
\numberwithin{equation}{section}

\graphicspath{{./figs/}}

% Unitats
\sisetup{
	inter-unit-product = \ensuremath{ \, },
	allow-number-unit-breaks = true,
	math-celsius = {}^{\circ}\kern-\scriptspace C,
	detect-family = true,
	mode = text,
	list-final-separator = { y },
	list-pair-separator = { y },
	list-units = single,
	separate-uncertainty = true
}

\newcommand{\Z}{\mathbb{Z}}
\newcommand{\N}{\mathbb{N}}
\newcommand{\R}{\mathbb{R}}
\newcommand{\Ry}{\mathit{Ry}}
\newcommand{\conv}[2]{\filldraw[fill = white!50!-red, fill opacity = 0.5, draw = -red!] (#1,0) ellipse [x radius = 0.1, y radius = #2];}
\newcommand{\data}[3]{\SI{#1 \pm #2}{#3}}
\newcommand{\unc}[2]{\ensuremath{{}\pm \SI{#1}{#2}}}
\DeclareMathOperator{\gr}{gr}
\newcommand{\abs}[1]{\left\lvert #1 \right\rvert}
\newcommand{\inn}[2]{\left\langle #1 , #2 \right\rangle}
\newcommand{\parbreak}{
	\begin{center}
		--- $\ast$ ---
	\end{center} 
}
\makeatletter
\newcommand*{\defeq}{\mathrel{\rlap{%
			\raisebox{0.3ex}{$\m@th\cdot$}}%
		\raisebox{-0.3ex}{$\m@th\cdot$}}%
	=
}
\makeatother

\newpagestyle{pagina}{
	\headrule
	\sethead*{\sffamily \bfseries Práctica 4}{}{\theauthor}
	\footrule
	\setfoot*{}{}{\sffamily \thepage}
}
\renewpagestyle{plain}{
	\footrule
	\setfoot*{}{}{\sffamily \thepage}
}
\pagestyle{pagina}

\title{\sffamily {\bfseries Práctica 4:} Espectros ópticos. Determinación de longitudes de onda con un espectrómetro de prisma}
\author{\sffamily B2 2: Arnau Mas, Alejandro Plaza}
\date{\sffamily 14 de marzo de 2019}

\begin{document}
\maketitle
\renewcommand{\abstractname}{\sffamily \bfseries Resumen:}
\begin{abstract}
	En esta práctica se hará uso de un espectrómetro de prisma para la observación de los espectros ópticos de distintas fuentes de luz. Primeramente se realizan una serie de observaciones cualitativas para comprender el funcionamiento de la refracción y de la dispersión cromática. Seguidamente se medirán los ángulos de desviación mínima de las distintas franjas de un espectro conocido, el del mercurio, con el fin de calibrar el espectrómetro mediante la fórmula de Hartmann. Hecho esto se medirán las longitudes de onda de un segundo espectro, el del cadmio y se contrastan los resultados con los valores teóricos. Finakmente se realizan observaciones de distintas fuentes de luz.
\end{abstract}
\hrule

\section{Objetivos}
Los objetivos de esta práctica son el uso de un espectrómetro de prisma para observar los espectros de distintas fuentes de luz gracias a la acción refractora del prisma. También se usará la fórmula de Hartmann para calibrar el espectrómetro con un espectro conocido, el del mercurio, para luego medir las longitudes de onda de un espectro desconocido, el  del cadmio.

\section{Puesta a punto del sistema}
El espectrómetro consta de una fuente de luz, seguida de una rendija que a su vez está seguida de una lente colimadora. La función de la lente colimadora es hacer que la luz que se emite desde la rendija salga con frentes de onda planos, es decir, que los rayos sean paralelos. Para conseguir esto hace falta que la rendija esté en el foco de la lente colimadora. 

Para poder hacer las observaciones se dispone de un telescopio. Para ello es necesario que el telescopio funcione como tal, es decir, es necesario que los focos del objetivo y el ocular coincidan. Esta configuración se muestra en la \cref{fig:esquema}.

\begin{figure}[htb]
	\centering \small \sffamily
	\begin{tikzpicture}
		% Fuente
		\shade[inner color = orange!50!yellow, outer color = white] (0,0) circle (.5);
		\node[above] at (0,0.5) {Fuente};

		% Rendija
		\draw[-|, thick, color = brown] (0.75,0.5) -- +(0,-0.45);
		\draw[-|, thick, color = brown] [yscale = -1] (0.75,0.5) -- +(0,-0.45);
		\node[below] at (0.75,-0.5) {Rendija};

		% Lente colimadora
		\filldraw[fill = white!50!-red, fill opacity = 0.5, draw = -red!] (2,0) ellipse [x radius = 0.1, y radius = 0.75];
		\node[above] at (2,1) {Lente colimadora};

		% Centro del prisma
		\coordinate (prisma) at (5,-0.3);
		% Prisma
		\begin{scope}[shift = {(prisma)}, rotate = -10]
			\coordinate (A) at (90:1.5);
			\coordinate (B) at (210:1.5);
			\coordinate (C) at (330:1.5);

			\filldraw[fill = white!50!-red, fill opacity = 0.5, draw = -red!, name path = prisma] (A) -- (B) -- (C) -- cycle;
		\end{scope}
		\node[above = 1.7] at (prisma) {Prisma};

		\tikzset{raig/.style = {
				color = orange!50!yellow,
				decoration = {markings, mark = between positions 0.25 and 0.75 step 0.25 with {\arrow{>}}},
				postaction = decorate
		}}

		\path[name path = raig11] (0,0) -- (5,0);
		\path[name intersections = {of = prisma and raig11, by = entrada1}];  
		\path[name path = raig12] (entrada1) -- + (-7:2);
		\path[name intersections = {of = prisma and raig12, by = sortida1}];  

		% Telescopio
		\begin{scope}[shift = {(sortida1)}, rotate = -15]
			\filldraw[fill = white!50!-red, fill opacity = 0.5, draw = -red!] (4,0) ellipse [x radius = 0.1, y radius = 0.75];
\node[above] at (4,1) {Objetivo};
			\filldraw[fill = white!50!-red, fill opacity = 0.5, draw = -red!] (6,0) ellipse [x radius = 0.1, y radius = 0.75];
\node[above] at (6,1) {Ocular};
		\end{scope}

		\draw[raig, name path = raig1] (0,0) -- (entrada1) -- (sortida1) [shift = {(sortida1)}, rotate = -15] -- (4,0) coordinate (obj) -- +(3,0);

		\path[name path = raig21] (2, 0.5) -- (2,0.5 -| prisma);
		\path[name intersections = {of = prisma and raig21, by = entrada2}];  
		\path[name path = raig22]  [rotate around = {-7:(entrada2)}] (entrada2) ++ (0.1,0) -- +(3,0);
		\path[name intersections = {of = prisma and raig22, by = sortida2}];  

		\path[name path = raig31] (2, -0.5) -- (2,-0.5 -| prisma);
		\path[name intersections = {of = prisma and raig31, by = entrada3}];  
		\path[name path = raig32]  [rotate around = {-7:(entrada3)}] (entrada3) ++ (0.1,0) -- +(3,0);
		\path[name intersections = {of = prisma and raig32, by = sortida3}];  

		\draw[raig, name path = raig2] (0.75,0) -- (2,0.5) -- (entrada2) -- (sortida2) [rotate around = {-15:(sortida2)}, shift = {(obj)}] -- (0,0.5) -- (2,-0.5) -- +(1,0);
		\draw[raig, name path = raig3] (0.75,0) -- (2,-0.5) -- (entrada3) -- (sortida3) [rotate around = {-15:(sortida3)}, shift = {(obj)}] -- (0,-0.5) -- (2,0.5) -- +(1,0);

		\begin{scope}[shift = {(sortida1)}, rotate = -15]
			\draw (8,0) arc [start angle = 270, end angle = 180, radius = 0.3];
			\draw (8,0) arc [start angle = 90, end angle = 180, radius = 0.3];
			\draw (8,0) ++ (150:0.3) arc [start angle = 150, end angle = 210, radius = 0.3];
		\end{scope}
	\end{tikzpicture}
	\caption{Esquema de la disposición del espectrómetro. El sistema óptico objetivo-ocular tiene un efecto magnificador que no se muestra en la figura.}	
	\label{fig:esquema}
\end{figure}

\section{Observación con el espectrómetro}

\begin{figure}[htb]
	\centering \small \sffamily
	\begin{tikzpicture}
		% Fuente
		\shade[inner color = orange!50!yellow, outer color = white] (0,0) circle (.5);
		\node[above] at (0,0.5) {Fuente};

		% Rendija
		\draw[-|, thick, color = brown] (0.75,0.5) -- +(0,-0.45);
		\draw[-|, thick, color = brown] [yscale = -1] (0.75,0.5) -- +(0,-0.45);
		\node[below] at (0.75,-0.5) {Rendija};

		% Lente colimadora
		\filldraw[fill = white!50!-red, fill opacity = 0.5, draw = -red!] (2,0) ellipse [x radius = 0.1, y radius = 0.75];
		\node[above] at (2,1) {Lente colimadora};

		% Centro del prisma
		\coordinate (prisma) at (5,-0.3);
		% Prisma
		\begin{scope}[shift = {(prisma)}, rotate = -10]
			\coordinate (A) at (90:1.5);
			\coordinate (B) at (210:1.5);
			\coordinate (C) at (330:1.5);

			\filldraw[fill = white!50!-red, fill opacity = 0.5, draw = -red!, name path = prisma] (A) -- (B) -- (C) -- cycle;
		\end{scope}
		\node[above = 1.7] at (prisma) {Prisma};

		\tikzset{raig/.style = {
				color = orange!50!yellow,
				decoration = {markings, mark = between positions 0.25 and 0.75 step 0.25 with {\arrow{>}}},
				postaction = decorate
		}}

		\path[name path = raig11] (0,0) -- (5,0);
		\path[name intersections = {of = prisma and raig11, by = entrada1}];  
		\path[name path = raig21] (entrada1) -- + (-7:2);
		\path[name intersections = {of = prisma and raig21, by = sortida1}];  
		\path[name path = raig22] (entrada1) -- + (-3:2);
		\path[name intersections = {of = prisma and raig22, by = sortida2}];  
		\path[name path = raig23] (entrada1) -- + (-11:2);
		\path[name intersections = {of = prisma and raig23, by = sortida3}];  


		\draw[raig, name path = raig1] (0,0) -- (entrada1);
		\draw[raig, color = green] (entrada1) -- (sortida1) [shift = {(sortida1)}, rotate = -15] -- (4,0) coordinate (obj) -- +(3,0);
		\draw[raig, color = red] (entrada1) -- (sortida2) [shift = {(sortida2)}, rotate = -11] -- (7,0);
		\draw[raig, color = blue] (entrada1) -- (sortida3) [shift = {(sortida3)}, rotate = -19] -- (7,0);

		\draw[raig, color = green, dashed] [shift = {(sortida1)}, rotate = -15] (-4,0) -- (sortida1);
		\draw[raig, color = red, dashed] [shift = {(sortida2)}, rotate = -11] (-4,0) -- (sortida2);
		\draw[raig, color = blue, dashed] [shift = {(sortida3)}, rotate = -19] (-4,0) -- (sortida3);

		\begin{scope}[shift = {(sortida1)}, rotate = -15]
			\draw (8,0) arc [start angle = 270, end angle = 180, radius = 0.3];
			\draw (8,0) arc [start angle = 90, end angle = 180, radius = 0.3];
			\draw (8,0) ++ (150:0.3) arc [start angle = 150, end angle = 210, radius = 0.3];
		\end{scope}

	\end{tikzpicture}
	\caption{Recorrido de los rayos a través del prisma sin el telescopio. Los rayos discontínuos representan el camino que interpreta el ojo.}	
	\label{fig:sin telescopio}
\end{figure}

\begin{figure}[htb]
	\centering \small \sffamily
	\begin{tikzpicture}
		% Fuente
		\shade[inner color = orange!50!yellow, outer color = white] (0,0) circle (.5);
		\node[above] at (0,0.5) {Fuente};

		% Rendija
		\draw[-|, thick, color = brown] (0.75,0.5) -- +(0,-0.45);
		\draw[-|, thick, color = brown] [yscale = -1] (0.75,0.5) -- +(0,-0.45);
		\node[below] at (0.75,-0.5) {Rendija};

		% Lente colimadora
		\filldraw[fill = white!50!-red, fill opacity = 0.5, draw = -red!] (2,0) ellipse [x radius = 0.1, y radius = 0.75];
		\node[above] at (2,1) {Lente colimadora};

		% Centro del prisma
		\coordinate (prisma) at (5,-0.3);
		% Prisma
		\begin{scope}[shift = {(prisma)}, rotate = -10]
			\coordinate (A) at (90:1.5);
			\coordinate (B) at (210:1.5);
			\coordinate (C) at (330:1.5);

			\filldraw[fill = white!50!-red, fill opacity = 0.5, draw = -red!, name path = prisma] (A) -- (B) -- (C) -- cycle;
		\end{scope}
		\node[above = 1.7] at (prisma) {Prisma};

		\tikzset{raig/.style = {
				color = orange!50!yellow,
				decoration = {markings, mark = between positions 0.25 and 0.75 step 0.25 with {\arrow{>}}},
				postaction = decorate
		}}

		\path[name path = raig11] (0,0) -- (5,0);
		\path[name intersections = {of = prisma and raig11, by = entrada1}];  
		\path[name path = raig21] (entrada1) -- + (-7:2);
		\path[name intersections = {of = prisma and raig21, by = sortida1}];  
		\path[name path = raig22] (entrada1) -- + (-3:2);
		\path[name intersections = {of = prisma and raig22, by = sortida2}];  
		\path[name path = raig23] (entrada1) -- + (-5:2);
		\path[name intersections = {of = prisma and raig23, by = sortida3}];  

		% Telescopio
		\begin{scope}[shift = {(sortida1)}, rotate = -15]
			\filldraw[fill = white!50!-red, fill opacity = 0.5, draw = -red!] (4,0) ellipse [x radius = 0.1, y radius = 0.75];
			\node[above] at (4,1) {Objetivo};
			\filldraw[fill = white!50!-red, fill opacity = 0.5, draw = -red!] (6,0) ellipse [x radius = 0.1, y radius = 0.75];
			\node[above] at (6,1) {Ocular};
		\end{scope}


		\draw[raig, name path = raig1] (0,0) -- (entrada1);
		\draw[raig, color = blue] (entrada1) -- (sortida1) [shift = {(sortida1)}, rotate = -15] -- (4,0) coordinate (obj) -- ++(2,0) coordinate (ocu) -- ++(1,0);

		\path[name path = obj] [rotate around = {(-15:(sortida1))}] (obj) ++ (0,0.75) -- ++(0,-1.5);
		\path[name path = ocu] [rotate around = {(-15:(sortida1))}] (ocu) ++ (0,0.75) -- ++(0,-1.5);

		\path[name path = raig32] (sortida2) [shift = {(sortida2)}, rotate = -11] -- (7,0);
		\path[name intersections = {of = obj and raig32, by = obj2}];  
		\path[name path = raig33] (sortida3) [shift = {(sortida3)}, rotate = -13] -- (7,0);
		\path[name intersections = {of = obj and raig33, by = obj3}];  

		\path[name path = raig42] (obj2) [shift = {(sortida1)}, rotate = -15] -- ($(obj2)!3!(4.8,0)$);
		\path[name intersections = {of = ocu and raig42, by = ocu2}];  
		\path[name path = raig43] (obj3) [shift = {(sortida1)}, rotate = -15] -- ($(obj3)!3!(4.8,0)$);
		\path[name intersections = {of = ocu and raig43, by = ocu3}];  

		\draw[raig, color = red] (entrada1) -- (sortida2) -- (obj2) -- (ocu2) [shift = {(ocu2)}, rotate = -19] -- (1,0);
		\draw[raig, color = green] (entrada1) -- (sortida3) -- (obj3) -- (ocu3) [shift = {(ocu3)}, rotate = -17] -- (1,0);

		\draw[raig, color = blue, dashed] [shift = {(ocu)}, rotate = -15] (-10,0) -- (ocu);
		\draw[raig, color = red, dashed] [shift = {(ocu2)}, rotate = -19] (-10,0) -- (ocu2);
		\draw[raig, color = green, dashed] [shift = {(ocu3)}, rotate = -17] (-10,0) -- (ocu3);

		\begin{scope}[shift = {(sortida1)}, rotate = -15]
			\draw (8,0) arc [start angle = 270, end angle = 180, radius = 0.3];
			\draw (8,0) arc [start angle = 90, end angle = 180, radius = 0.3];
			\draw (8,0) ++ (150:0.3) arc [start angle = 150, end angle = 210, radius = 0.3];
		\end{scope}

	\end{tikzpicture}
	\caption{Recorrido de los rayos a través del prisma con el telescopio. Los rayos discontínuos representan el camino que interpreta el ojo.}	
	\label{fig:con telescopio}
\end{figure}

Una vez enfocado todo el sistema óptico podemos pasar a la observación. Con el telescopio observamos que las franjas correspondientes a las longitudes de onda más cortas (azul, morado) se encuentran en el extremo de máxima desviación\footnote{Dada la configuración del experimento en el laboratorio de hecho este extremo se encontraba a la derecha mientras que en el esquema corresponde al lado izquierdo por como están dispuestos los elementos en él} mientras que las que tienen longitudes de onda largas (rojo) se encuentran en el otro extremo. Esto es el resultado que esperaríamos puesto que las longitudes de onda más cortas se refractan más. Aún así, si observamos el trazado de rayos en la \cref{fig:con telescopio} vemos que de hecho los rayos de longitudes más cortas llegan por la derecha, y no por la izquierda, ya que el telescopio los está invirtiendo. Lo que ocurre es que el cerebro reconstruye la posición de la fuente de los rayos prolongándolos de forma rectílinea, tal y como muestran las líneas discontinuas de la figura. Vemos, pues, que interpreta que la posición de la fuente es en el orden que observamos. Por contra, si observamos sin el telescopio vemos el patrón de rayas invertido, lo que tiene la misma explicación: el ojo invierte el orden al reconstruir los caminos pero ahora no hay el telescopio para deshacer esta inversión, como vemos en la \cref{fig:sin telescopio}.   

La lámpara de mercuerio emite luz extensa, pues su superficie de emisión no es despreciable. La función de la rendija, pues, es la de reducir la extensión de la fuente. Ahora bien, solo lo hace en el plano horizontal, pues si lo hiciese en ambas direcciones solo observaríamos puntos y no rayas, lo que dificultaría la medición de la desviación de las franjas. Podemos considerar entonces que la fuente de luz es puntual en el plano horizontal, lo que nos permite realizar los diagramas anteriores. En particular podríamos sustituir la lente esférica del montaje por una cílindrica orientada correctamente (la lente debe refractar en el plano horizontal). 

Cuando rotamos el prisma hacemos variar el ángulo de incidencia de los rayos con el mismo. Con el telescopio podemos observar como el ángulo de desviación va disminuyendo de forma que las franjas se desplazan a la derecha, hasta que empieza a aumentar otra vez y las franjas se desplazan hacia la otra dirección. Observamos, pues, un ángulo de desviación mínima. Las franjas de distintos colores alcanzan su mínima desviación en distintos ángulos, que es lo que medimos con el goniómetro.

\section{Calibración del espectrómetro}
\begin{figure}[htb]
	\small \centering \sffamily
	%% Creator: Matplotlib, PGF backend
%%
%% To include the figure in your LaTeX document, write
%%   \input{<filename>.pgf}
%%
%% Make sure the required packages are loaded in your preamble
%%   \usepackage{pgf}
%%
%% Figures using additional raster images can only be included by \input if
%% they are in the same directory as the main LaTeX file. For loading figures
%% from other directories you can use the `import` package
%%   \usepackage{import}
%% and then include the figures with
%%   \import{<path to file>}{<filename>.pgf}
%%
%% Matplotlib used the following preamble
%%   \usepackage{siunitx}
%%   \usepackage{fontspec}
%%
\begingroup%
\makeatletter%
\begin{pgfpicture}%
\pgfpathrectangle{\pgfpointorigin}{\pgfqpoint{5.000000in}{3.500000in}}%
\pgfusepath{use as bounding box, clip}%
\begin{pgfscope}%
\pgfsetbuttcap%
\pgfsetmiterjoin%
\definecolor{currentfill}{rgb}{1.000000,1.000000,1.000000}%
\pgfsetfillcolor{currentfill}%
\pgfsetlinewidth{0.000000pt}%
\definecolor{currentstroke}{rgb}{1.000000,1.000000,1.000000}%
\pgfsetstrokecolor{currentstroke}%
\pgfsetdash{}{0pt}%
\pgfpathmoveto{\pgfqpoint{0.000000in}{0.000000in}}%
\pgfpathlineto{\pgfqpoint{5.000000in}{0.000000in}}%
\pgfpathlineto{\pgfqpoint{5.000000in}{3.500000in}}%
\pgfpathlineto{\pgfqpoint{0.000000in}{3.500000in}}%
\pgfpathclose%
\pgfusepath{fill}%
\end{pgfscope}%
\begin{pgfscope}%
\pgfsetbuttcap%
\pgfsetmiterjoin%
\definecolor{currentfill}{rgb}{1.000000,1.000000,1.000000}%
\pgfsetfillcolor{currentfill}%
\pgfsetlinewidth{0.000000pt}%
\definecolor{currentstroke}{rgb}{0.000000,0.000000,0.000000}%
\pgfsetstrokecolor{currentstroke}%
\pgfsetstrokeopacity{0.000000}%
\pgfsetdash{}{0pt}%
\pgfpathmoveto{\pgfqpoint{0.625000in}{0.437500in}}%
\pgfpathlineto{\pgfqpoint{4.500000in}{0.437500in}}%
\pgfpathlineto{\pgfqpoint{4.500000in}{3.080000in}}%
\pgfpathlineto{\pgfqpoint{0.625000in}{3.080000in}}%
\pgfpathclose%
\pgfusepath{fill}%
\end{pgfscope}%
\begin{pgfscope}%
\pgfpathrectangle{\pgfqpoint{0.625000in}{0.437500in}}{\pgfqpoint{3.875000in}{2.642500in}}%
\pgfusepath{clip}%
\pgfsetrectcap%
\pgfsetroundjoin%
\pgfsetlinewidth{0.803000pt}%
\definecolor{currentstroke}{rgb}{0.690196,0.690196,0.690196}%
\pgfsetstrokecolor{currentstroke}%
\pgfsetdash{}{0pt}%
\pgfpathmoveto{\pgfqpoint{1.035381in}{0.437500in}}%
\pgfpathlineto{\pgfqpoint{1.035381in}{3.080000in}}%
\pgfusepath{stroke}%
\end{pgfscope}%
\begin{pgfscope}%
\pgfsetbuttcap%
\pgfsetroundjoin%
\definecolor{currentfill}{rgb}{0.000000,0.000000,0.000000}%
\pgfsetfillcolor{currentfill}%
\pgfsetlinewidth{0.803000pt}%
\definecolor{currentstroke}{rgb}{0.000000,0.000000,0.000000}%
\pgfsetstrokecolor{currentstroke}%
\pgfsetdash{}{0pt}%
\pgfsys@defobject{currentmarker}{\pgfqpoint{0.000000in}{-0.048611in}}{\pgfqpoint{0.000000in}{0.000000in}}{%
\pgfpathmoveto{\pgfqpoint{0.000000in}{0.000000in}}%
\pgfpathlineto{\pgfqpoint{0.000000in}{-0.048611in}}%
\pgfusepath{stroke,fill}%
}%
\begin{pgfscope}%
\pgfsys@transformshift{1.035381in}{0.437500in}%
\pgfsys@useobject{currentmarker}{}%
\end{pgfscope}%
\end{pgfscope}%
\begin{pgfscope}%
\pgftext[x=1.035381in,y=0.340278in,,top]{\sffamily\fontsize{10.000000}{12.000000}\selectfont 67}%
\end{pgfscope}%
\begin{pgfscope}%
\pgfpathrectangle{\pgfqpoint{0.625000in}{0.437500in}}{\pgfqpoint{3.875000in}{2.642500in}}%
\pgfusepath{clip}%
\pgfsetrectcap%
\pgfsetroundjoin%
\pgfsetlinewidth{0.803000pt}%
\definecolor{currentstroke}{rgb}{0.690196,0.690196,0.690196}%
\pgfsetstrokecolor{currentstroke}%
\pgfsetdash{}{0pt}%
\pgfpathmoveto{\pgfqpoint{1.497243in}{0.437500in}}%
\pgfpathlineto{\pgfqpoint{1.497243in}{3.080000in}}%
\pgfusepath{stroke}%
\end{pgfscope}%
\begin{pgfscope}%
\pgfsetbuttcap%
\pgfsetroundjoin%
\definecolor{currentfill}{rgb}{0.000000,0.000000,0.000000}%
\pgfsetfillcolor{currentfill}%
\pgfsetlinewidth{0.803000pt}%
\definecolor{currentstroke}{rgb}{0.000000,0.000000,0.000000}%
\pgfsetstrokecolor{currentstroke}%
\pgfsetdash{}{0pt}%
\pgfsys@defobject{currentmarker}{\pgfqpoint{0.000000in}{-0.048611in}}{\pgfqpoint{0.000000in}{0.000000in}}{%
\pgfpathmoveto{\pgfqpoint{0.000000in}{0.000000in}}%
\pgfpathlineto{\pgfqpoint{0.000000in}{-0.048611in}}%
\pgfusepath{stroke,fill}%
}%
\begin{pgfscope}%
\pgfsys@transformshift{1.497243in}{0.437500in}%
\pgfsys@useobject{currentmarker}{}%
\end{pgfscope}%
\end{pgfscope}%
\begin{pgfscope}%
\pgftext[x=1.497243in,y=0.340278in,,top]{\sffamily\fontsize{10.000000}{12.000000}\selectfont 68}%
\end{pgfscope}%
\begin{pgfscope}%
\pgfpathrectangle{\pgfqpoint{0.625000in}{0.437500in}}{\pgfqpoint{3.875000in}{2.642500in}}%
\pgfusepath{clip}%
\pgfsetrectcap%
\pgfsetroundjoin%
\pgfsetlinewidth{0.803000pt}%
\definecolor{currentstroke}{rgb}{0.690196,0.690196,0.690196}%
\pgfsetstrokecolor{currentstroke}%
\pgfsetdash{}{0pt}%
\pgfpathmoveto{\pgfqpoint{1.959105in}{0.437500in}}%
\pgfpathlineto{\pgfqpoint{1.959105in}{3.080000in}}%
\pgfusepath{stroke}%
\end{pgfscope}%
\begin{pgfscope}%
\pgfsetbuttcap%
\pgfsetroundjoin%
\definecolor{currentfill}{rgb}{0.000000,0.000000,0.000000}%
\pgfsetfillcolor{currentfill}%
\pgfsetlinewidth{0.803000pt}%
\definecolor{currentstroke}{rgb}{0.000000,0.000000,0.000000}%
\pgfsetstrokecolor{currentstroke}%
\pgfsetdash{}{0pt}%
\pgfsys@defobject{currentmarker}{\pgfqpoint{0.000000in}{-0.048611in}}{\pgfqpoint{0.000000in}{0.000000in}}{%
\pgfpathmoveto{\pgfqpoint{0.000000in}{0.000000in}}%
\pgfpathlineto{\pgfqpoint{0.000000in}{-0.048611in}}%
\pgfusepath{stroke,fill}%
}%
\begin{pgfscope}%
\pgfsys@transformshift{1.959105in}{0.437500in}%
\pgfsys@useobject{currentmarker}{}%
\end{pgfscope}%
\end{pgfscope}%
\begin{pgfscope}%
\pgftext[x=1.959105in,y=0.340278in,,top]{\sffamily\fontsize{10.000000}{12.000000}\selectfont 69}%
\end{pgfscope}%
\begin{pgfscope}%
\pgfpathrectangle{\pgfqpoint{0.625000in}{0.437500in}}{\pgfqpoint{3.875000in}{2.642500in}}%
\pgfusepath{clip}%
\pgfsetrectcap%
\pgfsetroundjoin%
\pgfsetlinewidth{0.803000pt}%
\definecolor{currentstroke}{rgb}{0.690196,0.690196,0.690196}%
\pgfsetstrokecolor{currentstroke}%
\pgfsetdash{}{0pt}%
\pgfpathmoveto{\pgfqpoint{2.420967in}{0.437500in}}%
\pgfpathlineto{\pgfqpoint{2.420967in}{3.080000in}}%
\pgfusepath{stroke}%
\end{pgfscope}%
\begin{pgfscope}%
\pgfsetbuttcap%
\pgfsetroundjoin%
\definecolor{currentfill}{rgb}{0.000000,0.000000,0.000000}%
\pgfsetfillcolor{currentfill}%
\pgfsetlinewidth{0.803000pt}%
\definecolor{currentstroke}{rgb}{0.000000,0.000000,0.000000}%
\pgfsetstrokecolor{currentstroke}%
\pgfsetdash{}{0pt}%
\pgfsys@defobject{currentmarker}{\pgfqpoint{0.000000in}{-0.048611in}}{\pgfqpoint{0.000000in}{0.000000in}}{%
\pgfpathmoveto{\pgfqpoint{0.000000in}{0.000000in}}%
\pgfpathlineto{\pgfqpoint{0.000000in}{-0.048611in}}%
\pgfusepath{stroke,fill}%
}%
\begin{pgfscope}%
\pgfsys@transformshift{2.420967in}{0.437500in}%
\pgfsys@useobject{currentmarker}{}%
\end{pgfscope}%
\end{pgfscope}%
\begin{pgfscope}%
\pgftext[x=2.420967in,y=0.340278in,,top]{\sffamily\fontsize{10.000000}{12.000000}\selectfont 70}%
\end{pgfscope}%
\begin{pgfscope}%
\pgfpathrectangle{\pgfqpoint{0.625000in}{0.437500in}}{\pgfqpoint{3.875000in}{2.642500in}}%
\pgfusepath{clip}%
\pgfsetrectcap%
\pgfsetroundjoin%
\pgfsetlinewidth{0.803000pt}%
\definecolor{currentstroke}{rgb}{0.690196,0.690196,0.690196}%
\pgfsetstrokecolor{currentstroke}%
\pgfsetdash{}{0pt}%
\pgfpathmoveto{\pgfqpoint{2.882828in}{0.437500in}}%
\pgfpathlineto{\pgfqpoint{2.882828in}{3.080000in}}%
\pgfusepath{stroke}%
\end{pgfscope}%
\begin{pgfscope}%
\pgfsetbuttcap%
\pgfsetroundjoin%
\definecolor{currentfill}{rgb}{0.000000,0.000000,0.000000}%
\pgfsetfillcolor{currentfill}%
\pgfsetlinewidth{0.803000pt}%
\definecolor{currentstroke}{rgb}{0.000000,0.000000,0.000000}%
\pgfsetstrokecolor{currentstroke}%
\pgfsetdash{}{0pt}%
\pgfsys@defobject{currentmarker}{\pgfqpoint{0.000000in}{-0.048611in}}{\pgfqpoint{0.000000in}{0.000000in}}{%
\pgfpathmoveto{\pgfqpoint{0.000000in}{0.000000in}}%
\pgfpathlineto{\pgfqpoint{0.000000in}{-0.048611in}}%
\pgfusepath{stroke,fill}%
}%
\begin{pgfscope}%
\pgfsys@transformshift{2.882828in}{0.437500in}%
\pgfsys@useobject{currentmarker}{}%
\end{pgfscope}%
\end{pgfscope}%
\begin{pgfscope}%
\pgftext[x=2.882828in,y=0.340278in,,top]{\sffamily\fontsize{10.000000}{12.000000}\selectfont 71}%
\end{pgfscope}%
\begin{pgfscope}%
\pgfpathrectangle{\pgfqpoint{0.625000in}{0.437500in}}{\pgfqpoint{3.875000in}{2.642500in}}%
\pgfusepath{clip}%
\pgfsetrectcap%
\pgfsetroundjoin%
\pgfsetlinewidth{0.803000pt}%
\definecolor{currentstroke}{rgb}{0.690196,0.690196,0.690196}%
\pgfsetstrokecolor{currentstroke}%
\pgfsetdash{}{0pt}%
\pgfpathmoveto{\pgfqpoint{3.344690in}{0.437500in}}%
\pgfpathlineto{\pgfqpoint{3.344690in}{3.080000in}}%
\pgfusepath{stroke}%
\end{pgfscope}%
\begin{pgfscope}%
\pgfsetbuttcap%
\pgfsetroundjoin%
\definecolor{currentfill}{rgb}{0.000000,0.000000,0.000000}%
\pgfsetfillcolor{currentfill}%
\pgfsetlinewidth{0.803000pt}%
\definecolor{currentstroke}{rgb}{0.000000,0.000000,0.000000}%
\pgfsetstrokecolor{currentstroke}%
\pgfsetdash{}{0pt}%
\pgfsys@defobject{currentmarker}{\pgfqpoint{0.000000in}{-0.048611in}}{\pgfqpoint{0.000000in}{0.000000in}}{%
\pgfpathmoveto{\pgfqpoint{0.000000in}{0.000000in}}%
\pgfpathlineto{\pgfqpoint{0.000000in}{-0.048611in}}%
\pgfusepath{stroke,fill}%
}%
\begin{pgfscope}%
\pgfsys@transformshift{3.344690in}{0.437500in}%
\pgfsys@useobject{currentmarker}{}%
\end{pgfscope}%
\end{pgfscope}%
\begin{pgfscope}%
\pgftext[x=3.344690in,y=0.340278in,,top]{\sffamily\fontsize{10.000000}{12.000000}\selectfont 72}%
\end{pgfscope}%
\begin{pgfscope}%
\pgfpathrectangle{\pgfqpoint{0.625000in}{0.437500in}}{\pgfqpoint{3.875000in}{2.642500in}}%
\pgfusepath{clip}%
\pgfsetrectcap%
\pgfsetroundjoin%
\pgfsetlinewidth{0.803000pt}%
\definecolor{currentstroke}{rgb}{0.690196,0.690196,0.690196}%
\pgfsetstrokecolor{currentstroke}%
\pgfsetdash{}{0pt}%
\pgfpathmoveto{\pgfqpoint{3.806552in}{0.437500in}}%
\pgfpathlineto{\pgfqpoint{3.806552in}{3.080000in}}%
\pgfusepath{stroke}%
\end{pgfscope}%
\begin{pgfscope}%
\pgfsetbuttcap%
\pgfsetroundjoin%
\definecolor{currentfill}{rgb}{0.000000,0.000000,0.000000}%
\pgfsetfillcolor{currentfill}%
\pgfsetlinewidth{0.803000pt}%
\definecolor{currentstroke}{rgb}{0.000000,0.000000,0.000000}%
\pgfsetstrokecolor{currentstroke}%
\pgfsetdash{}{0pt}%
\pgfsys@defobject{currentmarker}{\pgfqpoint{0.000000in}{-0.048611in}}{\pgfqpoint{0.000000in}{0.000000in}}{%
\pgfpathmoveto{\pgfqpoint{0.000000in}{0.000000in}}%
\pgfpathlineto{\pgfqpoint{0.000000in}{-0.048611in}}%
\pgfusepath{stroke,fill}%
}%
\begin{pgfscope}%
\pgfsys@transformshift{3.806552in}{0.437500in}%
\pgfsys@useobject{currentmarker}{}%
\end{pgfscope}%
\end{pgfscope}%
\begin{pgfscope}%
\pgftext[x=3.806552in,y=0.340278in,,top]{\sffamily\fontsize{10.000000}{12.000000}\selectfont 73}%
\end{pgfscope}%
\begin{pgfscope}%
\pgfpathrectangle{\pgfqpoint{0.625000in}{0.437500in}}{\pgfqpoint{3.875000in}{2.642500in}}%
\pgfusepath{clip}%
\pgfsetrectcap%
\pgfsetroundjoin%
\pgfsetlinewidth{0.803000pt}%
\definecolor{currentstroke}{rgb}{0.690196,0.690196,0.690196}%
\pgfsetstrokecolor{currentstroke}%
\pgfsetdash{}{0pt}%
\pgfpathmoveto{\pgfqpoint{4.268414in}{0.437500in}}%
\pgfpathlineto{\pgfqpoint{4.268414in}{3.080000in}}%
\pgfusepath{stroke}%
\end{pgfscope}%
\begin{pgfscope}%
\pgfsetbuttcap%
\pgfsetroundjoin%
\definecolor{currentfill}{rgb}{0.000000,0.000000,0.000000}%
\pgfsetfillcolor{currentfill}%
\pgfsetlinewidth{0.803000pt}%
\definecolor{currentstroke}{rgb}{0.000000,0.000000,0.000000}%
\pgfsetstrokecolor{currentstroke}%
\pgfsetdash{}{0pt}%
\pgfsys@defobject{currentmarker}{\pgfqpoint{0.000000in}{-0.048611in}}{\pgfqpoint{0.000000in}{0.000000in}}{%
\pgfpathmoveto{\pgfqpoint{0.000000in}{0.000000in}}%
\pgfpathlineto{\pgfqpoint{0.000000in}{-0.048611in}}%
\pgfusepath{stroke,fill}%
}%
\begin{pgfscope}%
\pgfsys@transformshift{4.268414in}{0.437500in}%
\pgfsys@useobject{currentmarker}{}%
\end{pgfscope}%
\end{pgfscope}%
\begin{pgfscope}%
\pgftext[x=4.268414in,y=0.340278in,,top]{\sffamily\fontsize{10.000000}{12.000000}\selectfont 74}%
\end{pgfscope}%
\begin{pgfscope}%
\pgftext[x=2.562500in,y=0.161389in,,top]{\sffamily\fontsize{10.000000}{12.000000}\selectfont \( \delta \, (\si{\degree}) \)}%
\end{pgfscope}%
\begin{pgfscope}%
\pgfpathrectangle{\pgfqpoint{0.625000in}{0.437500in}}{\pgfqpoint{3.875000in}{2.642500in}}%
\pgfusepath{clip}%
\pgfsetrectcap%
\pgfsetroundjoin%
\pgfsetlinewidth{0.803000pt}%
\definecolor{currentstroke}{rgb}{0.690196,0.690196,0.690196}%
\pgfsetstrokecolor{currentstroke}%
\pgfsetdash{}{0pt}%
\pgfpathmoveto{\pgfqpoint{0.625000in}{0.493647in}}%
\pgfpathlineto{\pgfqpoint{4.500000in}{0.493647in}}%
\pgfusepath{stroke}%
\end{pgfscope}%
\begin{pgfscope}%
\pgfsetbuttcap%
\pgfsetroundjoin%
\definecolor{currentfill}{rgb}{0.000000,0.000000,0.000000}%
\pgfsetfillcolor{currentfill}%
\pgfsetlinewidth{0.803000pt}%
\definecolor{currentstroke}{rgb}{0.000000,0.000000,0.000000}%
\pgfsetstrokecolor{currentstroke}%
\pgfsetdash{}{0pt}%
\pgfsys@defobject{currentmarker}{\pgfqpoint{-0.048611in}{0.000000in}}{\pgfqpoint{0.000000in}{0.000000in}}{%
\pgfpathmoveto{\pgfqpoint{0.000000in}{0.000000in}}%
\pgfpathlineto{\pgfqpoint{-0.048611in}{0.000000in}}%
\pgfusepath{stroke,fill}%
}%
\begin{pgfscope}%
\pgfsys@transformshift{0.625000in}{0.493647in}%
\pgfsys@useobject{currentmarker}{}%
\end{pgfscope}%
\end{pgfscope}%
\begin{pgfscope}%
\pgftext[x=0.250000in,y=0.445453in,left,base]{\sffamily\fontsize{10.000000}{12.000000}\selectfont 4000}%
\end{pgfscope}%
\begin{pgfscope}%
\pgfpathrectangle{\pgfqpoint{0.625000in}{0.437500in}}{\pgfqpoint{3.875000in}{2.642500in}}%
\pgfusepath{clip}%
\pgfsetrectcap%
\pgfsetroundjoin%
\pgfsetlinewidth{0.803000pt}%
\definecolor{currentstroke}{rgb}{0.690196,0.690196,0.690196}%
\pgfsetstrokecolor{currentstroke}%
\pgfsetdash{}{0pt}%
\pgfpathmoveto{\pgfqpoint{0.625000in}{0.836816in}}%
\pgfpathlineto{\pgfqpoint{4.500000in}{0.836816in}}%
\pgfusepath{stroke}%
\end{pgfscope}%
\begin{pgfscope}%
\pgfsetbuttcap%
\pgfsetroundjoin%
\definecolor{currentfill}{rgb}{0.000000,0.000000,0.000000}%
\pgfsetfillcolor{currentfill}%
\pgfsetlinewidth{0.803000pt}%
\definecolor{currentstroke}{rgb}{0.000000,0.000000,0.000000}%
\pgfsetstrokecolor{currentstroke}%
\pgfsetdash{}{0pt}%
\pgfsys@defobject{currentmarker}{\pgfqpoint{-0.048611in}{0.000000in}}{\pgfqpoint{0.000000in}{0.000000in}}{%
\pgfpathmoveto{\pgfqpoint{0.000000in}{0.000000in}}%
\pgfpathlineto{\pgfqpoint{-0.048611in}{0.000000in}}%
\pgfusepath{stroke,fill}%
}%
\begin{pgfscope}%
\pgfsys@transformshift{0.625000in}{0.836816in}%
\pgfsys@useobject{currentmarker}{}%
\end{pgfscope}%
\end{pgfscope}%
\begin{pgfscope}%
\pgftext[x=0.250000in,y=0.788621in,left,base]{\sffamily\fontsize{10.000000}{12.000000}\selectfont 4250}%
\end{pgfscope}%
\begin{pgfscope}%
\pgfpathrectangle{\pgfqpoint{0.625000in}{0.437500in}}{\pgfqpoint{3.875000in}{2.642500in}}%
\pgfusepath{clip}%
\pgfsetrectcap%
\pgfsetroundjoin%
\pgfsetlinewidth{0.803000pt}%
\definecolor{currentstroke}{rgb}{0.690196,0.690196,0.690196}%
\pgfsetstrokecolor{currentstroke}%
\pgfsetdash{}{0pt}%
\pgfpathmoveto{\pgfqpoint{0.625000in}{1.179984in}}%
\pgfpathlineto{\pgfqpoint{4.500000in}{1.179984in}}%
\pgfusepath{stroke}%
\end{pgfscope}%
\begin{pgfscope}%
\pgfsetbuttcap%
\pgfsetroundjoin%
\definecolor{currentfill}{rgb}{0.000000,0.000000,0.000000}%
\pgfsetfillcolor{currentfill}%
\pgfsetlinewidth{0.803000pt}%
\definecolor{currentstroke}{rgb}{0.000000,0.000000,0.000000}%
\pgfsetstrokecolor{currentstroke}%
\pgfsetdash{}{0pt}%
\pgfsys@defobject{currentmarker}{\pgfqpoint{-0.048611in}{0.000000in}}{\pgfqpoint{0.000000in}{0.000000in}}{%
\pgfpathmoveto{\pgfqpoint{0.000000in}{0.000000in}}%
\pgfpathlineto{\pgfqpoint{-0.048611in}{0.000000in}}%
\pgfusepath{stroke,fill}%
}%
\begin{pgfscope}%
\pgfsys@transformshift{0.625000in}{1.179984in}%
\pgfsys@useobject{currentmarker}{}%
\end{pgfscope}%
\end{pgfscope}%
\begin{pgfscope}%
\pgftext[x=0.250000in,y=1.131790in,left,base]{\sffamily\fontsize{10.000000}{12.000000}\selectfont 4500}%
\end{pgfscope}%
\begin{pgfscope}%
\pgfpathrectangle{\pgfqpoint{0.625000in}{0.437500in}}{\pgfqpoint{3.875000in}{2.642500in}}%
\pgfusepath{clip}%
\pgfsetrectcap%
\pgfsetroundjoin%
\pgfsetlinewidth{0.803000pt}%
\definecolor{currentstroke}{rgb}{0.690196,0.690196,0.690196}%
\pgfsetstrokecolor{currentstroke}%
\pgfsetdash{}{0pt}%
\pgfpathmoveto{\pgfqpoint{0.625000in}{1.523153in}}%
\pgfpathlineto{\pgfqpoint{4.500000in}{1.523153in}}%
\pgfusepath{stroke}%
\end{pgfscope}%
\begin{pgfscope}%
\pgfsetbuttcap%
\pgfsetroundjoin%
\definecolor{currentfill}{rgb}{0.000000,0.000000,0.000000}%
\pgfsetfillcolor{currentfill}%
\pgfsetlinewidth{0.803000pt}%
\definecolor{currentstroke}{rgb}{0.000000,0.000000,0.000000}%
\pgfsetstrokecolor{currentstroke}%
\pgfsetdash{}{0pt}%
\pgfsys@defobject{currentmarker}{\pgfqpoint{-0.048611in}{0.000000in}}{\pgfqpoint{0.000000in}{0.000000in}}{%
\pgfpathmoveto{\pgfqpoint{0.000000in}{0.000000in}}%
\pgfpathlineto{\pgfqpoint{-0.048611in}{0.000000in}}%
\pgfusepath{stroke,fill}%
}%
\begin{pgfscope}%
\pgfsys@transformshift{0.625000in}{1.523153in}%
\pgfsys@useobject{currentmarker}{}%
\end{pgfscope}%
\end{pgfscope}%
\begin{pgfscope}%
\pgftext[x=0.250000in,y=1.474959in,left,base]{\sffamily\fontsize{10.000000}{12.000000}\selectfont 4750}%
\end{pgfscope}%
\begin{pgfscope}%
\pgfpathrectangle{\pgfqpoint{0.625000in}{0.437500in}}{\pgfqpoint{3.875000in}{2.642500in}}%
\pgfusepath{clip}%
\pgfsetrectcap%
\pgfsetroundjoin%
\pgfsetlinewidth{0.803000pt}%
\definecolor{currentstroke}{rgb}{0.690196,0.690196,0.690196}%
\pgfsetstrokecolor{currentstroke}%
\pgfsetdash{}{0pt}%
\pgfpathmoveto{\pgfqpoint{0.625000in}{1.866322in}}%
\pgfpathlineto{\pgfqpoint{4.500000in}{1.866322in}}%
\pgfusepath{stroke}%
\end{pgfscope}%
\begin{pgfscope}%
\pgfsetbuttcap%
\pgfsetroundjoin%
\definecolor{currentfill}{rgb}{0.000000,0.000000,0.000000}%
\pgfsetfillcolor{currentfill}%
\pgfsetlinewidth{0.803000pt}%
\definecolor{currentstroke}{rgb}{0.000000,0.000000,0.000000}%
\pgfsetstrokecolor{currentstroke}%
\pgfsetdash{}{0pt}%
\pgfsys@defobject{currentmarker}{\pgfqpoint{-0.048611in}{0.000000in}}{\pgfqpoint{0.000000in}{0.000000in}}{%
\pgfpathmoveto{\pgfqpoint{0.000000in}{0.000000in}}%
\pgfpathlineto{\pgfqpoint{-0.048611in}{0.000000in}}%
\pgfusepath{stroke,fill}%
}%
\begin{pgfscope}%
\pgfsys@transformshift{0.625000in}{1.866322in}%
\pgfsys@useobject{currentmarker}{}%
\end{pgfscope}%
\end{pgfscope}%
\begin{pgfscope}%
\pgftext[x=0.250000in,y=1.818127in,left,base]{\sffamily\fontsize{10.000000}{12.000000}\selectfont 5000}%
\end{pgfscope}%
\begin{pgfscope}%
\pgfpathrectangle{\pgfqpoint{0.625000in}{0.437500in}}{\pgfqpoint{3.875000in}{2.642500in}}%
\pgfusepath{clip}%
\pgfsetrectcap%
\pgfsetroundjoin%
\pgfsetlinewidth{0.803000pt}%
\definecolor{currentstroke}{rgb}{0.690196,0.690196,0.690196}%
\pgfsetstrokecolor{currentstroke}%
\pgfsetdash{}{0pt}%
\pgfpathmoveto{\pgfqpoint{0.625000in}{2.209490in}}%
\pgfpathlineto{\pgfqpoint{4.500000in}{2.209490in}}%
\pgfusepath{stroke}%
\end{pgfscope}%
\begin{pgfscope}%
\pgfsetbuttcap%
\pgfsetroundjoin%
\definecolor{currentfill}{rgb}{0.000000,0.000000,0.000000}%
\pgfsetfillcolor{currentfill}%
\pgfsetlinewidth{0.803000pt}%
\definecolor{currentstroke}{rgb}{0.000000,0.000000,0.000000}%
\pgfsetstrokecolor{currentstroke}%
\pgfsetdash{}{0pt}%
\pgfsys@defobject{currentmarker}{\pgfqpoint{-0.048611in}{0.000000in}}{\pgfqpoint{0.000000in}{0.000000in}}{%
\pgfpathmoveto{\pgfqpoint{0.000000in}{0.000000in}}%
\pgfpathlineto{\pgfqpoint{-0.048611in}{0.000000in}}%
\pgfusepath{stroke,fill}%
}%
\begin{pgfscope}%
\pgfsys@transformshift{0.625000in}{2.209490in}%
\pgfsys@useobject{currentmarker}{}%
\end{pgfscope}%
\end{pgfscope}%
\begin{pgfscope}%
\pgftext[x=0.250000in,y=2.161296in,left,base]{\sffamily\fontsize{10.000000}{12.000000}\selectfont 5250}%
\end{pgfscope}%
\begin{pgfscope}%
\pgfpathrectangle{\pgfqpoint{0.625000in}{0.437500in}}{\pgfqpoint{3.875000in}{2.642500in}}%
\pgfusepath{clip}%
\pgfsetrectcap%
\pgfsetroundjoin%
\pgfsetlinewidth{0.803000pt}%
\definecolor{currentstroke}{rgb}{0.690196,0.690196,0.690196}%
\pgfsetstrokecolor{currentstroke}%
\pgfsetdash{}{0pt}%
\pgfpathmoveto{\pgfqpoint{0.625000in}{2.552659in}}%
\pgfpathlineto{\pgfqpoint{4.500000in}{2.552659in}}%
\pgfusepath{stroke}%
\end{pgfscope}%
\begin{pgfscope}%
\pgfsetbuttcap%
\pgfsetroundjoin%
\definecolor{currentfill}{rgb}{0.000000,0.000000,0.000000}%
\pgfsetfillcolor{currentfill}%
\pgfsetlinewidth{0.803000pt}%
\definecolor{currentstroke}{rgb}{0.000000,0.000000,0.000000}%
\pgfsetstrokecolor{currentstroke}%
\pgfsetdash{}{0pt}%
\pgfsys@defobject{currentmarker}{\pgfqpoint{-0.048611in}{0.000000in}}{\pgfqpoint{0.000000in}{0.000000in}}{%
\pgfpathmoveto{\pgfqpoint{0.000000in}{0.000000in}}%
\pgfpathlineto{\pgfqpoint{-0.048611in}{0.000000in}}%
\pgfusepath{stroke,fill}%
}%
\begin{pgfscope}%
\pgfsys@transformshift{0.625000in}{2.552659in}%
\pgfsys@useobject{currentmarker}{}%
\end{pgfscope}%
\end{pgfscope}%
\begin{pgfscope}%
\pgftext[x=0.250000in,y=2.504465in,left,base]{\sffamily\fontsize{10.000000}{12.000000}\selectfont 5500}%
\end{pgfscope}%
\begin{pgfscope}%
\pgfpathrectangle{\pgfqpoint{0.625000in}{0.437500in}}{\pgfqpoint{3.875000in}{2.642500in}}%
\pgfusepath{clip}%
\pgfsetrectcap%
\pgfsetroundjoin%
\pgfsetlinewidth{0.803000pt}%
\definecolor{currentstroke}{rgb}{0.690196,0.690196,0.690196}%
\pgfsetstrokecolor{currentstroke}%
\pgfsetdash{}{0pt}%
\pgfpathmoveto{\pgfqpoint{0.625000in}{2.895828in}}%
\pgfpathlineto{\pgfqpoint{4.500000in}{2.895828in}}%
\pgfusepath{stroke}%
\end{pgfscope}%
\begin{pgfscope}%
\pgfsetbuttcap%
\pgfsetroundjoin%
\definecolor{currentfill}{rgb}{0.000000,0.000000,0.000000}%
\pgfsetfillcolor{currentfill}%
\pgfsetlinewidth{0.803000pt}%
\definecolor{currentstroke}{rgb}{0.000000,0.000000,0.000000}%
\pgfsetstrokecolor{currentstroke}%
\pgfsetdash{}{0pt}%
\pgfsys@defobject{currentmarker}{\pgfqpoint{-0.048611in}{0.000000in}}{\pgfqpoint{0.000000in}{0.000000in}}{%
\pgfpathmoveto{\pgfqpoint{0.000000in}{0.000000in}}%
\pgfpathlineto{\pgfqpoint{-0.048611in}{0.000000in}}%
\pgfusepath{stroke,fill}%
}%
\begin{pgfscope}%
\pgfsys@transformshift{0.625000in}{2.895828in}%
\pgfsys@useobject{currentmarker}{}%
\end{pgfscope}%
\end{pgfscope}%
\begin{pgfscope}%
\pgftext[x=0.250000in,y=2.847633in,left,base]{\sffamily\fontsize{10.000000}{12.000000}\selectfont 5750}%
\end{pgfscope}%
\begin{pgfscope}%
\pgftext[x=0.194444in,y=1.758750in,,bottom,rotate=90.000000]{\sffamily\fontsize{10.000000}{12.000000}\selectfont \( \lambda \, (\si{\angstrom}) \)}%
\end{pgfscope}%
\begin{pgfscope}%
\pgfpathrectangle{\pgfqpoint{0.625000in}{0.437500in}}{\pgfqpoint{3.875000in}{2.642500in}}%
\pgfusepath{clip}%
\pgfsetbuttcap%
\pgfsetroundjoin%
\pgfsetlinewidth{1.505625pt}%
\definecolor{currentstroke}{rgb}{0.678431,0.847059,0.901961}%
\pgfsetstrokecolor{currentstroke}%
\pgfsetdash{}{0pt}%
\pgfpathmoveto{\pgfqpoint{4.311740in}{0.557614in}}%
\pgfpathlineto{\pgfqpoint{4.323864in}{0.557614in}}%
\pgfusepath{stroke}%
\end{pgfscope}%
\begin{pgfscope}%
\pgfpathrectangle{\pgfqpoint{0.625000in}{0.437500in}}{\pgfqpoint{3.875000in}{2.642500in}}%
\pgfusepath{clip}%
\pgfsetbuttcap%
\pgfsetroundjoin%
\pgfsetlinewidth{1.505625pt}%
\definecolor{currentstroke}{rgb}{0.678431,0.847059,0.901961}%
\pgfsetstrokecolor{currentstroke}%
\pgfsetdash{}{0pt}%
\pgfpathmoveto{\pgfqpoint{4.188758in}{0.600990in}}%
\pgfpathlineto{\pgfqpoint{4.219767in}{0.600990in}}%
\pgfusepath{stroke}%
\end{pgfscope}%
\begin{pgfscope}%
\pgfpathrectangle{\pgfqpoint{0.625000in}{0.437500in}}{\pgfqpoint{3.875000in}{2.642500in}}%
\pgfusepath{clip}%
\pgfsetbuttcap%
\pgfsetroundjoin%
\pgfsetlinewidth{1.505625pt}%
\definecolor{currentstroke}{rgb}{0.678431,0.847059,0.901961}%
\pgfsetstrokecolor{currentstroke}%
\pgfsetdash{}{0pt}%
\pgfpathmoveto{\pgfqpoint{3.224806in}{0.985476in}}%
\pgfpathlineto{\pgfqpoint{3.228501in}{0.985476in}}%
\pgfusepath{stroke}%
\end{pgfscope}%
\begin{pgfscope}%
\pgfpathrectangle{\pgfqpoint{0.625000in}{0.437500in}}{\pgfqpoint{3.875000in}{2.642500in}}%
\pgfusepath{clip}%
\pgfsetbuttcap%
\pgfsetroundjoin%
\pgfsetlinewidth{1.505625pt}%
\definecolor{currentstroke}{rgb}{0.678431,0.847059,0.901961}%
\pgfsetstrokecolor{currentstroke}%
\pgfsetdash{}{0pt}%
\pgfpathmoveto{\pgfqpoint{1.933315in}{1.751017in}}%
\pgfpathlineto{\pgfqpoint{1.960507in}{1.751017in}}%
\pgfusepath{stroke}%
\end{pgfscope}%
\begin{pgfscope}%
\pgfpathrectangle{\pgfqpoint{0.625000in}{0.437500in}}{\pgfqpoint{3.875000in}{2.642500in}}%
\pgfusepath{clip}%
\pgfsetbuttcap%
\pgfsetroundjoin%
\pgfsetlinewidth{1.505625pt}%
\definecolor{currentstroke}{rgb}{0.678431,0.847059,0.901961}%
\pgfsetstrokecolor{currentstroke}%
\pgfsetdash{}{0pt}%
\pgfpathmoveto{\pgfqpoint{1.155755in}{2.498713in}}%
\pgfpathlineto{\pgfqpoint{1.171588in}{2.498713in}}%
\pgfusepath{stroke}%
\end{pgfscope}%
\begin{pgfscope}%
\pgfpathrectangle{\pgfqpoint{0.625000in}{0.437500in}}{\pgfqpoint{3.875000in}{2.642500in}}%
\pgfusepath{clip}%
\pgfsetbuttcap%
\pgfsetroundjoin%
\pgfsetlinewidth{1.505625pt}%
\definecolor{currentstroke}{rgb}{0.678431,0.847059,0.901961}%
\pgfsetstrokecolor{currentstroke}%
\pgfsetdash{}{0pt}%
\pgfpathmoveto{\pgfqpoint{0.824826in}{2.922732in}}%
\pgfpathlineto{\pgfqpoint{0.848213in}{2.922732in}}%
\pgfusepath{stroke}%
\end{pgfscope}%
\begin{pgfscope}%
\pgfpathrectangle{\pgfqpoint{0.625000in}{0.437500in}}{\pgfqpoint{3.875000in}{2.642500in}}%
\pgfusepath{clip}%
\pgfsetbuttcap%
\pgfsetroundjoin%
\pgfsetlinewidth{1.505625pt}%
\definecolor{currentstroke}{rgb}{0.678431,0.847059,0.901961}%
\pgfsetstrokecolor{currentstroke}%
\pgfsetdash{}{0pt}%
\pgfpathmoveto{\pgfqpoint{0.801136in}{2.950323in}}%
\pgfpathlineto{\pgfqpoint{0.806472in}{2.950323in}}%
\pgfusepath{stroke}%
\end{pgfscope}%
\begin{pgfscope}%
\pgfpathrectangle{\pgfqpoint{0.625000in}{0.437500in}}{\pgfqpoint{3.875000in}{2.642500in}}%
\pgfusepath{clip}%
\pgfsetbuttcap%
\pgfsetroundjoin%
\pgfsetlinewidth{1.003750pt}%
\definecolor{currentstroke}{rgb}{0.854902,0.439216,0.839216}%
\pgfsetstrokecolor{currentstroke}%
\pgfsetdash{{6.400000pt}{1.600000pt}{1.000000pt}{1.600000pt}}{0.000000pt}%
\pgfpathmoveto{\pgfqpoint{0.803804in}{2.959886in}}%
\pgfpathlineto{\pgfqpoint{0.849990in}{2.894700in}}%
\pgfpathlineto{\pgfqpoint{0.896176in}{2.831340in}}%
\pgfpathlineto{\pgfqpoint{0.942363in}{2.769731in}}%
\pgfpathlineto{\pgfqpoint{0.988549in}{2.709801in}}%
\pgfpathlineto{\pgfqpoint{1.034735in}{2.651483in}}%
\pgfpathlineto{\pgfqpoint{1.080921in}{2.594713in}}%
\pgfpathlineto{\pgfqpoint{1.127107in}{2.539429in}}%
\pgfpathlineto{\pgfqpoint{1.173293in}{2.485575in}}%
\pgfpathlineto{\pgfqpoint{1.219480in}{2.433094in}}%
\pgfpathlineto{\pgfqpoint{1.265666in}{2.381936in}}%
\pgfpathlineto{\pgfqpoint{1.311852in}{2.332052in}}%
\pgfpathlineto{\pgfqpoint{1.358038in}{2.283393in}}%
\pgfpathlineto{\pgfqpoint{1.404224in}{2.235916in}}%
\pgfpathlineto{\pgfqpoint{1.450410in}{2.189578in}}%
\pgfpathlineto{\pgfqpoint{1.496597in}{2.144338in}}%
\pgfpathlineto{\pgfqpoint{1.542783in}{2.100158in}}%
\pgfpathlineto{\pgfqpoint{1.588969in}{2.057001in}}%
\pgfpathlineto{\pgfqpoint{1.635155in}{2.014832in}}%
\pgfpathlineto{\pgfqpoint{1.681341in}{1.973617in}}%
\pgfpathlineto{\pgfqpoint{1.727527in}{1.933325in}}%
\pgfpathlineto{\pgfqpoint{1.773714in}{1.893925in}}%
\pgfpathlineto{\pgfqpoint{1.819900in}{1.855386in}}%
\pgfpathlineto{\pgfqpoint{1.866086in}{1.817683in}}%
\pgfpathlineto{\pgfqpoint{1.912272in}{1.780787in}}%
\pgfpathlineto{\pgfqpoint{1.958458in}{1.744672in}}%
\pgfpathlineto{\pgfqpoint{2.004645in}{1.709316in}}%
\pgfpathlineto{\pgfqpoint{2.050831in}{1.674693in}}%
\pgfpathlineto{\pgfqpoint{2.097017in}{1.640781in}}%
\pgfpathlineto{\pgfqpoint{2.143203in}{1.607558in}}%
\pgfpathlineto{\pgfqpoint{2.189389in}{1.575005in}}%
\pgfpathlineto{\pgfqpoint{2.235575in}{1.543100in}}%
\pgfpathlineto{\pgfqpoint{2.281762in}{1.511824in}}%
\pgfpathlineto{\pgfqpoint{2.327948in}{1.481160in}}%
\pgfpathlineto{\pgfqpoint{2.374134in}{1.451088in}}%
\pgfpathlineto{\pgfqpoint{2.420320in}{1.421593in}}%
\pgfpathlineto{\pgfqpoint{2.466506in}{1.392658in}}%
\pgfpathlineto{\pgfqpoint{2.512692in}{1.364267in}}%
\pgfpathlineto{\pgfqpoint{2.558879in}{1.336405in}}%
\pgfpathlineto{\pgfqpoint{2.605065in}{1.309057in}}%
\pgfpathlineto{\pgfqpoint{2.651251in}{1.282209in}}%
\pgfpathlineto{\pgfqpoint{2.697437in}{1.255847in}}%
\pgfpathlineto{\pgfqpoint{2.743623in}{1.229959in}}%
\pgfpathlineto{\pgfqpoint{2.789809in}{1.204532in}}%
\pgfpathlineto{\pgfqpoint{2.835996in}{1.179553in}}%
\pgfpathlineto{\pgfqpoint{2.882182in}{1.155011in}}%
\pgfpathlineto{\pgfqpoint{2.928368in}{1.130895in}}%
\pgfpathlineto{\pgfqpoint{2.974554in}{1.107193in}}%
\pgfpathlineto{\pgfqpoint{3.020740in}{1.083895in}}%
\pgfpathlineto{\pgfqpoint{3.066926in}{1.060990in}}%
\pgfpathlineto{\pgfqpoint{3.113113in}{1.038470in}}%
\pgfpathlineto{\pgfqpoint{3.159299in}{1.016323in}}%
\pgfpathlineto{\pgfqpoint{3.205485in}{0.994541in}}%
\pgfpathlineto{\pgfqpoint{3.251671in}{0.973116in}}%
\pgfpathlineto{\pgfqpoint{3.297857in}{0.952038in}}%
\pgfpathlineto{\pgfqpoint{3.344044in}{0.931298in}}%
\pgfpathlineto{\pgfqpoint{3.390230in}{0.910890in}}%
\pgfpathlineto{\pgfqpoint{3.436416in}{0.890805in}}%
\pgfpathlineto{\pgfqpoint{3.482602in}{0.871034in}}%
\pgfpathlineto{\pgfqpoint{3.528788in}{0.851572in}}%
\pgfpathlineto{\pgfqpoint{3.574974in}{0.832411in}}%
\pgfpathlineto{\pgfqpoint{3.621161in}{0.813544in}}%
\pgfpathlineto{\pgfqpoint{3.667347in}{0.794964in}}%
\pgfpathlineto{\pgfqpoint{3.713533in}{0.776665in}}%
\pgfpathlineto{\pgfqpoint{3.759719in}{0.758640in}}%
\pgfpathlineto{\pgfqpoint{3.805905in}{0.740883in}}%
\pgfpathlineto{\pgfqpoint{3.852091in}{0.723389in}}%
\pgfpathlineto{\pgfqpoint{3.898278in}{0.706152in}}%
\pgfpathlineto{\pgfqpoint{3.944464in}{0.689165in}}%
\pgfpathlineto{\pgfqpoint{3.990650in}{0.672424in}}%
\pgfpathlineto{\pgfqpoint{4.036836in}{0.655923in}}%
\pgfpathlineto{\pgfqpoint{4.083022in}{0.639657in}}%
\pgfpathlineto{\pgfqpoint{4.129208in}{0.623621in}}%
\pgfpathlineto{\pgfqpoint{4.175395in}{0.607811in}}%
\pgfpathlineto{\pgfqpoint{4.221581in}{0.592221in}}%
\pgfpathlineto{\pgfqpoint{4.267767in}{0.576847in}}%
\pgfpathlineto{\pgfqpoint{4.313953in}{0.561685in}}%
\pgfusepath{stroke}%
\end{pgfscope}%
\begin{pgfscope}%
\pgfpathrectangle{\pgfqpoint{0.625000in}{0.437500in}}{\pgfqpoint{3.875000in}{2.642500in}}%
\pgfusepath{clip}%
\pgfsetbuttcap%
\pgfsetroundjoin%
\definecolor{currentfill}{rgb}{0.678431,0.847059,0.901961}%
\pgfsetfillcolor{currentfill}%
\pgfsetlinewidth{1.003750pt}%
\definecolor{currentstroke}{rgb}{0.121569,0.466667,0.705882}%
\pgfsetstrokecolor{currentstroke}%
\pgfsetdash{}{0pt}%
\pgfsys@defobject{currentmarker}{\pgfqpoint{-0.048611in}{-0.048611in}}{\pgfqpoint{0.048611in}{0.048611in}}{%
\pgfpathmoveto{\pgfqpoint{-0.048611in}{0.000000in}}%
\pgfpathlineto{\pgfqpoint{0.048611in}{0.000000in}}%
\pgfpathmoveto{\pgfqpoint{0.000000in}{-0.048611in}}%
\pgfpathlineto{\pgfqpoint{0.000000in}{0.048611in}}%
\pgfusepath{stroke,fill}%
}%
\begin{pgfscope}%
\pgfsys@transformshift{4.317802in}{0.557614in}%
\pgfsys@useobject{currentmarker}{}%
\end{pgfscope}%
\begin{pgfscope}%
\pgfsys@transformshift{4.204263in}{0.600990in}%
\pgfsys@useobject{currentmarker}{}%
\end{pgfscope}%
\begin{pgfscope}%
\pgfsys@transformshift{3.226654in}{0.985476in}%
\pgfsys@useobject{currentmarker}{}%
\end{pgfscope}%
\begin{pgfscope}%
\pgfsys@transformshift{1.946911in}{1.751017in}%
\pgfsys@useobject{currentmarker}{}%
\end{pgfscope}%
\begin{pgfscope}%
\pgfsys@transformshift{1.163671in}{2.498713in}%
\pgfsys@useobject{currentmarker}{}%
\end{pgfscope}%
\begin{pgfscope}%
\pgfsys@transformshift{0.836519in}{2.922732in}%
\pgfsys@useobject{currentmarker}{}%
\end{pgfscope}%
\begin{pgfscope}%
\pgfsys@transformshift{0.803804in}{2.950323in}%
\pgfsys@useobject{currentmarker}{}%
\end{pgfscope}%
\end{pgfscope}%
\begin{pgfscope}%
\pgfsetrectcap%
\pgfsetmiterjoin%
\pgfsetlinewidth{0.803000pt}%
\definecolor{currentstroke}{rgb}{0.000000,0.000000,0.000000}%
\pgfsetstrokecolor{currentstroke}%
\pgfsetdash{}{0pt}%
\pgfpathmoveto{\pgfqpoint{0.625000in}{0.437500in}}%
\pgfpathlineto{\pgfqpoint{0.625000in}{3.080000in}}%
\pgfusepath{stroke}%
\end{pgfscope}%
\begin{pgfscope}%
\pgfsetrectcap%
\pgfsetmiterjoin%
\pgfsetlinewidth{0.803000pt}%
\definecolor{currentstroke}{rgb}{0.000000,0.000000,0.000000}%
\pgfsetstrokecolor{currentstroke}%
\pgfsetdash{}{0pt}%
\pgfpathmoveto{\pgfqpoint{4.500000in}{0.437500in}}%
\pgfpathlineto{\pgfqpoint{4.500000in}{3.080000in}}%
\pgfusepath{stroke}%
\end{pgfscope}%
\begin{pgfscope}%
\pgfsetrectcap%
\pgfsetmiterjoin%
\pgfsetlinewidth{0.803000pt}%
\definecolor{currentstroke}{rgb}{0.000000,0.000000,0.000000}%
\pgfsetstrokecolor{currentstroke}%
\pgfsetdash{}{0pt}%
\pgfpathmoveto{\pgfqpoint{0.625000in}{0.437500in}}%
\pgfpathlineto{\pgfqpoint{4.500000in}{0.437500in}}%
\pgfusepath{stroke}%
\end{pgfscope}%
\begin{pgfscope}%
\pgfsetrectcap%
\pgfsetmiterjoin%
\pgfsetlinewidth{0.803000pt}%
\definecolor{currentstroke}{rgb}{0.000000,0.000000,0.000000}%
\pgfsetstrokecolor{currentstroke}%
\pgfsetdash{}{0pt}%
\pgfpathmoveto{\pgfqpoint{0.625000in}{3.080000in}}%
\pgfpathlineto{\pgfqpoint{4.500000in}{3.080000in}}%
\pgfusepath{stroke}%
\end{pgfscope}%
\end{pgfpicture}%
\makeatother%
\endgroup%

	\caption{Medidas de desviación mínima con la curva de Hartmann \( \lambda(\delta) \) con los valores estimadoss. Los errores en el ángulo son del orden de \num{e-2} por lo que no se pueden apreciar}
	\label{fig:hartmann Hg}
\end{figure}

Hechas las medidas de los ángulos de desviación mínima para un espectro conocido, el del mercurio (véase el \cref{tab:datos Hg}) podemos calibrar el espectrómetro para determinar las longitudes de onda de un espectro desconocido. Esto es, determinaremos las constantes de la fórmula de Hartmann que relaciona la desviación mínima con la longitud de onda:
\begin{equation} \label{eqn:hartmann}
	\lambda(\delta) = \lambda_0 + \frac{C}{\delta - \delta_0}.
\end{equation}
Con la función \texttt{curve\_fit} del módulo \textsf{SciPy} podemos hacer este ajuste y obtenemos
\begin{equation} \label{eqn:parametros hartmann}
	\begin{aligned}
		\lambda_0 & = \data{245}{4e1}{\angstrom} \\
		\delta_0 & = \data{59.6}{0.2}{\degree} \\
		C & = \data{232}{8e2}{\angstrom}.
	\end{aligned}
\end{equation}
La fórmula de Hartmann con estos valores es 
\begin{equation} \label{eqn:hartmann con parametros}
	\lambda(\delta) = \num{2450} + \frac{\num{23200}}{\delta - \num{59.6}}.
\end{equation}
En la \cref{fig:hartmann Hg} se muestran los puntos correspondientes a las medidas así como el gráfico del ajuste hecho. El ajuste tiene un coeficiente de correlación de \( R^2 = \num{0.99997} \).

\section{Determinación del espectro del cadmio}
Una vez hemos calibrado el espectrómetro podemos utilizarlo para determinar el espectro del cadmio. En el \cref{tab:datos Cd} se muestran las medidas de desviación mínimas tomadas. Para cada franja se tomó la media de las observaciones.  

\begin{table}[hbt]
	\centering \small \sffamily
	\caption{Comparación de las longitudes de onda del espectro del cadmio teóricas y las obtenidas experimentalmente}
	\label{tab:longs cd}
	\begin{tabular}{@{}S[table-parse-only]SSSc@{}}
		\toprule
		{\( \delta_m \) (\si{\degree})} & {\( \lambda_\text{teó} \) (\si{nm})} & {\( \lambda_\text{exp} \) (\si{nm})} & {\( \abs{\lambda_\text{teó} - \lambda_\text{exp}} \) (\si{nm})} & Color \\
		\midrule
		71.40 \pm 0.02 & 441.3 & 441 \pm 3 & 0.038 & Morado \\
		69.953 \pm 0.006 & 467.82 & 469 \pm 1 & 0.76 & Azul \\
		69.43 \pm 0.01 & 479.99 & 480 \pm 3 & 0.36 & Azul \\
		68.361 \pm 0.006 & 508.58 & 509 \pm 2 & 0.35 & Verde \\
		68.14 \pm 0.01 & 515.51 & 515 \pm 4 & 0.056 & Verde\\
		65.42 \pm 0.03 & 643.85 & 641 \pm 10 & 2.6 & Rojo \\
		\bottomrule
	\end{tabular}
\end{table}

En el \cref{tab:longs cd} se muestran los resultados de calcular las longitudes de onda con la fórmula de Hartmann y las diferencias con las longitudes de onda teóricas ---los errores asociados se han calculado con la \cref{eqn:error longitud}---. Como vemos, los resultados se ajustan muy bien a los valores teóricos. Cabe subrayar el error y la discrepancia que aparecen asociados a la línea de longitud \( \lambda = \SI{643}{nm} \), que es mucho más grande que el resto. Esto es debido principalmente a que esta longitud corresponde a una desviación de \SI{65.42}{\degree}, que está fuera del rango de ángulos con los que se hizo el ajuste y por lo tanto el error estadístico aumenta. 

\section{Curva de dispersión del vidrio}
\begin{table}[htb]
	\centering \small \sffamily
	\caption{Distintos valores del índice de refracción para diferentes longitudes de onda}
	\label{tab:refraccion}
	\begin{tabular}{@{}SScS[table-parse-only]@{}}
	\toprule
	{\( \delta_m \) (\si{\degree})} & {\( \lambda \) (\si{\angstrom})} & {Color} & {\( n \)} \\
	\midrule
	74.107 & 4046.6 & Morado & 1.84 \pm 0.01 \\
	73.861 & 4078.2 & Morado & 1.84 \pm 0.03 \\
	71.744 & 4358.3 & Azul & 1.82 \pm 0.005 \\
	68.974 & 4916.0 & Verde-azul & 1.80 \pm 0.03 \\
	67.278 & 5460.7 & Verde & 1.79 \pm 0.02 \\
	66.569 & 5769.6 & Amarillo & 1.79 \pm 0.02 \\
	66.499 & 5789.7 & Amarillo & 1.786 \pm 0.007 \\
	71.403 & 4413   & Morado & 1.82 \pm 0.01 \\
	69.953 & 4678.2 & Azul & 1.812 \pm 0.005 \\
	69.432 & 4799.9 & Azul & 1.81 \pm 0.01 \\
	68.361 & 5085.8 & Verde & 1.800 \pm 0.005 \\
	68.149 & 5155.1 & Verde & 1.80 \pm 0.01 \\
	65.419 & 6438.5 & Rojo & 1.78 \pm 0.02 \\
	\bottomrule
\end{tabular}
\end{table}

Existe una relación experimental entre el índice de refracción de un material a una cierta longitud de onda y el ángulo de desviación mínima a sea mísma desviación,
\begin{equation} \label{eqn:dispersión}
	n(\lambda) = \frac{\sin{\tfrac{1}{2}(\delta_m + \alpha)}}{\sin{\tfrac{1}{2}\alpha}}
\end{equation}
donde \( \alpha \) es el ángulo de birefringencia del prisma. El ángulo de birefringencia del prisma en cuestión es de \SI{60}{\degree} por lo que
\begin{equation*}
	n(\lambda) = 2 \sin{\left(\tfrac{1}{2}\delta_m + \SI{30}{\degree}\right)}.
\end{equation*}
En el \cref{tab:refraccion} se muestran los índices de refracción obtenidos a partir los valores del ángulo de desviación mínima medidos. Las incertidumbres se han calculado con la \cref{eqn:refraccion}. Con estos resultados podemos calcular el número de Abbe del material del prisma, que es
\begin{equation*}
	\nu_\text{Abbe} = \frac{n_\text{amarillo} - 1}{n_\text{azul} - n_\text{rojo}}.
\end{equation*}
Si tomamos la media de los índices de refracción correspondientes a longitudes de onda azules, amarillas y rojas entonces obtenemos
\begin{equation*}
	\nu_\text{Abbe} = \frac{\num{1.79 \pm 0.01} - 1}{\num{1.814 \pm 0.007} - \num{1.78 \pm 0.02}} = \num{23.2\pm 0.6}
\end{equation*}
donde el error se ha calculado mediante la \cref{eqn:error abbe}. Hemos obtenido un número de Abbe menor que 55, lo que indica que se trata de un vidrio flint. Esta clase de vidrios tienen alta dispersión.

\section{Observación de distintos espectros}
\subsection{Lámpara de sodio}
Al encenderse la lámpara de sodio se pueden ver una gran cantidad de líneas que se van atenuando conforme la lámpara se calienta. Durante este tiempo la lámpara no emite la luz anaranjada característica del sodio sino más bien luz de un color rojizo. Al cabo de unos minutos pasa a emitir la luz naranja característica y con el espectrómetro se puede ver una raya naranja intensa. Si la rendija se estrecha lo suficiente se puede comprovar que esta raya de hecho es un doblete, el doblete del sodio. La explicación de este cambio en el espectro es que la mayoría de lámapras de sodio contienen una mezcla de gases que sirven para calentar la lámpara hasta que el sodio empieza a evaporarse y a emitir, por lo que la luz que se emite al principio no proviene del sodio sino de la mezcla de otros gases.  

\subsection{Lámpara LED}
El espectro que emite la lámpara LED es continuo sin rayas. Esto indica que un LED emite en todas las frecuencias y que por lo tanto emite luz blanca. La tecnología LED hace uso de semiconductores para replicar el  espectro de emisión de un cuerpo negro.

\subsection{Lámpara incandescente}
El espectro que emite una lámpara incandescente es continuo desde el rojo hasta llegar aproximadamente al verde. Lo que vemos, pues, es que no emite a frecuencias altas por lo que la luz no es blanca. Efectivamente, una lámpara incandescente emite luz anaranjada, indicativo de que la luz de altas frecuencias no está presente. Las lámparas incandescentes funcionan gracias a la emisión de energía de un filamento de tungsteno u otros materiales que es calentado por efecto Joule. El mecanismo, pues, no se basa en la emisión por transiciones energéticas de los átomos por lo que su espectro es contínuo.

\subsection{Fluorescente}
En el espectro del fluorescente se pueden obsevar cuatro rayas bien definidas: un doblete amarillo tenue, una raya verde intensa y una raya morada intensa. Subyacente a estas rayas se puede ver una franja continua muy ténue. Vemos, pues, que el espectro de un fluorescente no es contínuo. Las lámparas fluoresecentes funcionan por la excitación de un gas, típicamente mercurio, por lo que tienen un espectro de emisión discreto.

\section{Conclusiones}
Hemos realizado observaciones con un espectrómetro basado en la refracción de un prisma. Con el montaje hemos podido estudiar los fenómenos de la refracción y de la desviación mínima y como varían a causa de la dispersión cromática del prisma. Este último efecto es precisamente el que nos permite observar los distintos componentes del espectro de emisión de varias fuentes.

Disponiendo de un espectro conocido, el del mercurio, hemos calibrado el espectrómetro ajustando los coeficientes en la fórmula de Abbe. El ajuste ha sido lo suficientemente bueno como para obtener las longitudes de onda de un segundo espectro, el del cadmio, con precisión del orden de de nanómetros. Sin embargo, hemos observado que si intentamos determinar longitudes de onda fuera del rango en el que hemos hecho el ajuste nos encotramos con que el error aumenta, tal y como ha ocurrido con la raya roja del cadmio.

Con el espectrómetro también hemos podido observar el espectro de distintas fuentes de luz como una lámpara de sodio, una lámpara incandescente, una lámpara LED y un fluorescente. Hemos podido distinguir entre espectros continuos y discretos así como establecer la relación entre la luz que emite cada fuente y los rasgos más característicos de su espectro.

\newpage
\appendix
\section{Cálculo de las incertidumbres}\label{sec:errores}
A continuación se detallan los cálculos realizados para estimar las incertidumbres en los resultados.

En primer lugar, dado que se tomaron más de una medida para cada ángulo de desviación mínima, se tomó como valor su media y entonces el error viene dado por
\begin{equation} \label{eqn:error desviacion}
	u(\delta)^2 = \frac{s^2_{\delta} + u_\text{exp}(\delta)^2}{n}
\end{equation}
donde \( s^2_{\delta} \) es la variancia muestral de los datos tomados, \( n \) el número de datos tomados y \( u_\text{exp}(\delta) \) la precisión en la medida. Con el goniómetre que se utilizó se podían realizar medidas con una precisión de medio minuto de ángulo, aproximadamente \( \SI{0.008}{\degree} \).

Con la serie de desviación mínima del espectro del mercurio se realizó un ajuste para encontrar los parámetros en la fórmula de Hartmann, cuyos errores asociados figuran en la \cref{eqn:parametros hartmann}.

A continuación discutimos la estimación de las incertidumbres asociadas a la determinación de las longitudes de onda del espectro del cadmio. Las calculamos usando las medidas del ángulo de desviación mínima del cadmio usando la fórmula de Hartmann con los parámetros estimados. Así pues, su error asociado tendrá dos componentes, una parte debida a la propagación del error en la medida de la desviación mínima a través de la fórmula de Hartmann y otra asociada a las fluctuaciones estadísticas del ajuste. Entonces tenemos
\begin{equation} \label{eqn:error longitud}
	u(\lambda(\delta))^2 = \frac{\sigma^2}{n(\delta)} + \left(\frac{C}{(\delta - \delta_0)^2}\right)^2 u(\delta)^2,
\end{equation}
donde \( \sigma^2 \) es la variancia de las fluctuaciones estadísticas, \( n(\delta) \) es el tama~o efectivo de la muestra y \( u(\delta) \) es el error en el ángulo, calculado según la \cref{eqn:error desviacion}. El segundo término es el que corresponde a la propagación del error en el ángulo. El primero es el que va asociado a fluctuaciones estadísticas. Por un lado tenemos \( \sigma^2 \), que se calcula como
\begin{equation*}
	\sigma^2 = \frac{1}{n-2} \sum_{k = 1}^{n} (\lambda_k - \lambda(\delta_k))^2,
\end{equation*}
donde \( \lambda_k \) son las longitudes de onda teóricas del espectro del mercurio y \( \lambda(\delta_k) \) son las correspondientes longitudes calculadas con Hartmann. Finalmente \( n \) es el número de franjas observadas (7 en este caso). Por otra parte tenemos el tamaño efectivo de la muestra:
\begin{equation*}
	\frac{1}{n(\delta)} = \frac{1}{n} + \frac{(\delta - \bar{\delta})^2}{\sum_{k = 1}^{n}(\delta_k - \bar{\delta})^2 },
\end{equation*}
donde \( \delta_k \) son las deviaciones mínimas medidas para el mercurio y \( \bar{\delta} \) su promedio. Vemos que como más nos alejamos del rango donde hemos hecho el ajuste el tamaño efectivo de la muestra aumenta y por lo tanto incrementa el error estadístico.

La incertidumbre asociada al índice de refracción viene dada por
\begin{equation} \label{eqn:refraccion}
	u(n) = 2\cos{\left(\tfrac{1}{2}\delta + \SI{30}{\degree}\right)} u(\delta).
\end{equation}

El error en el cálculo del número de Abbe es de
\begin{equation} \label{eqn:error abbe}
	u(\nu_\text{Abbe})^2 = \frac{u(n_\text{amarillo})^2}{(n_\text{azul} - n_\text{rojo})^2} + \left(\frac{n_\text{amarillo} - 1}{(n_\text{azul} - n_\text{rojo})^2}\right)^2(u(n_\text{azul})^2 + u(n_\text{rojo})^2).
\end{equation}

\newpage
\section{Datos experimentales}
\begin{table}[htb]
	\footnotesize \centering \sffamily
	\begin{minipage}{0.45\textwidth}
		\centering
		\caption{Medidas de la desviación mínima de cada franja del espectro del mercurio}
		\label{tab:datos Hg}
		\begin{tabular}{@{}SS@{}}
			\toprule
			{\( \lambda \) (\si{\angstrom})} & {\( \delta \) (\unc{0.008}{\degree})} \\
			\midrule
			4046.60 & 35.950 \\ 
 			{} & 35.925 \\
			\midrule
			4078.20 & 36.150 \\ 
 			{} & 36.217 \\
			\midrule
			4358.30 & 38.300 \\ 
 			{} & 38.300 \\
			\midrule
			4916.00 & 41.042 \\ 
 			{} & 41.100 \\
			\midrule
			5460.70 & 42.783 \\ 
 			{} & 42.750 \\
			\midrule
			5769.60 & 43.450 \\ 
 			{} & 43.500 \\
			\midrule
			5789.70 & 43.550 \\ 
 			{} & 43.542 \\
			\bottomrule
		\end{tabular}
	\end{minipage}
	\hfill
	\begin{minipage}{0.45\textwidth}
		\centering
		\caption{Medidas de la desviación mínima de cada franja del espectro del cadmio}
		\label{tab:datos Cd}
		\begin{tabular}{@{}SS@{}}
			\toprule
			{\( \lambda \) (\si{\angstrom})} & {\( \delta \) (\unc{0.008}{\degree})} \\
			\midrule
			4413.00 & 38.625 \\ 
 			{} & 38.658 \\
			\midrule
			4678.20 & 40.092 \\ 
 			{} & 40.092 \\
			\midrule
			4799.90 & 40.600 \\ 
 			{} & 40.625 \\
			\midrule
			5085.80 & 41.683 \\ 
 			{} & 41.683 \\
			\midrule
			5155.10 & 41.883 \\ 
 			{} & 41.908 \\
			\midrule
			6438.50 & 44.600 \\ 
 			{} & 44.650 \\
 			\bottomrule
		\end{tabular}
	\end{minipage}
\end{table}

\newpage\null\newpage

\end{document}

