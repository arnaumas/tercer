\documentclass[12pt]{article}

\usepackage[utf8]{inputenc}
\usepackage[T1]{fontenc}
\usepackage[catalan]{babel}
\usepackage{lmodern}
\usepackage{geometry}
\usepackage{hyperref}
\usepackage[dvipsnames]{xcolor}
\usepackage[bf,sf,pagestyles]{titlesec}
\usepackage{titling}
\usepackage[font={footnotesize, sf}, labelfont=bf]{caption} 
\usepackage{siunitx}
\usepackage{graphicx}
\usepackage{booktabs}
\usepackage{amsmath,amssymb}
\usepackage[catalan,sort]{cleveref}
\usepackage{enumitem}

\geometry{
	a4paper,
	right = 2.5cm,
	left = 2.5cm,
	bottom = 3cm,
	top = 3cm
}

\hypersetup{
	colorlinks,
	linkcolor = {red!50!blue},
	citecolor = {red!50!blue},
	linktoc = page
}

\crefname{figure}{figura}{figures}
\crefname{table}{taula}{taules}
\numberwithin{table}{section}
\numberwithin{figure}{section}
\numberwithin{equation}{section}

\graphicspath{{./figs/}}

% Unitats
\sisetup{
	inter-unit-product = \ensuremath{ \, },
	allow-number-unit-breaks = true,
	math-celsius = {}^{\circ}\kern-\scriptspace C,
	detect-family = true,
	detect-shape = true,
	list-final-separator = { i },
	list-pair-separator = { i },
	list-units = single,
	separate-uncertainty = true
}

\DeclareMathAlphabet{\mathsfit}{T1}{\sfdefault}{\mddefault}{\sldefault}

\newcommand{\Z}{\mathbb{Z}}
\newcommand{\N}{\mathbb{N}}
\newcommand{\R}{\mathbb{R}}
\newcommand{\Ry}{\mathit{Ry}}
\newcommand{\data}[3]{\SI{#1 \pm #2}{#3}}
\newcommand{\unc}[2]{\ensuremath{{}\pm \SI{#1}{#2}}}
\DeclareMathOperator{\gr}{gr}
\newcommand{\abs}[1]{\left\lvert #1 \right\rvert}
\newcommand{\inn}[2]{\left\langle #1 , #2 \right\rangle}
\newcommand{\parbreak}{
	\begin{center}
		--- $\ast$ ---
	\end{center} 
}
\makeatletter
\newcommand*{\defeq}{\mathrel{\rlap{%
			\raisebox{0.3ex}{$\m@th\cdot$}}%
		\raisebox{-0.3ex}{$\m@th\cdot$}}%
	=
}
\makeatother

\newpagestyle{pagina}{
	\headrule
	\sethead*{\sffamily \bfseries Informe IV}{}{\theauthor}
	\footrule
	\setfoot*{}{}{\sffamily \thepage}
}
\renewpagestyle{plain}{
	\footrule
	\setfoot*{}{}{\sffamily \thepage}
}
\pagestyle{pagina}

\title{\sffamily {\bfseries Informe IV:} Gasos reals. Isotermes d'Andrews}
\author{\sffamily A7: Arnau Mas}
\date{\sffamily 12 de novembre de 2018}

\begin{document}
\maketitle
\section{Introducció}
En aquesta pràctica s'estudia el procés de compressió isotèrmica d'un gas no ideal en un règim de volums molars i pressions on no són aplicables ni l'equació dels gasos ideals ni tampoc l'equació de Van der Waals. Sabem que per a volums molars grans qualsevol graf es pot considerar com ideal. Per a volums molars petits l'equació de Van der Waals passa a ser una millor aproximació. Tot i així, quan el volum molar baixa d'un cert llindar s'observa una transició de fase que no es pot explicar mitjançant l'equació de Van der Waals. Mentre dura el canvi de fase, la pressió es manté constant. Amb les mesures d'aquest experiment s'observa aquest comportament.    

Per a realitzar les mesures es disposa d'una quantitat de hexafluorur de sofre tancat dins d'un tub de vidre per una columna de mercuri. L'alçada de la columna de mercuri es pot variar de manera precisa, podent controlar el volum del gas. El tub està envoltat d'un bany tèrmic per assegurar que totes les mesures es realitzen de forma isoterma. La temperatura del bany es pot controlar de manera que podem fer mesures per a diferents isotermes. 

\end{document}
