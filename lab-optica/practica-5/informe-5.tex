\documentclass[12pt]{article}

\usepackage[utf8]{inputenc}
\usepackage[T1]{fontenc}
\usepackage[catalan]{babel}
\usepackage{lmodern}
\usepackage{geometry}
\usepackage{hyperref}
\usepackage[dvipsnames]{xcolor}
\usepackage[bf,sf,pagestyles]{titlesec}
\usepackage{titling}
\usepackage[font={footnotesize, sf}, labelfont=bf]{caption} 
\usepackage{siunitx}
\usepackage{graphicx}
\usepackage{booktabs}
\usepackage{amsmath,amssymb}
\usepackage[catalan,sort]{cleveref}
\usepackage{enumitem}

\geometry{
	a4paper,
	right = 2.5cm,
	left = 2.5cm,
	bottom = 3cm,
	top = 3cm
}

\hypersetup{
	colorlinks,
	linkcolor = {red!50!blue},
	citecolor = {red!50!blue},
	linktoc = page
}

\crefname{figure}{figura}{figures}
\crefname{table}{taula}{taules}
\numberwithin{table}{section}
\numberwithin{figure}{section}
\numberwithin{equation}{section}

\graphicspath{{./figs/}}

% Unitats
\sisetup{
	inter-unit-product = \ensuremath{ \, },
	allow-number-unit-breaks = true,
	math-celsius = {}^{\circ}\kern-\scriptspace C,
	detect-family = true,
	detect-shape = true,
	list-final-separator = { i },
	list-pair-separator = { i },
	list-units = single,
	separate-uncertainty = true
}

\DeclareMathAlphabet{\mathsfit}{T1}{\sfdefault}{\mddefault}{\sldefault}

\newcommand{\Z}{\mathbb{Z}}
\newcommand{\N}{\mathbb{N}}
\newcommand{\R}{\mathbb{R}}
\newcommand{\Ry}{\mathit{Ry}}
\newcommand{\data}[3]{\SI{#1 \pm #2}{#3}}
\newcommand{\unc}[2]{\ensuremath{{}\pm \SI{#1}{#2}}}
\DeclareMathOperator{\gr}{gr}
\newcommand{\abs}[1]{\left\lvert #1 \right\rvert}
\newcommand{\inn}[2]{\left\langle #1 , #2 \right\rangle}
\newcommand{\parbreak}{
	\begin{center}
		--- $\ast$ ---
	\end{center} 
}
\makeatletter
\newcommand*{\defeq}{\mathrel{\rlap{%
			\raisebox{0.3ex}{$\m@th\cdot$}}%
		\raisebox{-0.3ex}{$\m@th\cdot$}}%
	=
}
\makeatother

\newpagestyle{pagina}{
	\headrule
	\sethead*{\sffamily \bfseries Informe IV}{}{\theauthor}
	\footrule
	\setfoot*{}{}{\sffamily \thepage}
}
\renewpagestyle{plain}{
	\footrule
	\setfoot*{}{}{\sffamily \thepage}
}
\pagestyle{pagina}

\title{\sffamily {\bfseries Informe IV:} Gasos reals. Isotermes d'Andrews}
\author{\sffamily A7: Arnau Mas}
\date{\sffamily 12 de novembre de 2018}

\begin{document}
\maketitle
\section{Introducció}
En aquesta pràctica s'estudia el procés de compressió isotèrmica d'un gas no ideal en un règim de volums molars i pressions on no són aplicables ni l'equació dels gasos ideals ni tampoc l'equació de Van der Waals. Sabem que per a volums molars grans qualsevol graf es pot considerar com ideal. Per a volums molars petits l'equació de Van der Waals passa a ser una millor aproximació. Tot i així, quan el volum molar baixa d'un cert llindar s'observa una transició de fase que no es pot explicar mitjançant l'equació de Van der Waals. Mentre dura el canvi de fase, la pressió es manté constant. Amb les mesures d'aquest experiment s'observa aquest comportament.    

Per a realitzar les mesures es disposa d'una quantitat de hexafluorur de sofre tancat dins d'un tub de vidre per una columna de mercuri. L'alçada de la columna de mercuri es pot variar de manera precisa, podent controlar el volum del gas. El tub està envoltat d'un bany tèrmic per assegurar que totes les mesures es realitzen de forma isoterma. La temperatura del bany es pot controlar de manera que podem fer mesures per a diferents isotermes. 

\section{Resultats i discussió}
\subsection{Anàlisi de les isotermes}
\begin{figure}[htb]
	\small \sffamily \centering
	\input{figs/isotermes}
	\caption{Representació de les diverses isotermes mesurades. Les línies que conecten els punts no representen cap dada, només hi són per ajudar a distingir cada isoterma}
	\label{fig:isotermes}
\end{figure}

\begin{figure}[htb]
	\small \sffamily \centering
	\input{figs/saturacio}
	\caption{Punts de la línia de saturació. Cada punt es calcula fent la mitjana entre el punt immediatament anterior a l'observació de les dues fases i l'immediatament posterior. També es representa la regressió polinòmica realitzada}
	\label{fig:saturacio}
\end{figure}

A la \cref{fig:isotermes} es mostren les dades recollides durant l'experiment ---les \crefrange{tab:T1}{tab:T8} mostren aquestes dades en detall---. Els punts que corresponen a una mateixa isoterma estan conectats. Podem apreciar com, per a volums grans, el comportament del sistema és el d'un gas usual ---la pressió decreix amb el volum---, però quan creuem un cert llindar la pressió passa a ser constant mentre el sistema experimenta un canvi de fase. Quan tota la fase gasosa s'ha condensat, la pressió incrementa ràpidament degut a la naturalesa incompressible dels líquids. Les regions de pressió constant es van fent més i més petites a mesura que ens apropem a la temperatura crítica, \( T_c \).  

\subsection{Determinació de la corva de saturació i del punt crític} \label{sec:saturacio}
Per a determinar el punt crític de la substància que estem estudiant ---en aquest cas és hexafluorur de sofre, \( \mathrm{SF}_6 \)--- hem de primer establir la seva corva de saturació. Aquesta és la frontera de la regió on observem coexistència de fases. Com que tenim dades tant del volum total com del volum de la fase gasosa, sabem quan comença a apareixer fase líquida, que és, naturalment, quan el volum de la fase gasosa és inferior al volum total. Aproximarem els punts de la corva de saturació prenent la mitjana dels punts que estan immediatament abans i després d'aquest fenòmen. És a dir, per a cada isoterma fem la mitjana entre l'últim punt on només observem fase gasosa i el primer punt on apareix fase líquida. Aquest càlcul dóna lloc als punts de la \cref{fig:saturacio}, que es mostren sobreimpressionats a les isotermes experimentals per a contextualitzar.

Seguidament interpolem la corva de saturació a partir dels punts obtinguts. Concretament fem una aproximació polinòmica de grau 6 per mínims quadrats. D'aquesta manera podem trobar una aproximació per a la pressió i volum crítics, que seran les coordenades del màxim de la interpolació ---sabem que la corva de saturació presenta un màxim al punt crític---. Així doncs obtenim
\begin{equation} \label{eq:punt critic}
	\begin{aligned}
		V_c = \data{0.8}{0.1}{cm^3} \\
		P_c = \data{43}{1e5}{Pa}.
	\end{aligned}
\end{equation}
Les incerteses donades són una estimació heurística considerant que el procés d'interpolar la corva de saturació fa créixer les incerteses de les mesures aproximadament en un ordre de magnitud.

\subsection{Determinació del volum molar crític} \label{sec:mols}
\begin{figure}[htb]
	\small \sffamily \centering
	\input{figs/pvs}
	\caption{Representació de \( \mathsfit{PV} \) en funció de \( \mathsfit{V^{-1}} \) per a poder determinar el nombre de mols del sistema}
	\label{fig:pvs}
\end{figure}

Per a calcular el nombre de mols de \( \mathrm{SF}_6 \) presents a l'experiment a partir de les dades donades podem fer servir l'expansió del virial de l'equació d'estat,
\begin{equation*}
	P(V, T) = RT \left( \frac{n}{V} + \sum_{k = 2}^\infty A_k(T) \left(\frac{n}{V}\right)^k\right),
\end{equation*}
on \( A_k(T) \) són coeficients que depenen de la temperatura. Per a volums molars grans podem aproximar fins al segon ordre i queda
\begin{equation*}
	PV = RT\left(n + \frac{n^2A_2(T)}{V}\right). 
\end{equation*}
Així, si fem una regressió lineal entre \( PV \) i \( V^{-1} \) tindrem que la seva ordenada a l'origen serà \( nRT \). Així obtenim un valor de \( n \) per a cada isoterma i podem prendre'n la mitjana. 

A la figura \cref{fig:pvs} es mostra \( PV \) en funció de \( V^{-1} \). Les regressions s'han fet amb les dades que satisfan \( V^{-1} \leq \SI{1}{cm^{-3}} \), que és la regió on el comportament és gairebé lineal. El nombre de mols obtingut és 
\begin{equation*}
	n = \data{4.0}{0.1e-3}{mol}.
\end{equation*}
Per a estimar aquesta incertesa podem observar que només propagar la incertesa instrumental associada a \( T \) ja dóna un resultat de l'ordre de \( \SI{e-4}{mol} \). 

Coneixent el nombre de mols podem determinar el volum molar crític, \( v_c \):
\begin{equation*}
 	v_c = \frac{V_c}{n} = \data{200}{30}{cm^3.mol^{-1}}.
\end{equation*}
A \cite{nist} es dóna un valor de \data{198.0}{0.4}{cm^3 . mol^{-1}} per al volum crític, que és compatible amb el resultat obtingut. 

\subsection{Determinació de la temperatura crítica}
\begin{figure}[htb]
	\small \sffamily \centering
	\input{figs/pts}
	\caption{Representació del producte de la pressió pel volum en funció de l'invers del volum per a poder determinar el nombre de mols del sistema}
	\label{fig:pts}
\end{figure}

Per acabar d'especificar el punt crític ens falta saber-ne la temperatura. A la segona part de l'experiment es van pendre mesures de pressió i temperatura per a un procés isòcor ---veure \cref{tab:PT}---. Tal i com s'aprecia a la \cref{fig:pts}, la relació és essencialment lineal. Fent una regressió determinem els coeficients de la recta de regressió \( a \) i \( b \), que també es mostra a la \cref{fig:pts}. Així podem calcular la temperatura crítica com
\begin{equation*}
	T_c = \frac{P_c - b}{a}.
\end{equation*}
Si fem aquest càlcul amb la pressió que hem obtingut a la \cref{eq:punt critic} aleshores \( T_c = \data{55}{1}{\celsius} \). Aquest valor no és coherent amb els resultats ja que, tal i com intuïm a les \cref{fig:isotermes,fig:saturacio}, el punt crític ha de pertànyer a una isoterma de temperatura entre els \SI{40}{\celsius} i \SI{45}{\celsius}. Això és probablement degut a la interpolació de la corva de saturació. Si bé és clar que el volum crític ha d'estar situat entre els \SI{0.8}{cm^3} i \SI{0.6}{cm^3}, no resulta tant evident per a la pressió. Quan es fan les mesures a isotermes properes al punt crític les fluctuacions de pressió són grans. A \cite{nist} es dóna una pressió crítica de \data{37.586}{0.005e5}{Pa}, inferior a la mesurada. Amb aquesta pressió s'obté una temperatura crítica de \data{44}{1}{\celsius}, que és molt més proper al resultat tabulat.    

\subsection{Isotermes de Van der Waals}
\begin{figure}[htb]
	\small \sffamily \centering
	\input{figs/vdw}
	\caption{Representació de les isotermes de Van der Waals del sistema, sobreimpressionades sobre les isotermes i punts de la corva de saturació experimentals}
	\label{fig:vdw}
\end{figure}

L'equació d'estat de Van der Waals és
\begin{equation*}
	\left(P + \frac{an^2}{V^2}\right)\left(\frac{V}{n} - b\right) = RT.
\end{equation*}
Per a determinar els coeficients \( a \) i \( b \) s'imposa que el punt crític que dóna l'equació de Van der Waals coincideixi amb el punt crític real. El punt crític satisfà
\begin{align*}
	P(V_c, T_c) &= P_c \\
	\frac{\partial}{\partial V} P(V_c, T_c) &= 0 \\
	\frac{\partial^2}{\partial V^2} P(V_c, T_c) &= 0.
\end{align*}
Això és perquè la isoterma crítica de Van der Waals és la que presenta un únic punt estacionari, mentre les que corresponen a temperatures inferiors presenten dos punts estacionaris ---i per tant una regió amb \( \partial_V P \) ngativa--- i les que corresponen a temperatures inferiors no tenen cap punt estacionari.  ngativa--- i les que corresponen a temperatures inferiors no tenen cap punt estacionari. 
Resolent aquestes equacions s'obtenen les relacions
\begin{align*}
	a &= \frac{27(RT_c)^2}{64P_c} \\
	b &= \frac{RT_c}{8P_c} = \frac{v_c}{3}. 
\end{align*}
Amb la pressió i temperatura crítiques que es donen a \cite{nist} resulta
\begin{gather*}
 a = \SI{7.81e6}{bar . cm^6 . mol^{-2}} \\
 b = \SI{87.74}{cm^3. mol^{-2}}. 
\end{gather*}
Per tant, el volum molar crític del gas de Van der Waals és de \SI{263.2}{cm^3.mol^{-1}}, que no coincideix ni amb el valor mesurat ni amb el valor real. 

\begin{figure}[htb]
	\small \sffamily \centering
	\input{figs/vdw1}
	\caption{Representació d'una isoterma experimental amb la corresponent isoterma de Van der Waals}
	\label{fig:vdw1}
\end{figure}

A la \cref{fig:vdw1} es mostra una isoterma experimental amb la seva corresponent isoterma de de Van der Waals per tal de comparar-les. Veiem que per a volums grans aquestes coincideixen força bé. En el moment en el que entrem a la regió de coexistència de fases, però, divergeixen. Hem observat que el es manté a pressió constant mentre dura el canvi de fase, la qual cosa no prediu l'equació de Van der Waals. El que sí que prediu, però, és que dins de la regió on hi ha un canvi de fase es té
\begin{equation*}
	\frac{\partial P}{\partial V} < 0,
\end{equation*}
que indica que el sistema es troba en inestabilitat. Això es tradueix a la realitat com el canvi de fase que s'observa.  

Una quantitat que dóna una idea del comportament d'un gas és 
\begin{equation*}
	c = \frac{Pv}{RT}.
\end{equation*}
Evidentment, si el gas és ideal, \( c \) és sempre 1. Si avaluem \( c \) al punt crític per al nostre sistema amb la pressió i temperatura crítica tabulades i el volum molar crític mesurat obtenim
\begin{equation*}
	c_\text{exp} \approx 0.28,
\end{equation*}
mentre que si fem el mateix càlcul fent servir el volum crític que prediu Van der Waals tenim
\begin{equation*}
c_\text{Van der Waals} = \frac{3P_cb}{RT_c} \approx 0.37. 
\end{equation*}
Així el comportament del gas prop del punt crític està allunyat del d'un gas ideal. 

Observem que per a un gas ideal \( c \) és precisament la seva compressibilitat isotèrmica. Si suposem que és una bona aproximació de la compressibilitat del gas al punt crític tenim que és menys compressible tant que un gas de Van der Waals com un gas ideal.  

\section{Conclusions}


\pagebreak
\bibliographystyle{ieeetr}
\bibliography{fonts}

\pagebreak
\appendix
\section{Avaluació d'incerteses}
A continuació es presenten les diverses expressions emprades per a estimar les incerteses associades als diversos càlculs realitzats. 
\begin{itemize}
	\item \emph{Punts de la corva de saturació}. Com es menciona a la \cref{sec:saturacio}, els punts de la corva de saturació es calculen fent la mitjana entre dos volums experimentals, posem \( V_1 \) i \( V_2 \). Així
		\begin{equation*}
			u(V_\text{sat}) = \sqrt{\frac{u(V_1)^2}{4} + \frac{u(V_2)^2}{4}} = \frac{u(V)}{\sqrt{2}},
		\end{equation*}
		on \( u(V) \) és la incertesa experimental en la mesura del volum (\unc{0.05}{cm^3}). Aquesta estimació, però, no té en compte que estem limitats per la separació entre mesures de volum successives, que és de \SI{0.2}{cm^3}, per tant s'ha pres un error de l'ordre de \SI{e-1}{cm^3}.
	\item \emph{Invers del volum i producte del volum i la pressió}. Tenim
		\begin{equation*}
			u\left(\frac{1}{V}\right) = \frac{u(V)}{V^2}
		\end{equation*}
		i
		\begin{equation*}
			u(PV) = \sqrt{P^2 u(V)^2 + V^2 u(P)^2}.
		\end{equation*}

	\item \emph{Nombre de mols}. Tal i com s'explica a la \cref{sec:mols}, el nombre de mols de substància es calculen observant que \( nRT \) és l'ordenada a l'origen d'una regressió lineal, \( a \). Si només considerem l'error provinent de \( T \) tenim
		\begin{equation*}
			u(n) = \frac{a}{RT^2}u(T).
		\end{equation*}
	\item \emph{Volum molar crític}. Trobem l'error associat al volum molar crític de manera directa a partir del seu càlcul:
		\begin{equation*}
			u(v_c) = \sqrt{\left(\frac{V_c u(n)}{n^2}\right)^2 + \left(\frac{u(V_c)^2}{n}\right)^2}
		\end{equation*}
		

\end{itemize}


\section{Regressions}
\begin{table}[htb]
	\sffamily \footnotesize \centering
	\caption{Dades per a les regressions emprades a la \cref{sec:mols} per a determinar el nombre de mols del sistema. \( \mathsfit{A} \) és l'ordenada a l'origen de cada regressió i \( \mathsfit n \) el corresponent nombre de mols}
	\label{tab:mols}
	\begin{tabular}{SS}
		\toprule
		{$\mathsfit A$ (\si{bar . cm^3})} & {$\mathsfit n$ (\SI{e3}{mol})}  \\
		\midrule
		89.8 & 3.8\\
		95.9 & 4.0\\
		99.0 & 4.1\\
		101.6 & 4.1\\
		102.3 & 4.1\\
		102.6 & 4.0\\
		102.5 & 3.9\\
		104.4 & 4.0\\	
		\bottomrule
	\end{tabular}
\end{table}

\section{Dades primàries}
\begin{table}[htb]
	\sffamily \footnotesize \centering
	\caption{Pressió, \( \mathsfit P \), volum total, \( \mathsfit V \), i volum de la fase gasosa, \( \mathsfit{V_g} \) per a la isoterma a \SI{10.9}{\celsius}}
	\label{tab:T1}
	\begin{tabular}{SSS}
		\toprule
		{$\mathsfit P$ (\unc{0.5 e5}{Pa})} & {$\mathsfit V$ (\unc{0.05}{cm^3})} & {$\mathsfit{ V_g}$ (\unc{0.05}{cm^3})} \\
		\midrule
		18.5 & 4.00 & 4.00\\
		19.0 & 3.80 & 3.80\\
		19.5 & 3.60 & 3.60\\
		19.5 & 3.40 & 3.35\\
		19.5 & 3.20 & 3.15\\
		19.5 & 3.00 & 2.90\\
		19.5 & 2.80 & 2.70\\
		19.5 & 2.60 & 2.45\\
		19.5 & 2.40 & 2.25\\
		19.5 & 2.20 & 2.05\\
		19.5 & 2.00 & 1.80\\
		20.0 & 1.80 & 1.60\\
		20.0 & 1.60 & 1.35\\
		20.0 & 1.40 & 1.15\\
		20.0 & 1.20 & 0.90\\
		20.0 & 1.00 & 0.70\\
		20.0 & 0.80 & 0.45\\
		20.5 & 0.60 & 0.25\\
		20.5 & 0.50 & 0.15\\
		24.5 & 0.45 & 0.00\\
		45.0 & 0.40 & 0.00\\		
		\bottomrule
	\end{tabular}
\end{table}

\begin{table}[htb]
	\sffamily \footnotesize \centering
	\caption{Pressió, \( \mathsfit P \), volum total, \( \mathsfit V \), i volum de la fase gasosa, \( \mathsfit{V_g} \) per a la isoterma a \SI{15.5}{\celsius}}
	\label{tab:T2}
	\begin{tabular}{SSS}
		\toprule
		{$\mathsfit P$ (\unc{0.5 e5}{Pa})} & {$\mathsfit V$ (\unc{0.05}{cm^3})} & {$\mathsfit{ V_g}$ (\unc{0.05}{cm^3})} \\
		\midrule
		19.0 & 4.00 & 4.00\\
		19.5 & 3.80 & 3.80\\
		20.5 & 3.60 & 3.60\\
		21.0 & 3.40 & 3.40\\
		21.5 & 3.20 & 3.20\\
		22.0 & 3.00 & 3.00\\
		22.0 & 2.80 & 2.75\\
		22.0 & 2.60 & 2.50\\
		22.0 & 2.40 & 2.30\\
		22.0 & 2.20 & 2.05\\
		22.0 & 2.00 & 1.85\\
		22.0 & 1.80 & 1.60\\
		22.0 & 1.60 & 1.40\\
		22.0 & 1.40 & 1.15\\
		22.0 & 1.20 & 0.95\\
		22.0 & 1.00 & 0.70\\
		22.0 & 0.80 & 0.50\\
		22.5 & 0.60 & 0.25\\
		26.0 & 0.45 & 0.00\\
		48.0 & 0.40 & 0.00\\		
		\bottomrule
	\end{tabular}
\end{table}

\begin{table}[htb]
	\sffamily \footnotesize \centering
	\caption{Pressió, \( \mathsfit P \), volum total, \( \mathsfit V \), i volum de la fase gasosa, \( \mathsfit{V_g} \) per a la isoterma a \SI{20.7}{\celsius}}
	\label{tab:T3}
	\begin{tabular}{SSS}
		\toprule
		{$\mathsfit P$ (\unc{0.5 e5}{Pa})} & {$\mathsfit V$ (\unc{0.05}{cm^3})} & {$\mathsfit{ V_g}$ (\unc{0.05}{cm^3})} \\
		\midrule
		19.5 & 4.00 & 4.00\\
		20.5 & 3.80 & 3.80\\
		21.0 & 3.60 & 3.60\\
		21.5 & 3.40 & 3.40\\
		22.5 & 3.20 & 3.20\\
		23.0 & 3.00 & 3.00\\
		24.0 & 2.80 & 2.80\\
		24.5 & 2.60 & 2.55\\
		24.5 & 2.40 & 2.35\\
		24.5 & 2.20 & 2.10\\
		24.5 & 2.00 & 1.90\\
		24.5 & 1.80 & 1.65\\
		24.5 & 1.60 & 1.40\\
		24.5 & 1.40 & 1.20\\
		25.0 & 1.20 & 0.95\\
		25.0 & 1.00 & 0.70\\
		25.0 & 0.80 & 0.45\\
		25.0 & 0.60 & 0.25\\
		44.5 & 0.40 & 0.00\\
		\bottomrule
	\end{tabular}
\end{table}

\begin{table}[htb]
	\sffamily \footnotesize \centering
	\caption{Pressió, \( \mathsfit P \), volum total, \( \mathsfit V \), i volum de la fase gasosa, \( \mathsfit{V_g} \) per a la isoterma a \SI{25.6}{\celsius}}
	\label{tab:T4}
	\begin{tabular}{SSS}
		\toprule
		{$\mathsfit P$ (\unc{0.5 e5}{Pa})} & {$\mathsfit V$ (\unc{0.05}{cm^3})} & {$\mathsfit{ V_g}$ (\unc{0.05}{cm^3})} \\
		\midrule
		20.5 & 4.00 & 4.00\\
		21.0 & 3.80 & 3.80\\
		21.5 & 3.60 & 3.60\\
		22.3 & 3.40 & 3.40\\
		23.0 & 3.20 & 3.20\\
		24.0 & 3.00 & 3.00\\
		25.0 & 2.80 & 2.80\\
		26.0 & 2.60 & 2.60\\
		27.0 & 2.40 & 2.40\\
		27.0 & 2.20 & 2.15\\
		27.0 & 2.00 & 1.90\\
		27.0 & 1.80 & 1.65\\
		27.0 & 1.60 & 1.45\\
		27.0 & 1.40 & 1.20\\
		27.5 & 1.20 & 0.95\\
		27.5 & 1.00 & 0.70\\
		27.5 & 0.80 & 0.35\\
		28.0 & 0.60 & 0.10\\
		47.0 & 0.40 & 0.00\\		
		\bottomrule
	\end{tabular}
\end{table}

\begin{table}[htb]
	\sffamily \footnotesize \centering
	\caption{Pressió, \( \mathsfit P \), volum total, \( \mathsfit V \), i volum de la fase gasosa, \( \mathsfit{V_g} \) per a la isoterma a \SI{30.3}{\celsius}}
	\label{tab:T5}
	\begin{tabular}{SSS}
		\toprule
		{$\mathsfit P$ (\unc{0.5 e5}{Pa})} & {$\mathsfit V$ (\unc{0.05}{cm^3})} & {$\mathsfit{ V_g}$ (\unc{0.05}{cm^3})} \\
		\midrule
		21.0 & 4.00 & 4.00\\
		21.5 & 3.80 & 3.80\\
		22.0 & 3.60 & 3.60\\
		23.0 & 3.40 & 3.40\\
		24.0 & 3.20 & 3.20\\
		24.5 & 3.00 & 3.00\\
		25.5 & 2.80 & 2.80\\
		26.5 & 2.60 & 2.60\\
		27.5 & 2.40 & 2.40\\
		29.0 & 2.20 & 2.20\\
		30.0 & 2.00 & 2.00\\
		30.0 & 1.80 & 1.70\\
		30.0 & 1.60 & 1.50\\
		30.0 & 1.40 & 1.25\\
		30.0 & 1.20 & 1.00\\
		30.5 & 1.00 & 0.75\\
		30.0 & 0.80 & 0.45\\
		30.0 & 0.60 & 0.25\\
		31.0 & 0.50 & 0.15\\
		42.5 & 0.45 & 0.00\\
		45.0 & 0.40 & 0.00\\		
		\bottomrule
	\end{tabular}
\end{table}

\begin{table}[htb]
	\sffamily \footnotesize \centering
	\caption{Pressió, \( \mathsfit P \), volum total, \( \mathsfit V \), i volum de la fase gasosa, \( \mathsfit{V_g} \) per a la isoterma a \SI{35.3}{\celsius}}
	\label{tab:T6}
	\begin{tabular}{SSS}
		\toprule
		{$\mathsfit P$ (\unc{0.5 e5}{Pa})} & {$\mathsfit V$ (\unc{0.05}{cm^3})} & {$\mathsfit{ V_g}$ (\unc{0.05}{cm^3})} \\
		\midrule
		21.5 & 4.00 & 4.00\\
		22.0 & 3.80 & 3.80\\
		22.5 & 3.60 & 3.60\\
		23.5 & 3.40 & 3.40\\
		24.5 & 3.20 & 3.20\\
		25.5 & 3.00 & 3.00\\
		26.5 & 2.80 & 2.80\\
		27.5 & 2.60 & 2.60\\
		28.5 & 2.40 & 2.40\\
		30.0 & 2.20 & 2.20\\
		31.0 & 2.00 & 2.00\\
		32.5 & 1.80 & 1.80\\
		33.5 & 1.60 & 1.55\\
		33.5 & 1.40 & 1.30\\
		33.5 & 1.20 & 1.10\\
		33.5 & 1.00 & 0.75\\
		33.5 & 0.80 & 0.50\\
		34.0 & 0.60 & 0.20\\
		42.0 & 0.45 & 0.00\\
		49.5 & 0.40 & 0.00\\		
		\bottomrule
	\end{tabular}
\end{table}

\begin{table}[htb]
	\sffamily \footnotesize \centering
	\caption{Pressió, \( \mathsfit P \), volum total, \( \mathsfit V \), i volum de la fase gasosa, \( \mathsfit{V_g} \) per a la isoterma a \SI{40.0}{\celsius}}
	\label{tab:T7}
	\begin{tabular}{SSS}
		\toprule
		{$\mathsfit P$ (\unc{0.5 e5}{Pa})} & {$\mathsfit V$ (\unc{0.05}{cm^3})} & {$\mathsfit{ V_g}$ (\unc{0.05}{cm^3})} \\
		\midrule
		21.5 & 4.00 & 4.00\\
		22.5 & 3.80 & 3.80\\
		23.0 & 3.60 & 3.60\\
		24.0 & 3.40 & 3.40\\
		25.0 & 3.20 & 3.20\\
		26.0 & 3.00 & 3.00\\
		27.0 & 2.80 & 2.80\\
		28.5 & 2.60 & 2.60\\
		30.0 & 2.40 & 2.40\\
		31.0 & 2.20 & 2.20\\
		32.5 & 2.00 & 2.00\\
		34.0 & 1.80 & 1.80\\
		35.5 & 1.60 & 1.60\\
		36.5 & 1.40 & 1.40\\
		37.0 & 1.20 & 1.10\\
		37.0 & 1.00 & 0.80\\
		37.0 & 0.80 & 0.50\\
		37.5 & 0.60 & 0.20\\
		46.0 & 0.45 & 0.00\\		
		\bottomrule
	\end{tabular}
\end{table}

\begin{table}[htb]
	\sffamily \footnotesize \centering
	\caption{Pressió, \( \mathsfit P \), volum total, \( \mathsfit V \), i volum de la fase gasosa, \( \mathsfit{V_g} \) per a la isoterma a \SI{44.8}{\celsius}}
	\label{tab:T8}
	\begin{tabular}{SSS}
		\toprule
		{$\mathsfit P$ (\unc{0.5 e5}{Pa})} & {$\mathsfit V$ (\unc{0.05}{cm^3})} & {$\mathsfit{ V_g}$ (\unc{0.05}{cm^3})} \\
		\midrule
		22.3 & 4.00 & 4.00\\
		23.0 & 3.80 & 3.80\\
		24.0 & 3.60 & 3.60\\
		24.5 & 3.40 & 3.40\\
		25.5 & 3.20 & 3.20\\
		27.0 & 3.00 & 3.00\\
		28.0 & 2.80 & 2.80\\
		29.0 & 2.60 & 2.60\\
		30.5 & 2.40 & 2.40\\
		32.0 & 2.20 & 2.20\\
		33.5 & 2.00 & 2.00\\
		35.5 & 1.80 & 1.80\\
		37.0 & 1.60 & 1.60\\
		38.5 & 1.40 & 1.40\\
		40.0 & 1.20 & 1.20\\
		41.0 & 1.00 & 1.00\\
		41.0 & 0.80 & 0.55\\
		41.0 & 0.60 & 0.05\\
		49.5 & 0.45 & 0.00\\		
		\bottomrule
	\end{tabular}
\end{table}

\begin{table}[htb]
	\sffamily \footnotesize \centering
	\caption{Pressió, \( \mathsfit P \) i temperatura, \( \mathsfit T \) per al procés isòcor per determinar la relació entre la pressió i la temperatura}
	\label{tab:PT}
	\begin{tabular}{SS}
		\toprule
		{$\mathsfit P$ (\unc{0.5 e5}{Pa})} & {$\mathsfit T$ (\unc{0.1}{\celsius})}  \\
		\midrule
		20.0 & 11.1\\
		21.5 & 14.1\\
		23.0 & 17.1\\
		24.5 & 19.8\\
		26.0 & 22.7\\
		27.5 & 25.6\\
		29.0 & 28.6\\
		30.5 & 30.5\\		
		\bottomrule
	\end{tabular}
\end{table}


\end{document}
