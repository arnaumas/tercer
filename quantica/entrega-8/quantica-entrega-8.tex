\documentclass[12pt]{article}

\usepackage[utf8]{inputenc}
\usepackage[T1]{fontenc}
\usepackage[catalan]{babel}
\usepackage{lmodern}
\usepackage{geometry}
\usepackage{hyperref}
\usepackage[dvipsnames]{xcolor}
\usepackage[bf,sf,small,pagestyles]{titlesec}
\usepackage{titling}
\usepackage[font={footnotesize, sf}, labelfont=bf]{caption} 
\usepackage{siunitx}
\usepackage{graphicx}
\usepackage{booktabs}
\usepackage{amsmath,amssymb}
\usepackage[catalan,sort]{cleveref}
\usepackage{enumitem}

\geometry{
	a4paper,
	right = 2.5cm,
	left = 2.5cm,
	bottom = 3cm,
	top = 3cm
}

\hypersetup{
	colorlinks,
	linkcolor = {red!50!blue},
	linktoc = page
}

\crefname{figure}{figura}{figures}
\crefname{table}{taula}{taules}
\numberwithin{table}{section}
\numberwithin{figure}{section}
\numberwithin{equation}{section}

\graphicspath{{./figs/}}

% Unitats
\sisetup{
	inter-unit-product = \ensuremath{ \cdot },
	allow-number-unit-breaks = true,
	detect-family = true,
	list-final-separator = { i },
	list-units = single
}

\renewcommand{\arraystretch}{1.4}

\renewcommand{\up}{\uparrow}
\newcommand{\down}{\downarrow}
\newcommand{\Z}{\mathbb{Z}}
\renewcommand{\vec}[1]{\mathbf{#1}}
\newcommand{\N}{\mathbb{N}}
\newcommand{\R}{\mathbb{R}}
\newcommand{\C}{\mathbb{C}}
\newcommand{\Ry}{\mathit{Ry}}
\let\Im\relax
\DeclareMathOperator{\Im}{Im}
\newcommand{\abs}[1]{\lvert #1 \rvert}
\newcommand{\ket}[1]{\left\vert {#1} \right\rangle}
\newcommand{\bra}[1]{\left\langle #1 \right\vert}
\newcommand{\braket}[2]{\langle {#1} \vert {#2} \rangle}
\newcommand{\parbreak}{
	\begin{center}
		--- $\ast$ ---
	\end{center} 
}
\makeatletter
\newcommand*{\defeq}{\mathrel{\rlap{%
    \raisebox{0.3ex}{$\m@th\cdot$}}%
  \raisebox{-0.3ex}{$\m@th\cdot$}}%
	=
}
\makeatother

\newpagestyle{pagina}{
	\headrule
	\sethead*{\sffamily {\bfseries Entrega 2:} Addició del moment angular. Estats de barions}{}{\theauthor}
	\footrule
	\setfoot*{}{}{\sffamily \thepage}
}
\renewpagestyle{plain}{
	\footrule
	\setfoot*{}{}{\sffamily \thepage}
}
\pagestyle{pagina}

\title{\sffamily {\bfseries Entrega 2:} Addició del moment angular. Estats de barions}
\author{\sffamily Arnau Mas}
\date{\sffamily 10 de maig de 2019}

\begin{document}
\maketitle

\section{Spin}
A continuació estudiem els possibles estats d'spin de tres quarks, tenint en compte que els quarks són partícules d'spin \( \tfrac{1}{2} \). Els estats més senzills són els de moment angular màxim, \( \ket{\up\up\up} \) i \( \ket{\down\down\down} \). Tenim
\begin{gather*}
	S_z \ket{\up\up\up} = \frac{\hbar}{2}\ket{\up\up\up} + \frac{\hbar}{2}\ket{\up\up\up} + \frac{\hbar}{2}\ket{\up\up\up} = \frac{3\hbar}{2}\ket{\up\up\up}, \\
	S_z \ket{\down\down\down} = - \frac{\hbar}{2}\ket{\down\down\down} - \frac{\hbar}{2}\ket{\down\down\down} - \frac{\hbar}{2}\ket{\down\down\down} = -\frac{3\hbar}{2}\ket{\down\down\down}. \\
\end{gather*}
Per tant aquests dos estats són propis de \( S_z \) amb valors propis \( \frac{3}{2}\hbar \) i \( -\frac{3}{2}\hbar \) respectivament. Fent servir que
\begin{equation} \label{eq:S2}
	S^2 = S_z^2 + \hbar S_z + S_-S_+ = S_z^2 - \hbar S_z + S_+S_-
\end{equation}
obtenim
\begin{align*}
	S^2 \ket{\up\up\up} & = S_z^2\ket{\up\up\up} + \hbar S_z\ket{\up\up\up} + S_-S_+\ket{\up\up\up} = \left(\frac{3\hbar}{2}\right)^2\ket{\up\up\up} + \frac{3\hbar^2}{2}\ket{\up\up\up} + 0 \\ 
											& = \tfrac{3}{2}\left(\tfrac{3}{2} + 1\right)\hbar^2\ket{\up\up\up}
\end{align*}
i
\begin{align*}
	S^2 \ket{\down\down\down} & = S_z^2\ket{\down\down\down} - \hbar S_z\ket{\down\down\down} + S_+S_-\ket{\down\down\down} = \left(\frac{3\hbar}{2}\right)^2\ket{\down\down\down} + \frac{3\hbar^2}{2}\ket{\down\down\down} + 0 \\ 
											& = \tfrac{3}{2}\left(\tfrac{3}{2} + 1\right)\hbar^2\ket{\down\down\down}.
\end{align*}
Per tant aquests dos estats tenen spin total \( \frac{3}{2}\hbar \). D'aquí deduïm que \( \ket{\up\up\up} = \ket{\tfrac{3}{2}, \tfrac{3}{2}} \) i que \( \ket{\down\down\down} = \ket{\tfrac{3}{2}, -\tfrac{3}{2}} \). 

\parbreak

Per a trobar els altres dos estats amb spin total \( \frac{3}{2}\hbar \) podem fer servir que
\begin{equation*}
	S_- \ket{\tfrac{3}{2}, \tfrac{3}{2}} = \hbar \sqrt{\tfrac{3}{2}\left(\tfrac{3}{2} + 1\right) - \tfrac{3}{2}\left(\tfrac{3}{2} - 1\right)}\ket{\tfrac{3}{2}, \tfrac{1}{2}} = \sqrt{3}\hbar \ket{\tfrac{3}{2}, \tfrac{1}{2}},
\end{equation*}
i
\begin{equation*}
	S_+ \ket{\tfrac{3}{2}, -\tfrac{3}{2}} = \hbar \sqrt{\tfrac{3}{2}\left(\tfrac{3}{2} + 1\right) + \tfrac{3}{2}\left(- \tfrac{3}{2} + 1\right)}\ket{\tfrac{3}{2}, -\tfrac{1}{2}} = \sqrt{3}\hbar \ket{\tfrac{3}{2}, -\tfrac{1}{2}}.
\end{equation*}
D'altra banda
\begin{equation*}
	S_- \ket{\tfrac{3}{2}, \tfrac{3}{2}} = S_-\ket{\up\up\up} = \hbar\left(\ket{\up\up\down} + \ket{\up\down\up} + \ket{\down\up\up}\right)
\end{equation*}
i
\begin{equation*}
	S_+ \ket{\tfrac{3}{2}, -\tfrac{3}{2}} = S_+\ket{\down\down\down} = \hbar\left(\ket{\down\down\up} + \ket{\down\up\down} + \ket{\up\down\down}\right).
\end{equation*}
Per tant
\begin{align*}
	\ket{\tfrac{3}{2}, \tfrac{1}{2}} & = \frac{\ket{\up\up\down} + \ket{\up\down\up} + \ket{\down\up\up}}{\sqrt{3}} \\
	\ket{\tfrac{3}{2}, -\tfrac{1}{2}} & = \frac{\ket{\down\down\up} + \ket{\down\up\down} + \ket{\up\down\down}}{\sqrt{3}}.
\end{align*}

\parbreak

\begin{table}[htb]
	\centering \sffamily \small
	\caption{Estats propis d'spin de tres quarks}
	\label{tab:estats d'spin}
	\begin{tabular}{@{}llll@{}}
		\toprule
		\( S^2 \) & \( S_z \) & Estat \\
		\midrule	
		\( \tfrac{3}{2}\hbar \) & \( \tfrac{3}{2}\hbar \) & \( \ket{S_{123}^1} = \ket{\up\up\up} \) \\
		\( \tfrac{3}{2}\hbar \) & \( \tfrac{1}{2}\hbar \) & \( \ket{S_{123}^2} = \frac{\ket{\up\up\down} + \ket{\up\down\up} + \ket{\down\up\up}}{\sqrt{3}} \) \\
		\( \tfrac{3}{2}\hbar \) & \( -\tfrac{1}{2}\hbar \) & \( \ket{S_{123}^3} = \frac{\ket{\down\down\up} + \ket{\down\up\down} + \ket{\up\down\down}}{\sqrt{3}} \) \\
		\( \tfrac{3}{2}\hbar \) & \( -\tfrac{3}{2}\hbar \) & \( \ket{S_{123}^4} = \ket{\down\down\down} \) \\
		\( \tfrac{1}{2}\hbar \) & \( \tfrac{1}{2}\hbar \) & \( \ket{A_{12}^1} = \frac{\ket{\up\down\up} - \ket{\down\up\up}}{\sqrt{2}} \) \\
		\( \tfrac{1}{2}\hbar \) & \( \tfrac{1}{2}\hbar \) & \( \ket{S_{12}^1} = \frac{2\ket{\up\up\down} - (\ket{\up\down\up} + \ket{\down\up\up})}{\sqrt{6}} \) \\
		\( \tfrac{1}{2}\hbar \) & \( -\tfrac{1}{2}\hbar \) & \( \ket{A_{12}^2} = \frac{\ket{\down\up\down} - \ket{\up\down\down}}{\sqrt{2}} \) \\
		\( \tfrac{1}{2}\hbar \) & \( -\tfrac{1}{2}\hbar \) & \( \ket{S_{12}^2} = \frac{2\ket{\down\down\up} - (\ket{\down\up\down} + \ket{\up\down\down})}{\sqrt{6}} \) \\
		\bottomrule
	\end{tabular}
\end{table}

És clar que tots els vectors propis de \( S_z \) de valor propi \( \frac{1}{2}\hbar \) són combinacions lineals de \( \ket{\up\up\down}, \ket{\up\down\up} \text{ i } \ket{\down\up\up} \), que generen un subespai de dimensió 3. En aquest subespai hi ha \( \ket{\frac{3}{2}, \frac{1}{2}} \). Aquest estat és l'únic que pot tenir aquests valors propis ja que l'obtenim aplicant \( S_- \) a l'estat \( \ket{\up\up\up} \), que és l'únic estat propi de \( S_z \) amb valor propi \( \frac{3}{2}\hbar \). Així doncs, l'ortogonal de \( \ket{\frac{3}{2}, \frac{1}{2}} \)  és subespai propi de \( S^2 \) amb valor propi \( \frac{1}{2}\left(\frac{1}{2} + 1\right)\hbar \), de manera que tenim una degeneració pel que fa a l'estat \( \ket{\frac{1}{2}, \frac{1}{2}} \). Per a trencar aquesta degneració imposarem que els estats han de ser simètrics o antisimètrics respecte l'intercanvi dels quarks 1 i 2. Considerem un estat qualsevol \( \alpha\ket{\up\up\down} + \beta\ket{\up\down\up} + \gamma\ket{\down\up\up} \). L'ortogonalitat equival a \( \alpha + \beta + \gamma = 0 \). Si requerim que sigui antisimètric aleshores \( \alpha = 0 \) i \( \beta = -\gamma \), de manera que un cop normalitzat l'estat és
\begin{equation*}
	\frac{\ket{\up\down\up} - \ket{\down\up\up}}{\sqrt{2}}.
\end{equation*}
Si imposem que sigui simètric aleshores \( \beta = \gamma \), i per ortogonalitat, \( 2\beta = -\alpha \). Un cop normalitzat, l'estat és
\begin{equation*}
	\frac{2\ket{\up\up\down} - (\ket{\up\down\up} + \ket{\down\up\up})}{\sqrt{6}}.
\end{equation*}

Pel que fa als estats amb \( s = \frac{1}{2} \) i \( m_s = -\frac{1}{2} \), podem aplicar exactament els mateixos arguments simplement intercanviant \( \up \) per \( \down \) i viceversa. A la \cref{tab:estats d'spin} es mostren els diversos estats d'spin pels tres quarks.


\section{Sabor}
Si tenim quarks només de dos sabors, \emph{up} i \emph{down}, aleshores els estats \( \ket{u} \) i \( \ket{d} \) es comporten de la mateixa manera que els estats \( \ket{\up} \) i \( \ket{\down} \), és a dir, són propis d'uns operadors, anomenats d'isospin, que satisfan les relacions de commutació del moment angular d'una partícula d'spin \( \frac{1}{2} \). Per tant els càlculs que hauriem de fer per a trobar els estats de sabor són els mateixos que els de la secció anterior. A la \cref{tab:estats de sabor} es mostren tots els possibles estats de sabor amb els corresponents valors d'isospin co\l.lectiu. 

\begin{table}[htb]
	\centering \sffamily \small
	\caption{Estats propis d'isospin per a tres quarks, considerant només els sabors \emph{up}, \( u \), i \emph{down}, \( d \)}
	\label{tab:estats de sabor}
	\begin{tabular}{@{}llll@{}}
		\toprule
		\( I^2 \) & \( I_z \) & Estat \\
		\midrule	
		\( \tfrac{3}{2} \) & \( \tfrac{3}{2} \) & \( \ket{S_{123}^1} = \ket{uuu} \) \\
		\( \tfrac{3}{2} \) & \( \tfrac{1}{2} \) & \( \ket{S_{123}^2} = \frac{\ket{uud} + \ket{udu} + \ket{duu}}{\sqrt{3}} \) \\
		\( \tfrac{3}{2} \) & \( -\tfrac{1}{2} \) & \( \ket{S_{123}^3} = \frac{\ket{ddu} + \ket{dud} + \ket{udd}}{\sqrt{3}} \) \\
		\( \tfrac{3}{2} \) & \( -\tfrac{3}{2} \) & \( \ket{S_{123}^4} = \ket{ddd} \) \\
		\( \tfrac{1}{2} \) & \( \tfrac{1}{2} \) & \( \ket{A_{12}^1} = \frac{\ket{udu} - \ket{duu}}{\sqrt{2}} \) \\
		\( \tfrac{1}{2} \) & \( \tfrac{1}{2} \) & \( \ket{S_{12}^1} = \frac{2\ket{uud} - (\ket{udu} + \ket{duu})}{\sqrt{6}} \) \\
		\( \tfrac{1}{2} \) & \( -\tfrac{1}{2} \) & \( \ket{A_{12}^2} = \frac{\ket{dud} - \ket{udd}}{\sqrt{2}} \) \\
		\( \tfrac{1}{2} \) & \( -\tfrac{1}{2} \) & \( \ket{S_{12}^2} = \frac{2\ket{ddu} - (\ket{dud} + \ket{udd})}{\sqrt{6}} \) \\
		\bottomrule
	\end{tabular}
\end{table}

\section{Color}
Els possibles estats ortogonals de color neutre de tres quarks són \( \ket{rgb} \), \( \ket{rbg} \), \( \ket{gbr} \), \( \ket{grb} \), \( \ket{brg} \) i \( \ket{bgr} \). Per tant, un estat de color neutre serà de la forma
\begin{equation*}
	\ket{\Psi} = \alpha_1 \ket{rgb} + \alpha_2 \ket{rbg} + \beta_1 \ket{gbr} + \beta_2 \ket{grb} + \gamma_1 \ket{brg} + \gamma_2 \ket{bgr}.
\end{equation*}

A la natura, però, només s'observen estats de color completament antisimètrics. Hem d'imposar \( P_{12}\ket{\Psi} = P_{23}\ket{\Psi} = P_{31}\ket{\Psi} = -\ket{\Psi} \). Quan requerim \( P_{12}\ket{\Psi} = -\ket{\Psi} \) trobem \( \alpha_1 = -\alpha_2 \), \( \beta_1 = -\beta_2 \) i \( \gamma_1 = -\gamma_2 \). Per tant podem escriure
\begin{equation*}
	\ket{\Psi} = \alpha(\ket{rgb} - \ket{rbg}) + \beta(\ket{gbr} - \ket{grb}) + \gamma(\ket{brg} - \ket{bgr}).
\end{equation*}
Aleshores
\begin{equation*}
	P_{12}\ket{\Psi} = \alpha(\ket{grb} - \ket{brg}) + \beta(\ket{bgr} - \ket{rgb}) + \gamma(\ket{rbg} - \ket{gbr}),
\end{equation*}
de manera que quan imposem \( P_{12}\ket{\Psi} = -\ket{\Psi} \) obtenim \( \alpha = \beta = \gamma \). I aleshores la condició \( P_{31}\ket{\Psi} = -\ket{\Psi} \) es compleix. Podem determinar l'últim paràmetre que queda per normalització, i per tant el singlet de color és 
\begin{equation*}
	\ket{\Psi} = \frac{\ket{rgb} - \ket{rbg} + \ket{gbr} - \ket{grb} + \ket{brg} - \ket{bgr}}{\sqrt{6}}.
\end{equation*}

\section{Estats bariònics}
Els estats propis de tres quarks seran productes d'un dels possibles estats d'spin, de sabor i de color. El factor de color és únic, el singlet, que és antisimètric. Per tant el producte de l'estat d'spin i sabor ha de ser simètric. Hi ha 16 possibilitats òbvies, que són els estats de la forma \( \ket{S_{123}^n}_S\ket{S_{123}^m}_I \), on \( n \) i \( m \) varient de 1 a 4, és a dir, productes d'estats amb spin total \( \frac{3}{2}\hbar \) i isospin total \( \frac{3}{2} \), segons les \cref{tab:estats d'spin,tab:estats de sabor}.

\parbreak

Per a determinar els altres possibles estats és útil saber com actuen les transposicions \( P_{12} \), \( P_{23} \) i \( P_{31} \) sobre els estats d'spin i sabor que hem trobat. Farem els cálculs sobre els estats d'spin, però és clar que els estats de sabor satisfaran les mateixes relacions. És clar que sobre els 4 estats simètrics actuen com la identitat. També és clar que \( P_{12}\ket{S_{12}^i} = \ket{S_{12}^i} \) i \( P_{12}\ket{A_{12}^i} = -\ket{A_{12}^i} \), per construcció.

Tenim
\begin{equation*}
	P_{23}\ket{A_{12}^1} = \frac{\ket{\up\up\down} - \ket{\down\up\up}}{\sqrt{2}}.
\end{equation*}
Per saber com s'escriu aquest estat en termes de la base que hem trobat podem calcular el seu producte amb la resta dels estats. Aquest estat té \( m_s = \frac{3}{2}\hbar \) i ortogonal a \( \ket{S_{123}^2} \), de manera que els únics productes que no són nuls són
\begin{equation*}
	\bra{A_{12}^1}P_{23}\ket{A_{12}^1} = \frac{1}{2} 
\end{equation*}
i
\begin{equation*}
	\bra{S_{12}^1}P_{23}\ket{A_{12}^1} = \frac{1}{\sqrt{3}} - \frac{1}{2\sqrt{3}} = \frac{\sqrt{3}}{2}.
\end{equation*}
És a dir
\begin{equation}\label{eq:estat P23A12}
	P_{23}\ket{A_{12}^i} = \frac{\ket{A_{12}^1} + \sqrt{3}\ket{S_{12}^1}}{2}.
\end{equation}
Similarment tenim
\begin{equation*}
	P_{31}\ket{A_{12}^1} = \frac{\ket{\up\down\up} - \ket{\up\up\down}}{2}.
\end{equation*}
Els productes són
\begin{equation*}
	\bra{A_{12}^1}P_{31}\ket{A_{12}^1} = \frac{1}{2}
\end{equation*}
i
\begin{equation*}
	\bra{S_{12}^1}P_{31}\ket{A_{12}^1} = -\frac{1}{\sqrt{3}} - \frac{1}{2\sqrt{3}} = -\frac{\sqrt{3}}{2}.
\end{equation*}
Per tant
\begin{equation} \label{eq:estat P31A12}
	P_{31}\ket{A_{12}^1} = \frac{\ket{A_{12}^1} - \sqrt{3}\ket{S_{12}^1}}{2}.
\end{equation}

Fem els mateixos càlculs per \( P_{23}\ket{S_{12}^1} \) i \( P_{31}\ket{S_{12}^1} \). Com que
\begin{equation*}
	P_{23}\ket{S_{12}^1} = \frac{2\ket{\up\down\up} - (\ket{\up\up\down} + \ket{\down\up\up})}{\sqrt{6}}
\end{equation*}
aleshores
\begin{equation*}
	\bra{A_{12}^1}P_{23}\ket{S_{12}^1} = \frac{1}{\sqrt{3}} + \frac{1}{2\sqrt{3}} = \frac{\sqrt{3}}{2}
\end{equation*}
i
\begin{equation*}
	\bra{S_{12}^1}P_{23}\ket{S_{12}^1} =  \frac{-2 - 2 + 1}{6} = -\frac{1}{2},
\end{equation*}
de manera que
\begin{equation} \label{eq:estat P23S12}
	P_{23}\ket{S_{12}^1} = \frac{\sqrt{3}\ket{A_{12}^1} - \ket{S_{12}^1}}{2}.
\end{equation}

L'últim càlcul que cal fer és per \( P_{31}\ket{S_{12}^1} \). Tenim
\begin{equation*}
	P_{31}\ket{S_{12}^1} = \frac{2\ket{\down\up\up} - (\ket{\up\down\up} + \ket{\up\up\down})}{\sqrt{6}}.
\end{equation*}
Aleshores
\begin{align*}
	\bra{A_{12}^1}P_{31}\ket{S_{12}^1} &= -\frac{\sqrt{3}}{2} \\
	\bra{S_{12}^1}P_{31}\ket{S_{12}^1} &= - \frac{1}{2}
\end{align*}
i per tant
\begin{equation}\label{eq:estat P31S12}
	P_{31}\ket{S_{12}^1} = - \frac{\sqrt{3}\ket{A_{12}^1} + \ket{S_{12}^1}}{2}.
\end{equation}

És clar que tots els càlculs que hem fet són vàlids si intercanviem \( \up \) per \( \down \), de manera que les \cref{eq:estat P23A12,eq:estat P31A12,eq:estat P23S12,eq:estat P31S12} continuen sent certes si canviem els superíndexs 1 per 2.

\parbreak

Per a trobar els estats que tenen spin total \( \frac{1}{2}\hbar \) i isospin total \( \frac{1}{2} \) considerem una combinació lineal dels estats propis d'spin i isospin amb aquests valors i la simetritzem. Concretament, posem
\begin{equation*}
	\ket{\Psi} = \alpha \ket{S_{12}^n}_S\ket{S_{12}^m}_I + \beta \ket{S_{12}^n}_S\ket{A_{12}^m}_I + \gamma \ket{A_{12}^n}_S\ket{S_{12}^m}_I + \delta \ket{A_{12}^n}_S\ket{A_{12}^m}_I
\end{equation*}
i aleshores requerim que \( P_{12}\ket{\Psi} = P_{23}\ket{\Psi} = P_{31}\ket{\Psi} = \ket{\Psi} \). Si \( P_{12}\ket{\Psi} = \ket{\Psi} \) aleshores \( \gamma = \delta = 0 \). També tenim, fent servir els resultats que hem obtingut prèviament,
\begin{align*}
	P_{23} \ket{\Psi} &= \alpha \left(\frac{\sqrt{3}\ket{A_{12}^n} - \ket{S_{12}^n}}{2}\right)\left(\frac{\sqrt{3}\ket{A_{12}^m} - \ket{S_{12}^m}}{2}\right) + \delta\left(\frac{\ket{A_{12}^n} + \sqrt{3}\ket{S_{12}^n}}{2}\right)\left(\frac{\ket{A_{12}^m} + \sqrt{3}\ket{S_{12}^m}}{2}\right) \\
										&= \frac{3\alpha + \delta}{2}\ket{A_{12}^n}\ket{A_{12}^m} + \frac{\alpha + 3\delta}{2}\ket{S_{12}^n}\ket{S_{12}^m} + \frac{\sqrt{3}(\alpha - \delta)}{4}(\ket{A_{12}^n}\ket{S_{12}^m} + \ket{S_{12}^n}\ket{A_{12}^m}).
\end{align*}
Si requerim que \( P_{23}\ket{\Psi} = \ket{\Psi} \) aleshores deduïm que \( \alpha = \delta \). Finalment
\begin{align*}
	P_{31} \ket{\Psi} &= \alpha \left(\frac{\sqrt{3}\ket{A_{12}^n} + \ket{S_{12}^n}}{2}\right)\left(\frac{\sqrt{3}\ket{A_{12}^m} + \ket{S_{12}^m}}{2}\right) + \alpha\left(\frac{\ket{A_{12}^n} - \sqrt{3}\ket{S_{12}^n}}{2}\right)\left(\frac{\ket{A_{12}^m} - \sqrt{3}\ket{S_{12}^m}}{2}\right) \\
										&= \alpha\ket{A_{12}^n}\ket{A_{12}^m} + \alpha\ket{S_{12}^n}\ket{S_{12}^m} = \ket{\Psi}, 
\end{align*}
i per tant se satisfà l'últim requeriment. Per normalització determinem que \( \alpha = \frac{1}{\sqrt{2}} \).

\parbreak

A continuació resumim els estats de tres quarks possibles. Només donem la part d'spin i sabor, ja que la part de color és sempre el singlet. Denotarem l'spin total per \( j_s \), l'spin en la direcció vertical per \( m_s \),  l'isospin total per \( j_I \) i el tercer component de l'isospin per \( m_I \).

Per a \( j_I = \frac{3}{2} \) tenim els quatres barions \( \Delta \), classificats en funció de \( m_I \). Aquests estats corresponen als estats de la forma \( \ket{S_{123}^n}_S \ket{S_{123}^m}_I \), per tant la part d'spin ha de tenir \( j_s = \frac{3}{2}\hbar \). 
\begin{equation*}
	\begin{aligned}
		m_I & = \frac{3}{2}, \\
    m_I & = \frac{1}{2}, \\
    m_I & = -\frac{1}{2}, \\
    m_I & = -\frac{3}{2},
	\end{aligned}
	\qquad
	\begin{aligned}
	 	\ket{\Delta^{++}} & = \ket{S_{123}^n}_S\ket{S_{123}^1}_I = \ket{S_{123}^n}_S \ket{uuu} \\
	 	\ket{\Delta^{+}} & = \ket{S_{123}^n}_S\ket{S_{123}^2}_I = \ket{S_{123}^n} \frac{\ket{uud} + \ket{udu} + \ket{duu}}{\sqrt{3}} \\
	 	\ket{\Delta^{0}} & = \ket{S_{123}^n}_S\ket{S_{123}^3}_I = \ket{S_{123}^n} \frac{\ket{ddu} + \ket{dud} + \ket{udd}}{\sqrt{3}} \\
	 	\ket{\Delta^{0}} & = \ket{S_{123}^n}_S\ket{S_{123}^4}_I = \ket{S_{123}^n}_S \ket{ddd}.
	\end{aligned}
\end{equation*}
Per a \( n \) entre 1 i 4 obtenim estats amb \( m_s \) entre \( -\frac{3}{2}\hbar \) i \( \frac{3}{2}\hbar \).

Per a \( j_I = \frac{1}{2} \) tenim el protó i el neutró, que es corresponen amb els estats de la forma \( \frac{1}{\sqrt{2}}(\ket{S_{12}^n}_S\ket{S_{12}^m}_I + \ket{A_{12}^n}_S\ket{A_{12}^m}_I) \). Veiem que \( j_S = \frac{1}{2}\hbar \), per tant els protons i els neutrons són partícules d'spin \( \frac{1}{2} \). El protó apareix amb \( m_I = \frac{1}{2} \):
\begin{align*}
	\ket{p^+} & = \frac{\ket{S_{12}^n}_S\ket{S_{12}^1}_I + \ket{A_{12}^n}_S\ket{A_{12}^1}_I}{\sqrt{2}} \\
						& = \ket{S_{12}^n}_S \frac{2\ket{uud} - (\ket{udu} + \ket{duu})}{2\sqrt{3}} + \ket{A_{12}^n}_S \frac{\ket{udu} - \ket{duu}}{2}. 
\end{align*}
Amb \( n = 1 \) tenim l'estat amb spin \( \frac{1}{2}\hbar \) i amb \( n = 2 \) l'estat amb spin \( -\frac{1}{2}\hbar \).
I el neutró té \( m_I = -\frac{1}{2} \):
\begin{align*}
	\ket{n^0} & = \frac{\ket{S_{12}^n}_S\ket{S_{12}^2}_I + \ket{A_{12}^n}_S\ket{A_{12}^2}_I}{\sqrt{2}} \\
						& = \ket{S_{12}^n}_S \frac{2\ket{ddu} - (\ket{dud} + \ket{udd})}{2\sqrt{3}} + \ket{A_{12}^n}_S \frac{\ket{dud} - \ket{udd}}{2}. 
\end{align*}
Com abans, amb \( n = 1 \) tenim l'estat d'spin \( \frac{1}{2}\hbar \) i amb \( n = 2 \) l'estat d'spin \( -\frac{1}{2}\hbar \).

\end{document}
