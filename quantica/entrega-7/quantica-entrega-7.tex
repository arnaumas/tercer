\documentclass[12pt]{article}

\usepackage[utf8]{inputenc}
\usepackage[T1]{fontenc}
\usepackage[catalan]{babel}
\usepackage{lmodern}
\usepackage{geometry}
\usepackage{hyperref}
\usepackage[dvipsnames]{xcolor}
\usepackage[bf,sf,small,pagestyles]{titlesec}
\usepackage{titling}
\usepackage[font={footnotesize, sf}, labelfont=bf]{caption} 
\usepackage{siunitx}
\usepackage{graphicx}
\usepackage{booktabs}
\usepackage{amsmath,amssymb}
\usepackage[catalan,sort]{cleveref}
\usepackage{enumitem}

\geometry{
	a4paper,
	right = 2.5cm,
	left = 2.5cm,
	bottom = 3cm,
	top = 3cm
}

\hypersetup{
	colorlinks,
	linkcolor = {red!50!blue},
	linktoc = page
}

\crefname{figure}{figura}{figures}
\crefname{table}{taula}{taules}
\numberwithin{table}{section}
\numberwithin{figure}{section}
\numberwithin{equation}{section}

\graphicspath{{./figs/}}

% Unitats
\sisetup{
	inter-unit-product = \ensuremath{ \cdot },
	allow-number-unit-breaks = true,
	detect-family = true,
	list-final-separator = { i },
	list-units = single
}

\newcommand{\Z}{\mathbb{Z}}
\renewcommand{\vec}[1]{\mathbf{#1}}
\newcommand{\N}{\mathbb{N}}
\newcommand{\R}{\mathbb{R}}
\newcommand{\C}{\mathbb{C}}
\newcommand{\Ry}{\mathit{Ry}}
\let\Im\relax
\DeclareMathOperator{\Im}{Im}
\newcommand{\abs}[1]{\lvert #1 \rvert}
\newcommand{\ket}[1]{\vert {#1} \rangle}
\newcommand{\bra}[1]{\langle #1 \vert}
\newcommand{\braket}[2]{\langle {#1} \vert {#2} \rangle}
\newcommand{\parbreak}{
	\begin{center}
		--- $\ast$ ---
	\end{center} 
}
\makeatletter
\newcommand*{\defeq}{\mathrel{\rlap{%
    \raisebox{0.3ex}{$\m@th\cdot$}}%
  \raisebox{-0.3ex}{$\m@th\cdot$}}%
	=
}
\makeatother

\newpagestyle{pagina}{
	\headrule
	\sethead*{\sffamily {\bfseries Entrega 1:} Estats coherents}{}{\theauthor}
	\footrule
	\setfoot*{}{}{\sffamily \thepage}
}
\renewpagestyle{plain}{
	\footrule
	\setfoot*{}{}{\sffamily \thepage}
}
\pagestyle{pagina}

\title{\sffamily {\bfseries Entrega 1:} Estats coherents}
\author{\sffamily Arnau Mas}
\date{\sffamily 18 de març de 2019}

\begin{document}
\maketitle

Considerem l'estat cat
\begin{equation*}
	\ket{\psi} = A \cos{\alpha a^{\dagger}} \ket{0}
\end{equation*}
on \( A \) és una constant de normalització i \( \alpha \) una constant positiva.

\begin{enumerate}[label=(\alph*), font=\bfseries \sffamily, wide, labelwidth=!, labelindent=0pt]
	\item Tenim que
		\begin{equation*}
		\cos{\alpha a^\dagger} = \frac{e^{i\alpha a^\dagger} + e^{-i \alpha a^\dagger}}{2}
		\end{equation*}
	i per tant
	\begin{equation*}
		\ket{\psi} = \frac{A}{2} \left(e^{i\alpha a^\dagger} \ket{0} + e^{-i \alpha a^\dagger} \ket{0} \right) = \frac{Ae^{\frac{\alpha^2}{2}}}{2}\left(\ket{i\alpha} + \ket{-i\alpha}\right).
	\end{equation*}
Aleshores
\begin{align*}
	\langle a \rangle_\psi & = \bra{\psi} a \ket{\psi} = \frac{A^2e^{\alpha^2}}{4} \left(\bra{i\alpha} a \ket{i\alpha} + \bra{-i\alpha} a \ket{-i\alpha} + \bra{i\alpha} a \ket{-i \alpha} + \bra{-i \alpha} a \ket{i \alpha} \right) \\
												 & = \frac{A^2e^{\alpha^2}}{4} \left(i\alpha - i\alpha - i\alpha \braket{i\alpha}{-i\alpha} + i\alpha \braket{-i\alpha}{i\alpha} \right).
\end{align*}
En general, per \( \alpha, \beta \in \C \) es té
\begin{equation*}
	\braket{\alpha}{\beta} = e^{-i \Im{\left(\alpha \beta^\ast\right)}} e^{- \frac{\abs{\alpha - \beta}^2}{2}}.
\end{equation*}
En el cas que estem considerant el primer factor de fet és 1 ja que \( i\alpha(-i \alpha)^\ast \) de fet és real. Així
\begin{equation*}
	\braket{i\alpha}{-i\alpha} = \braket{-i\alpha}{i\alpha} = e^{-2\alpha^2}.
\end{equation*}
I per tant \( \langle a \rangle_\psi = \langle a^\dagger \rangle_\psi^{\ast} = 0 \). Per tant, com que \( x \) i \( p \) són combinacions lineals de \( a \) i \( a^\dagger \), deduïm que \( \langle x \rangle_\psi = \langle p \rangle_\psi = 0 \).

\item Tenim
	\begin{equation*}
		a^2 \ket{\psi} = \frac{Ae^{\frac{\alpha^2}{2}}}{2}(a^2 \ket{i\alpha} + a^2 \ket{-i\alpha}) = \frac{Ae^{\frac{\alpha^2}{2}}}{2}(-\alpha^2 \ket{i\alpha} -\alpha^2 \ket{-i\alpha}) = -\alpha^2 \ket{\psi},
	\end{equation*}
	és a dir, \( \ket{\psi} \) és propi de \( a^2 \) amb valor propi \( -\alpha^2 \). 

\item \( \ket{\psi} \) és propi de \( \cos{\alpha a} \) amb valor propi \( \cosh{\alpha^2} \). En efecte
	\begin{align*}
		\cos{(\alpha a)} \ket{\psi} & = \frac{Ae^{\frac{\alpha^2}{2}}}{2}(\cos{(\alpha a)} \ket{i\alpha} + \cos{(\alpha a)} \ket{-i\alpha}) = \frac{Ae^{\frac{\alpha^2}{2}}}{2}(\cos{(i\alpha^2)} \ket{i\alpha} + \cos{(-i\alpha^2)} \ket{-i\alpha}) \\
																& = \frac{Ae^{\frac{\alpha^2}{2}}}{2}(\cos{(i\alpha^2)} \ket{i\alpha} + \cos{(i\alpha^2)} \ket{-i\alpha}) = \cos{(i\alpha^2)} \ket{\psi} = \cosh{(\alpha^2)} \ket{\psi}.
	\end{align*}
	
\item Tenim, fent servir el resultat anterior,
	\begin{equation*}
		\braket{\psi}{\psi} = \bra{\psi} A \cos{(\alpha a^{\dagger})} \ket{0} = A \cosh{\alpha^2} \braket{\psi}{0}.
	\end{equation*}
D'altra banda
\begin{equation*}
	\braket{\psi}{0} = \frac{A^\ast e^{\frac{\alpha^2}{2}}}{2}\left(\braket{i \alpha}{0} + \braket{-i \alpha}{0} \right) = \frac{A^\ast e^{\frac{\alpha^2}{2}}}{2}\left(e^{-\frac{\alpha^2}{2}} + e^{-\frac{\alpha^2}{2}} \right) = A^\ast. 
\end{equation*}
Així doncs, si \( \braket{\psi}{\psi} = 1 \) ha de ser \( AA^\ast \cosh{\alpha^2} = \abs{A}^2 \cosh{\alpha^2} = 1 \). Com que l'estat \( \ket{\psi} \) està determinat mòdul una fase, podem triar la seva constant de normalització real i positiva de manera que podem escriure'l com 
\begin{equation*}
	\ket{\psi} = \frac{e^{\frac{\alpha^2}{2}}}{2 \sqrt{\cosh{\alpha^2}}} \left(\ket{i \alpha} + \ket{-i \alpha}\right) = \frac{\ket{i \alpha} + \ket{-i \alpha}}{\sqrt{2\left(1 + e^{-2\alpha^2}\right)}} . 
\end{equation*}

\item En aquest cas
	\begin{equation*}
		\ket{\phi} = B \sin{\alpha a^\dagger} \ket{0} = -\frac{iB e^{\frac{\alpha^2}{2}}}{2}(\ket{i \alpha} - \ket{-i \alpha}). 
	\end{equation*}
	Calculem \( \braket{\phi}{\phi} \):
	\begin{align*}
		\braket{\phi}{\phi} & = \frac{BB^\ast e^{\alpha^2}}{4}(\braket{i\alpha}{i\alpha} + \braket{-i\alpha}{-i\alpha} - \braket{-i\alpha}{i\alpha} - \braket{i\alpha}{-i\alpha}) \\
												& = \frac{\abs{B}^2 e^{\alpha^2}}{4}(2 - \braket{-i\alpha}{i\alpha} - \braket{i\alpha}{-i\alpha}).
	\end{align*}
Aleshores
\begin{equation*}
	\braket{i\alpha}{-i\alpha} = e^{-i \Im{(i\alpha)(-i\alpha)^\ast}} e^{-\frac{\abs{i\alpha + i\alpha}^2}{2}} = e^{-2\alpha^2},
\end{equation*}
i per tant
\begin{equation*}
	\braket{\phi}{\phi} = \frac{\abs{B}^2 e^{\alpha^2}}{4}(2 - 2e^{-2\alpha^2}) = \abs{B}^2 \frac{e^{\alpha^2} - e^{-\alpha^2}}{2} = \abs{B}^2 \sinh{\alpha^2}.
\end{equation*}
Aleshores, si \( \braket{\phi}{\phi} = 1 \), ha de ser \( \abs{B}^2 \sinh{\alpha^2} = 1 \). Com abans, tenim llibertat de triar la fase de \( B \). Si fem \( B = \frac{i}{\sqrt{\sinh{\alpha^2}}} \) podem escriure
\begin{equation*}
	\ket{\phi} = \frac{e^{\frac{\alpha^2}{2}}}{2\sqrt{\sinh{\alpha^2}}}(\ket{i \alpha} - \ket{-i \alpha}) = \frac{\ket{i \alpha} - \ket{-i \alpha}}{\sqrt{2(1 - e^{-2\alpha^2})}}.
\end{equation*}

\item Calculem \( \langle H \rangle_\psi \) on \( H = \hbar \omega a^\dagger a \). Tenim
	\begin{align*}
		\langle a^\dagger a \rangle_\psi & = \bra{\psi} a^\dagger a	\ket{\psi} = \frac{e^{\alpha^2}}{4\cosh{\alpha^2}} (\bra{i\alpha} + \bra{-i\alpha})a^\dagger a(\ket{i\alpha} + \ket{-i\alpha}) \\
																		 & = \frac{e^{\alpha^2}}{4\cosh{\alpha^2}} (-i\alpha\bra{i\alpha} + i\alpha\bra{-i\alpha})(i\alpha\ket{i\alpha} - i\alpha\ket{-i\alpha}) \\
																		 & = \frac{e^{\alpha^2}}{4\cosh{\alpha^2}} (\bra{i\alpha} - \bra{-i\alpha})(\ket{i\alpha} - \ket{-i\alpha}) \\
																		 & = \frac{e^{\alpha^2}}{4\cosh{\alpha^2}} \frac{4}{\abs{B}^2 e^{\alpha^2}} = \alpha^2 \frac{\sinh{\alpha^2}}{\cosh{\alpha^2}},
	\end{align*}
on hem fet servir l'apartat anterior. Per tant 
\begin{equation*}
\langle H \rangle_\psi = \hbar\omega\alpha \tanh{\alpha^2}.
\end{equation*}

\item Aquest hamiltonià és un osci\l.lador harmònic desplaçat \( \frac{1}{2} \), i per tant té per estats propis els estats de la base de Fock amb energies \( E_n = \hbar \omega n \). Aleshores, per un estat coherent qualsevol
	\begin{equation*}
		\ket{\alpha(t)} = \sum_{n = 0}^{\infty} \frac{\alpha^n}{\sqrt{n!}} e^{-\frac{iE_n t}{\hbar}}\ket{n} = \sum_{n = 0}^\infty \frac{\alpha^n}{\sqrt{n!}} e^{-i\omega t} \ket{n} = \sum_{n = 0}^\infty \frac{(\alpha e^{-i\omega t})^n}{\sqrt{n!}} = \ket{\alpha e^{-i\omega t}}. 
	\end{equation*}
Per tant l'evolució de l'estat \( \ket{\psi} \) sota el hamiltonià \( H \) és
\begin{equation*}
	\ket{\psi(t)} = \frac{\ket{i\alpha e^{-i\omega t}} + \ket{- i\alpha e^{-i \omega t}}}{\sqrt{2(1 - e^{-2\alpha^2})}}.
\end{equation*}
Evidentment, quan \( t = \frac{2 \pi}{\omega} \) se satisfà \( \ket{\psi(t)} = \ket{\psi} \). Però per \( t = \frac{\pi}{\omega} \) això també passa:
\begin{equation*}
	\ket{\psi(\tfrac{\pi}{\omega})} = \frac{\ket{i\alpha e^{-i\pi}} + \ket{- i\alpha e^{-i \pi}}}{\sqrt{2(1 - e^{-2\alpha^2})}} = \frac{\ket{-i\alpha} + \ket{i\alpha}}{\sqrt{2(1 - e^{-2\alpha^2})}} = \ket{\psi}.
\end{equation*}


\end{enumerate}
\end{document}
