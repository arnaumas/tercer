\documentclass[12pt]{article}

\usepackage[utf8]{inputenc}
\usepackage[T1]{fontenc}
\usepackage[catalan]{babel}
\usepackage{lmodern}
\usepackage{geometry}
\usepackage{hyperref}
\usepackage[dvipsnames]{xcolor}
\usepackage[bf,sf,small,pagestyles]{titlesec}
\usepackage[font={footnotesize, sf}, labelfont=bf]{caption} 
\usepackage{siunitx}
\usepackage{graphicx}
\usepackage{booktabs}
\usepackage{amsmath,amssymb}
\usepackage[catalan,sort]{cleveref}
\usepackage{enumitem}

\geometry{
	a4paper,
	right = 2.5cm,
	left = 2.5cm,
	bottom = 3cm,
	top = 3cm
}

\hypersetup{
	colorlinks,
	linkcolor = {red!50!blue},
	linktoc = page
}

\crefname{figure}{figura}{figures}
\crefname{table}{taula}{taules}
\numberwithin{table}{section}
\numberwithin{figure}{section}
\numberwithin{equation}{section}

\graphicspath{{./figs/}}

% Unitats
\sisetup{
	inter-unit-product = \ensuremath{ \cdot },
	allow-number-unit-breaks = true,
	detect-family = true,
	list-final-separator = { i },
	list-units = single
}

\newcommand{\Z}{\mathbb{Z}}
\newcommand{\N}{\mathbb{N}}
\newcommand{\R}{\mathbb{R}}
\newcommand{\Ry}{\mathit{Ry}}
\newcommand{\abs}[1]{\left\lvert #1 \right\rvert}
\newcommand{\ket}[1]{\left\lvert #1 \right\rangle}
\newcommand{\bra}[1]{\left\langle #1 \right\rvert}
\newcommand{\inn}[2]{\left\langle #1 , #2 \right\rangle}
\newcommand{\braket}[2]{\left\langle #1 \mid #2 \right\rangle}
\newcommand{\parbreak}{
	\begin{center}
		--- $\ast$ ---
	\end{center} 
}
\makeatletter
\newcommand*{\defeq}{\mathrel{\rlap{%
    \raisebox{0.3ex}{$\m@th\cdot$}}%
  \raisebox{-0.3ex}{$\m@th\cdot$}}%
=}
\makeatother

\newpagestyle{pagina}{
	\headrule
	\sethead*{\sffamily {\bfseries Entrega 2:} Espais de Hilbert de dimensió finita}{}{\sffamily Arnau Mas}
	\footrule
	\setfoot*{}{}{\sffamily \thepage}
}
\renewpagestyle{plain}{
	\footrule
	\setfoot*{}{}{\sffamily \thepage}
}
\pagestyle{pagina}

\titleformat{\section}[hang]{ \sffamily \Large}{\bfseries Problema \thesection. }{0pt}{}{\thispagestyle{plain}}

\title{\sffamily {\bfseries Entrega 2:} Espais de Hilbert de dimensió finita}
\author{\sffamily Arnau Mas}
\date{\sffamily 4 d'octubre de 2018}

\begin{document}
\maketitle

Considerem les matrius
\begin{equation*}
	A = \frac{1}{2} \begin{pmatrix}
		3 & -1 & 0 \\
		-1 & 3 & 0 \\
		0 & 0 & 4
	\end{pmatrix} \,\text{ i }\;
	B = \frac{1}{3} \begin{pmatrix}
		5 & 1 & -1 \\
		1 & 5 & 1 \\
		-1 & 1 & 5
	\end{pmatrix}
\end{equation*}

\begin{enumerate}[label=(\alph*), font=\bfseries \sffamily, wide, labelwidth=!, labelindent=0pt]
	\item Com que \( A \) i \( B \) són matrius reals, seran hermítiques si i només si són simètriques, ja que un nombre és igual al seu conjugat si i només si és real. Així, com que tant \( A \) com \( B \) són simètriques també són hermítiques. 
	\item Observem que si una matriu \( A \) té un vector propi \( \ket{u} \) de valor propi \( \lambda \) aleshores la matriu \( \alpha A \) té el mateix vector propi però amb valor propi \( \alpha \lambda \). En efecte, tenim \( A \ket{u} = \lambda\ket{u} \) i per tant \( \alpha A\ket{u} = \alpha\lambda\ket{u} \). Per tant podem diagonalitzar \( 2A \) i \( 3B \) i els valors propis que obtinguem seran, respectivament, el doble i el triple dles valors propis buscats. 

		Busquem les arrels del polinomi característic de \( 2A \), que en són els valors propis:
		\begin{align*}
			p_{2A}(x) & = \begin{vmatrix}
				x-3 & 1 & 0 \\
				1 & x-3 & 0 \\
				0 & 0 & x-4
			\end{vmatrix}
			= (x-4) \begin{vmatrix}
				x-3 & 1\\
				1 & x-3
			\end{vmatrix} \\
			& = (x-4) \left( (x-3)^2 - 1 \right) = (x-4)^2(x-2). 
		\end{align*}
		Per tant \( A \) té com a valors propis \( 2 \) i \( 1 \). Busquem ara els corresponents vectors propis. Pel valor propi 2 hem de trobar les solucions de \( 2A - 4I_3 = 0 \):  	
		\begin{equation*}
			2A - 4I-3 = \begin{pmatrix}
				-1 & -1 & 0 \\
				-1 & -1 & 0 \\
				0 & 0 & 0
			\end{pmatrix} \to 
			\begin{pmatrix}
				1 & 1 & 0 \\
				0 & 0 & 0 \\
				0 & 0 & 0
			\end{pmatrix} .
		\end{equation*}
		Així doncs, el subespai propi de valor propi 2 d'\( A \) és
		\begin{equation*}
			\ker{(A - 2I_3)} = \left\langle \begin{pmatrix}
					1 \\ -1 \\ 0
					\end{pmatrix}, \begin{pmatrix}
					0 \\ 0 \\ 1 
			\end{pmatrix} \right\rangle .
		\end{equation*}
		Pel valor propi 1 hem de calcular \( \ker{(A - I_3)} \):
		\begin{equation*}
			2A - 2I_3 = \begin{pmatrix}
				1 & -1 & 0 \\
				-1 & 1 & 0 \\
				0 & 0 & 2
			\end{pmatrix} \to
			\begin{pmatrix}
				1 & -1 & 0 \\
				0 & 0 & 0 \\
				0 & 0 & 1 
			\end{pmatrix}.
		\end{equation*}
		I per tant 
		\begin{equation*}
			\ker{(A - I_3)} = \left\langle \begin{pmatrix}
					1 \\ 1 \\0
			\end{pmatrix} \right\rangle.
		\end{equation*}
		Ens interessa, però, trobar una base ortonormal de vectors propis. Així posem
		\begin{align*}
			\ket{u_1} & = \frac{1}{\sqrt{2}}\begin{pmatrix}
				1\\-1\\0
			\end{pmatrix}, \\
			\ket{u_2} & = \begin{pmatrix}
				0\\0\\1
			\end{pmatrix}, \\
			\ket{u_3} & = \frac{1}{\sqrt{2}}\begin{pmatrix}
				1\\1\\0
			\end{pmatrix}
		\end{align*}
		i podem escriure
		\begin{equation*}
			A = 2\ket{u_1}\bra{u_1} + 2\ket{u_2}\bra{u_2} + \ket{u_3}\bra{u_3}.
		\end{equation*}

		Repetim els càlculs per a diagonalitzar \( B \).
		\begin{align*}
			p_{3B}(x) & = \begin{vmatrix}
				x-5 & -1 & 1 \\
				-1 & x-5 & -1 \\
				1 & -1 & x-5
				\end{vmatrix} = (x-5) \begin{vmatrix}
				x-5 & -1 \\
				-1 & x-5
				\end{vmatrix} + 2 \begin{vmatrix}
				-1 & 1 \\
				-1 & x-5
			\end{vmatrix} \\
			& = (x-5) \left((x-5)^2 - 1\right) + 2(1 - (x-5)) = (x-5)(x-4)(x-6) - 2(x-6) \\
			& = (x-6)((x-4)(x-5) - 2) = (x-6)^2(x-3).
		\end{align*}
		Per tant \( B \) té 2 i 1 per valors propis. 

		Calculem \( \ker{(B - 2I_3)} \):
		\begin{equation*}
			3B - 6I_3 = \begin{pmatrix}
				-1 & 1 & -1 \\
				1 & -1 & 1 \\
				-1 & 1 & -1
			\end{pmatrix} \to 
			\begin{pmatrix}
				-1 & 1 & -1 \\
				0 & 0 & 0 \\
				0 & 0 & 0
			\end{pmatrix}.
		\end{equation*}
		I per tant
		\begin{equation*}
			\ker{(B - 2I_3)} = \left\langle \begin{pmatrix}
					1\\1\\0
					\end{pmatrix}, \begin{pmatrix}
					1\\0\\-1
			\end{pmatrix} \right\rangle. 
		\end{equation*}

		Pel valor propi 1 tenim
		\begin{align*}
			3B - 3I_3 & = \begin{pmatrix}
				2 & 1 & -1 \\
				1 & 2 & 1 \\
				-1 & 1 & 2
			\end{pmatrix} \to 
			\begin{pmatrix}
				2&1&-1\\
				1&2&1\\
				0&0&0
			\end{pmatrix} \to
			\begin{pmatrix}
				2 & 1 & -1 \\
				0 & 3 & 3 \\
				0 & 0 & 0
			\end{pmatrix} \\
			& \to
			\begin{pmatrix}
				2 & 1 & -1 \\
				0 & 1 & 1 \\
				0 & 0 & 0
			\end{pmatrix} \to 
			\begin{pmatrix}
				2 & 2 & 0 \\
				0 & 1 & 1 \\
				0 & 0 & 0
			\end{pmatrix} .
		\end{align*}
		Per tant 
		\begin{equation*}
			\ker{(B - I_3)} = \left\langle \begin{pmatrix}
					1\\-1\\1
			\end{pmatrix} \right\rangle.
		\end{equation*}
		Com abans, però, podem fer que la base de vectors propis sigui ortonormal, per exemple
		\begin{align*}
			\ket{v_1} & = \frac{1}{\sqrt{2}}\begin{pmatrix}
				1\\1\\0
			\end{pmatrix}, \\
			\ket{v_2} & = \frac{1}{\sqrt{2}}\begin{pmatrix}
				1\\0\\-1
			\end{pmatrix}, \\
			\ket{v_3} & = \frac{1}{\sqrt{3}}\begin{pmatrix}
				1\\-1\\1
			\end{pmatrix}
		\end{align*}
		i per tant
		\begin{equation*}
			B = 2\ket{v_1}\bra{v_1} + 2\ket{v_2}\bra{v_2} + \ket{v_3}\bra{v_3}.
		\end{equation*}

	\item Tenim que \( P_1 = \ket{u_3}\bra{u_3} \) és el projector al subespai propi de valor propi 1, i \( P_2 =  \ket{u_1}\bra{u_1} + \ket{u_2}\bra{u_2} \) és el projector al subespai propi de valor propi 2. Explícitament
		\begin{equation*}
			P_1 = \frac{1}{2}\begin{pmatrix}
				1 & 1 & 0
				\end{pmatrix} \begin{pmatrix}
				1\\1\\0
				\end{pmatrix} = \frac{1}{2}\begin{pmatrix}
				1 & 1 & 0\\
				1 & 1 & 0\\
				0 & 0 & 0
			\end{pmatrix},
		\end{equation*}
		i
		\begin{equation*}
			P_2 = \frac{1}{2}\begin{pmatrix}
				1 & -1 & 0
				\end{pmatrix} \begin{pmatrix}
				1\\-1\\0
				\end{pmatrix} + \begin{pmatrix}
				0 & 0 & 1	
				\end{pmatrix} \begin{pmatrix}
				0\\0\\1
				\end{pmatrix} = \frac{1}{2}\begin{pmatrix}
				1 & -1 & 0\\
				-1 & 1 & 0\\
				0 & 0 & 0
				\end{pmatrix} + \begin{pmatrix}
				0 & 0 & 0 \\
				0 & 0 & 0 \\
				0 & 0 & 1
			\end{pmatrix}.
		\end{equation*}
Per tant \( A = 2P_2 + P_1 \).

\item Tenim
	\begin{equation*}
		\ket{u_3} = \ket{v_1} \in \ker{(A-I_3)} \cap \ker{(B-2I_3)}.
	\end{equation*}
i
\begin{equation*}
	\ket{v_3} = \sqrt{\frac{2}{3}} \ket{u_1} + \frac{1}{\sqrt{2}} \ket{u_2} \in \ker{(A-2I_3)} \cap \ker{(B-I_3)}.
\end{equation*}
És a dir, hem trobat dos vectors que són vectors propis d'\( A \) i \( B \) simultàniament. I també
\begin{equation*}
	\frac{1}{\sqrt{6}}\begin{pmatrix}
		-1\\1\\2
	\end{pmatrix} = -\frac{1}{\sqrt{3}}\ket{u_1} + \sqrt{\frac{2}{3}}\ket{u_2} = \frac{1}{\sqrt{6}}\ket{v_1} - \frac{2}{\sqrt{3}}\ket{v_2} \in \ker{(A - 2I_3)} \cap \ker{(B-2I_3)}.
\end{equation*}
Per tant tenim tres vectors propis d'\( A \) i \( B \) linealment independents; és a dir, una base que diagonalitza a \( A \) i \( B \) simultàniament. 
	
\item	Tenim
	\begin{equation*}
	AB = \frac{1}{6} \begin{pmatrix}
		14 & -2 & -4 \\
		-2 & 14 & 4 \\
		-4 & 4 & 20,
	\end{pmatrix}
	\end{equation*}
	i per tant \( (AB)^{\dagger} \). Així
	\begin{equation*}
		[A,B] = AB - BA = AB - (A^{\dagger}B^{\dagger})^{\dagger} = AB - (AB)^{\dagger} = AB - AB = 0. 
	\end{equation*}
	

\end{enumerate}
\end{document}
