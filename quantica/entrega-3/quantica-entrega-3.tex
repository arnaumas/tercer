\documentclass[12pt]{article}

\usepackage[utf8]{inputenc}
\usepackage[T1]{fontenc}
\usepackage[catalan]{babel}
\usepackage{lmodern}
\usepackage{geometry}
\usepackage{hyperref}
\usepackage[dvipsnames]{xcolor}
\usepackage[bf,sf,small,pagestyles]{titlesec}
\usepackage{titling}
\usepackage[font={footnotesize, sf}, labelfont=bf]{caption} 
\usepackage{siunitx}
\usepackage{graphicx}
\usepackage{booktabs}
\usepackage{amsmath,amssymb}
\usepackage{physics}
\usepackage[catalan,sort]{cleveref}
\usepackage{enumitem}

\geometry{
	a4paper,
	right = 2.5cm,
	left = 2.5cm,
	bottom = 3cm,
	top = 3cm
}

\hypersetup{
	colorlinks,
	linkcolor = {red!50!blue},
	linktoc = page
}

\crefname{figure}{figura}{figures}
\crefname{table}{taula}{taules}
\numberwithin{table}{section}
\numberwithin{figure}{section}
\numberwithin{equation}{section}

\graphicspath{{./figs/}}

% Unitats
\sisetup{
	inter-unit-product = \ensuremath{ \cdot },
	allow-number-unit-breaks = true,
	detect-family = true,
	list-final-separator = { i },
	list-units = single
}

\newcommand{\Z}{\mathbb{Z}}
\renewcommand{\vec}[1]{\mathbf{#1}}
\newcommand{\N}{\mathbb{N}}
\newcommand{\R}{\mathbb{R}}
\newcommand{\Ry}{\mathit{Ry}}
\newcommand{\inn}[2]{\left\langle #1 , #2 \right\rangle}
\newcommand{\proj}[1]{\ketbra{#1}{#1}}
\newcommand{\parbreak}{
	\begin{center}
		--- $\ast$ ---
	\end{center} 
}
\makeatletter
\newcommand*{\defeq}{\mathrel{\rlap{%
    \raisebox{0.3ex}{$\m@th\cdot$}}%
  \raisebox{-0.3ex}{$\m@th\cdot$}}%
=
}
\makeatother

\newpagestyle{pagina}{
	\headrule
	\sethead*{\sffamily {\bfseries Entrega 3:} Discriminació d'estats quàntics}{}{\theauthor}
	\footrule
	\setfoot*{}{}{\sffamily \thepage}
}
\renewpagestyle{plain}{
	\footrule
	\setfoot*{}{}{\sffamily \thepage}
}
\pagestyle{pagina}

\titleformat{\section}[hang]{ \sffamily \Large}{\bfseries Problema \thesection}{0pt}{}{\thispagestyle{plain}}

\title{\sffamily {\bfseries Entrega 3:} Discriminació d'estats quàntics}
\author{\sffamily Sandro Barissi, Arnau Mas}
\date{\sffamily 31 d'octubre de 2018}

\begin{document}
\maketitle
\section{}
Si \( \ket{\psi_1} \) i \( \ket{\psi_2} \) són estats ortonormals aleshores l'operador \( A = \proj{\psi_1} - \proj{\psi_2} \) és hermític i per tant es correspon a un observable. Aleshores, si \( \ket{\psi} \) és l'estat desconegut i mesurem \( A \) sobre \( \ket{\psi} \) obtindrem 1 amb probabilitat 1 si \( \ket{\psi} = \ket{\psi_1} \) i \( -1 \) amb probabilitat 1 si \( \ket{\psi} = \ket{\psi_2} \).

\section{}
Si \( \ket{\psi_1} \) i \( \ket{\psi_2} \) són estats no ortogonals aleshores no hi ha cap mesura que els pugui diferenciar. Si existís aquesta mesura, el corresponent operador hauria de tenir-los com a vectors propis ---de manera que obtindriem amb probabilitat 1 un dels resultats si l'estat és \( \ket{\psi_1} \) i amb probabilitat 1 l'altre resultat si l'estat és \( \ket{\psi_2} \)---. Però això no pot ser ja que els vectors propis amb valor propi diferent d'un operador hermític són ortogonals. 

\section{}
Tenim \( \ket{\psi_1} = \ket{0} \) i \( \ket{\psi_2} = \frac{\sqrt{3}}{2}\ket{0} - \frac{1}{2}\ket{1} \) amb \( \braket{0}{1} = 0 \), i triem un estat d'entre els dos de manera aleatòria. 

\begin{enumerate}[label=(\alph*), font=\bfseries \sffamily, wide, labelwidth=!, labelindent=0pt]
	\item Tenim
		\begin{equation*}
			\braket{\psi_1}{\psi_2} = \frac{\sqrt{3}}{2}\braket{0}{1} - \frac{1}{2}\braket{0}{1} = \frac{\sqrt{3}}{2} \neq 0.
		\end{equation*}
	\item Considerem \( A = \proj{1} \). És clar que \( \ket{1} \) és vector propi de \( A \) amb valor propi 1. I com que \( A\ket{0} = \ket{1}\braket{1}{0} = 0 \), tenim que \( \ket{0} \) és l'altre vector propi amb valor propi 0. 
	\item Com que si mesurem \( A \) sobre \( \ket{\psi_1} \) obtenim 0 amb probabilitat 1, obtindrem 1 amb probabilitat 0. I per tant si obtenim 1 sabem segur que l'estat previ a la mesura era \( \ket{\psi_2} \). 
	\item	Per determinar per quin estat hem d'apostar hem de calcular les probabilitats \( p(\psi_1 \vert 0) \) i \( p(\psi_2 \vert 0) \). Tenim
		\begin{equation*}
			p(0 \vert \psi_1) = \abs{\braket{0}{\psi_1}}^2 = 1
		\end{equation*}
		i
		\begin{equation*}
			p(1 \vert \psi_2) = \abs{\braket{0}{\psi_2}}^2 = \abs{\frac{\sqrt{3}}{2}}^2 = \frac{3}{4}. 
		\end{equation*}
		Ara, aplicant el teorema de Bayes obtenim
		\begin{equation*}
			p(\psi_1 \vert 0) = \frac{p(\psi_1) p(0 \vert \psi_1)}{p(0)} = \frac{p(\psi_1) p(0 \vert \psi_1)}{p(\psi_1)p(0 \vert \psi_1) + p(\psi_2) p(0 \vert \psi_2)} = \frac{\frac{1}{2}}{\frac{1}{2}\cdot 1 + \frac{1}{2} \cdot \frac{3}{4}} = \frac{4}{7}.
		\end{equation*}
		I com que \( p(\psi_2 \vert 0) = 1 - p(\psi_1 \vert 0) = \frac{3}{7} \) i per tant si obtenim 0 hem d'apostar per l'estat \( \ket{\psi_1} \). 
	\item La probabilitat d'encertar serà \( p(1)p(\psi_2 \vert 1) + p(0)p(\psi_1 \vert 0) \). Tenim 
		\begin{equation*}
			p(1) = p(1 \vert \psi_1)p(\psi_1) + p(1 \vert \psi_2)p(\psi_2) = 0 \cdot \frac{1}{2} + \frac{1}{4}\cdot \frac{1}{2} = \frac{1}{8}. 
		\end{equation*}
		I per tant \( p(0) = 1 - p(1) = \frac{7}{8} \). I així la probabilitat d'encertar és
		\begin{equation*}
			\frac{1}{8}\cdot 1 + \frac{7}{8}\cdot \frac{4}{7} = \frac{5}{8} = \num{0.625}. 
		\end{equation*}
\end{enumerate}

\section{}
Si fixem l'eix \( z \) al llarg de l'spin que correspon a \( \ket{0} \) aleshores \( A = \proj{0} \) és precisament l'operador que correspon a un Stern-Gerlach en la direcció \( z \). 

Fixem a l'esfera de Bloch les coordenades que corresponen a la base \( \ket{0}, \ket{1} \) de manera que podem escriure 
\begin{equation*}
 	\ket{\psi_2} = \cos{\tfrac{\pi}{6}} \ket{0} + e^{i\pi} \sin{\tfrac{\pi}{6}}\ket{1} = \cos{\tfrac{\pi}{6}} \ket{0} - \sin{\tfrac{\pi}{6}}\ket{1} 
\end{equation*}
 L'operador que correspon a un Stern-Gerlach qualsevol ve donat per 
\begin{equation*}
	\sigma_{\vec{n}} = \proj{+n} - \proj{-n}
\end{equation*}
amb
\begin{equation*}
	\ket{+n} = \cos{\tfrac{\theta}{2}} \ket{0} + e^{i\phi}\sin{\tfrac{\theta}{2}} \ket{1}
\end{equation*}
i
\begin{equation*}
	\ket{-n} = \sin{\tfrac{\theta}{2}} \ket{0} - e^{i\phi}\cos{\tfrac{\theta}{2}} \ket{1}.
\end{equation*}
Observem que per simetria, \( \phi = 0 \) o \( \phi = \pi \). 

Repetim els càlculs de l'apartat anterior. Calculem primer les probabilitats de la mesura de \( \sigma_\vec{n} \) sobre els estats:
\begin{gather*}
	\braket{\psi_1}{+n} = \cos{\tfrac{\theta}{2}} \implies p(+n \vert \psi_1) = \cos{\left(\tfrac{\theta}{2}\right)}^2 \\
	\braket{\psi_1}{-n} = \sin{\tfrac{\theta}{2}} \implies p(-n \vert \psi_1) = \sin{\left(\tfrac{\theta}{2}\right)}^2 \\
	\braket{\psi_2}{+n} = \cos{\tfrac{\pi}{6}} \cos{\tfrac{\theta}{2}} \pm \sin{\tfrac{\pi}{6}}\sin{\tfrac{\theta}{2}} = \cos{\left(\tfrac{\theta}{2} \pm \tfrac{\pi}{6}\right)} \implies p(+n \vert \psi_2) = \cos{\left(\tfrac{\theta}{2} \pm \tfrac{\pi}{6}\right)}^2 \\
	\braket{\psi_2}{-n} = \cos{\tfrac{\pi}{6}} \sin{\tfrac{\theta}{2}} \pm \sin{\tfrac{\pi}{6}}\cos{\tfrac{\theta}{2}} = \sin{\left(\tfrac{\theta}{2} \pm \tfrac{\pi}{6}\right)} \implies p(-n \vert \psi_2) = \sin{\left(\tfrac{\theta}{2} \pm \tfrac{\pi}{6}\right)}^2. 
\end{gather*}

Per calcular la probabilitat d'encertar hauriem de trobar les probabilitats \( p(\psi_1 \vert \pm n) \) i \( p(\psi_2 \vert \pm n) \) per determinar quina és la millor estratègia, és a dir, per decidir per quin estat hem d'apostar en funció del resultat de la mesura. Ara bé, sigui com sigui hi ha només dues estratègies possibles: apostar per \( \psi_1 \) si obtenim \( +n \), o bé apostar per \( \psi_2 \) si obtenim \( +n \). En el primer cas, la probabilitat d'encertar serà \( p(+n)p(\psi_1 \vert +n) + p(-n)p(\psi_2 \vert -n) \). I en el segon serà \( p(+n)p(\psi_2 \vert +n) + p(-n)p(\psi_1 \vert -n) \). Però pel teorema de Bayes tenim:
\begin{gather*}
	p = p(+n)p(\psi_1 \vert +n) + p(-n)p(\psi_2 \vert -n) = p(\psi_1)p(+n \vert \psi_1) + p(\psi_2)p(-n \vert \psi_2) \\
1 - p = p(+n)p(\psi_2 \vert +n) + p(-n)p(\psi_1 \vert -n) = p(\psi_2)p(+n \vert \psi_2) + p(\psi_1)p(-n \vert \psi_1).
\end{gather*}

Per tant la probabilitat de encertar serà \( \max{\{p, 1 - p\}} \). Tenim, amb \( \phi = 0 \),
\begin{equation*}
	p_0 = \frac{1}{2}\left(\cos{\left(\tfrac{\theta}{2}\right)}^2 + \sin{\left(\tfrac{\theta}{2} + \tfrac{\pi}{6}\right)}^2\right).
\end{equation*}
i amb \( \phi = \pi \),
\begin{equation*}
	p_\pi = \frac{1}{2}\left(\cos{\left(\tfrac{\theta}{2}\right)}^2 + \sin{\left(\tfrac{\theta}{2} - \tfrac{\pi}{6}\right)}^2\right).
\end{equation*}
I per tot \( \theta \in [0, \pi] \), \( p_0 > p_\pi \), per tant, amb la primera estatègia tindrem probabilitat màxima amb \( \phi = 0 \). Si maximitzem \( p_0 \) respecte \( \theta \), trobem que \( p \) assoleix un valor màxim de \( \frac{3}{4} \) per \( \theta = \frac{\pi}{3} \). La representació en la base que hem triat d'aquest observable és
\begin{equation*}
\sigma_0 = \frac{1}{2}\begin{pmatrix}
	1 & \sqrt{3} \\
	\sqrt{3} & 1
\end{pmatrix}.
\end{equation*}

Obtenim també probabilitat d'encertar \( \frac{3}{4} \) seguint l'altra estratègia quan \( \phi = \pi \) i \( \theta = \frac{2\pi}{3} \). Si representem el corresponent observable obtenim
\begin{equation*}
\sigma_\pi = -\frac{1}{2}\begin{pmatrix}
	1 & \sqrt{3} \\
	\sqrt{3} & 1
\end{pmatrix} = -\sigma_0.
\end{equation*}
Observem, però, que de fet \( \sigma_\pi \) té els mateixos vectors propis amb els valors propis canviats de signe. Així que obtenir \( +n \) com a resultat d'una mesura de \( \sigma_0 \) és el mateix que obtenir \( -n \) com a resultat d'una mesura de \( \sigma_\pi \). Per tant, l'estratègia que ens ha portat a \( \sigma_\pi \) de fet és exactament la mateixa que seguim quan mesurem \( \sigma_0 \). En termes més planers, \( \sigma_\pi \) no és més que capgirar \( \sigma_0 \). 

\section{}
Considerem ara una parella d'estats qualssevol, \( \ket{\psi_1} \) i \( \ket{\psi_2} \). Fixem coordenades a l'esfera de Bloch de manera que l'eix \( z \) es correspongui amb l'spin \( \psi_1 \), i l'origen de \( \phi \) estigui al semipla generat per \( \psi_1 \)	i \( \psi_2 \). D'aquesta manera, si \( \ket{0} \) i \( \ket{1} \) són els estats propis de l'Stern-Gerlach en la direcció \( z \) en les coordenades que hem triat, podem escriure
\begin{gather*}
\ket{\psi_1} = \ket{0} \\
\ket{\psi_2} = \cos{\tfrac{\alpha}{2}} \ket{0} + \sin{\tfrac{\alpha}{2}} \ket{1} 
\end{gather*}
amb \( \alpha \in [0, \pi] \) l'angle que separa els espins \( \psi_1 \) i \( \psi_2 \). Ens preguntem per l'Stern-Gerlach que maximitza la probabilitat de distingir entre els estats, tenint en compte que el sistema està en l'estat \( \ket{\psi_1} \) amb probabilitat \( p \), i per tant amb probabilitat \( 1 - p \) en l'estat \( \ket{\psi_2} \). Els càlculs són idèntics a l'apartat anterior, amb \( \frac{\alpha}{2} \) en lloc de \( \frac{\pi}{6} \) i \( p \) i \( 1 -p  \) en lloc de \( \frac{1}{2} \).

Amb \( \phi = 0 \), denotem per \( P_0 \) la probabilitat d'encertar apostant per \( \psi_1 \) si obtenim \( +n \) i per \( \psi_2 \) si obtenim \( -n \), i per \( P_\pi \) la probabilitat d'encertar seguint la mateixa estratègia si \( \phi = \pi \). Aleshores
\begin{gather}
	P_0(\theta) = p \cos{\left(\tfrac{\theta}{2}\right)}^2 + (1-p) \sin{\left(\tfrac{\theta - \alpha}{2}\right)}^2 \\
	P_\pi(\theta) = p \cos{\left(\tfrac{\theta}{2}\right)}^2 + (1-p) \sin{\left(\tfrac{\theta + \alpha}{2}\right)}^2 
\end{gather}
Les probabilitats d'encertar seguint les estratègies oposades són, naturalment, \( 1 - P_0 \) i \( 1 - P_\pi \). Observem que \( P_0(\theta) = 1 - P_\pi(\pi - \theta) \), Que es correspon amb el que hem argumentat abans: capgirar l'aparell Stern-Gerlach i canviar l'estratègia ens porta al mateix resultat. Així la probabilitat d'encertar serà \( P(\theta) = \max{\{P_0(\theta), 1 - P_0(\theta)\}} \). 

Les pitjors condicions per encertar es donaran pels \( \alpha \) i \( \theta \)	tals que \( P(\theta) = \frac{1}{2} \), ja que això vol dir que la millor estratègia és equivalent a triar a l'atzar. Quan \( \alpha \to 0 \), \( P_0(\theta) \to \frac{1}{2} \), de manera que com més propers són dos espins, més díficil és de distingir-los. Evidentment si \( \alpha = 0 \) aleshores tenim dos estats idèntics i per tant la probabilitat d'encertar és 1. 

\section{}
Demostrem que poder clonar estats és equivalent a poder distingir estats. Per una banda, si podem distingir dos estats qualssevol, tenim un observable que els té com a estats propis. Per tant, si anem fent mesures d'aquest observable sobre estats qualssevol, anirem obtenint còpies dels dos estats originals.

D'altra banda, suposem que volem distingir entre dos estats qualssevol \( \ket{\psi_1} \) i \( \ket{\psi_2} \). Si \( \psi \) és el nostre estat desconegut i podem clonar-lo, podem fer tantes mesures sobre \( \psi \) sense alterar-lo (una sobre cada còpia). Si anem mesurant els observables \( \proj{\psi_1} \) i \( \proj{\psi_2} \) sobre \( \psi \) en algun moment obtindrem 0 com a resultat d'alguna de les mesures. Si obtenim 0 com a resultat de mesurar \( \proj{\psi_1} \), sabem segur que \( \ket{\psi} = \ket{\psi_2} \) ja que si \( \ket{\psi} \) fos \( \ket{\psi_1} \), el resultat de mesurar \( \proj{\psi_1} \)	és 1 amb probabilitat 1. De la mateixa manera, si obtenim 0 quan mesurem \( \proj{\psi_2} \) sabem del cert que \( \ket{\psi} = \ket{\psi_1} \).

Així doncs, com que poder distingir entre estats qualssevol i poder clonar-los és equivalent, com que ja hem vist que no es poden distingir estats no ortogonals, tampoc podem clonar-los. 

\end{document}
