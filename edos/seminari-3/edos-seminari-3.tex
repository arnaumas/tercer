\documentclass[12pt]{article}

\usepackage[utf8]{inputenc}
\usepackage[T1]{fontenc}
\usepackage[catalan]{babel}
\usepackage{lmodern}
\usepackage{geometry}
\usepackage{hyperref}
\usepackage[dvipsnames]{xcolor}
\usepackage[bf,sf,small,pagestyles]{titlesec}
\usepackage{titling}
\usepackage[font={footnotesize, sf}, labelfont=bf]{caption} 
\usepackage[font={footnotesize, sf}, labelfont=bf]{subcaption}
\usepackage{siunitx}
\usepackage{graphicx}
\usepackage{booktabs}
\usepackage{amsmath,amssymb}
\usepackage[sort]{cleveref}
\usepackage{enumitem}

\geometry{
	a4paper,
	right = 2.5cm,
	left = 2.5cm,
	bottom = 3cm,
	top = 3cm
}

\hypersetup{
	colorlinks,
	linkcolor = {red!50!blue},
	linktoc = page
}

\crefname{figure}{figura}{figures}
\crefname{table}{taula}{taules}
\numberwithin{table}{section}
\numberwithin{equation}{section}
\numberwithin{figure}{section}

\graphicspath{{./figs/}}

% Unitats
\sisetup{
	inter-unit-product = \ensuremath{ \cdot },
	allow-number-unit-breaks = true,
	detect-family = true,
	list-final-separator = { i },
	list-units = single
}

\newcommand{\Z}{\mathbb{Z}}
\renewcommand{\vec}[1]{\mathbf{#1}}
\newcommand{\N}{\mathbb{N}}
\newcommand{\R}{\mathbb{R}}
\newcommand{\C}{\mathbb{C}}
\newcommand{\Ry}{\mathit{Ry}}
\let\Im\relax
\let\Re\relax
\let\div\relax
\DeclareMathOperator{\Im}{Im}
\DeclareMathOperator{\Re}{Re}
\DeclareMathOperator{\div}{div}
\newcommand{\abs}[1]{\lvert #1 \rvert}
\newcommand{\inn}[2]{\left\langle #1 , #2 \right\rangle}
\newcommand{\parbreak}{
	\begin{center}
		--- $\ast$ ---
	\end{center} 
}
\makeatletter
\newcommand*{\defeq}{\mathrel{\rlap{%
    \raisebox{0.3ex}{$\m@th\cdot$}}%
  \raisebox{-0.3ex}{$\m@th\cdot$}}%
	=
}
\makeatother

\newpagestyle{pagina}{
	\headrule
	\sethead*{\sffamily {\bfseries Seminari 3:} {\sffamily EDPs quasilineals}}{}{\theauthor}
	\footrule
	\setfoot*{}{}{\sffamily \thepage}
}
\renewpagestyle{plain}{
	\footrule
	\setfoot*{}{}{\sffamily \thepage}
}
\pagestyle{pagina}

\title{\sffamily {\bfseries Seminari 3:} {\sffamily EDPs quasilineals}}
\author{\sffamily Arnau Mas}
\date{\sffamily 6 de juny de 2019}

\begin{document}
\maketitle

\addtocounter{section}{1}
\section*{Problema 1}
Una equació en derivades parcials lineal es pot escriure com 
\begin{equation*}
	A(x,y) \partial_x z + B(x,y) \partial_y z = C(x,y)z.
\end{equation*}
El corresponent sistema característic és
\begin{equation*}
	\left. 	
		\begin{aligned}
			\dot{x} & = A(x,y) \\
			\dot{y} & = B(x,y) \\
			\dot{z} & = C(x,y)z
		\end{aligned}
	\right\}
\end{equation*}
Suposem que existeix una integral primera \( H_1(x,y) \) tal que podem aïllar \( y \) de l'equació \( H_1(x,y) = h \) de la forma \( y = g(x,h) \). Aleshores, si substituïm a la primera i tercera equacions obtenim el sistema
\begin{equation*}
	\left. 	
		\begin{aligned}
			\dot{x} & = A(x,g(x,h)) \\
			\dot{z} & = C(x,g(x,h))z
		\end{aligned}
	\right\}
\end{equation*}
Podem intentar buscar una integral primera d'aquest sistema. Si dividim les dues equacions trobem
\begin{equation*}
	\frac{dz}{dx} = \frac{C(x,g(x,h))}{A(x, g(x,h))}z,
\end{equation*}
que és una equació separable. Si \( D(x,h) \) és una primitiva de \( \frac{C(x,h)}{A(x,h)} \) respecte \( x \) aleshores la solució és \( z(x) = K e^{D(x,h)} \) per alguna constant \( K \) i per tant una integral primera del sistema és, en termes de \( z \), \( x \) i \( h \)
\begin{equation*}
	V(x,h,z) = z e^{-D(x,h)}.
\end{equation*}

Si ara substituïm \( h = H_1(x,y) \) trobem que la segona integral primera del sistema característic és
\begin{equation*}
	H_2(x,y,z) = z e^{-D(x,H(x,y))}.
\end{equation*}

Comprovem que \( H_2 \) és efectivament una integral primera del sistema. Calculem les parcials de \( H_2 \) i trobem
\begin{equation*}
	\partial_z H_2(x,y,z) = e^{-D(x,H_1(x,y))}
\end{equation*}
i
\begin{equation*}
	\partial_y H_2(x,y,z) = -ze^{-D(x,H_1(x,y))} \big(\partial_h D(x, H_1(x,y)) \partial_y H_1(x,y)\big).
\end{equation*}
I finalment, com que \( g(x, H_1(x,y)) = y \), tenim
\begin{align*}
	\partial_x H_2(x,y,z) & = -ze^{-D(x,H_1(x,y))} \big(\partial_x D(x,H_1(x,y)) + \partial_h D(x, H_1(x,y)) \partial_x H_1(x,y)\big) \\
												& = -ze^{-D(x,H_1(x,y))} \left(\frac{C(x,y)}{A(x,y)} + \partial_h D(x, H_1(x,y)) \partial_x H_1(x,y)\right).
\end{align*}
Així doncs
\begin{align*}
	\dot{H}_2(x,y,z) & = A(x,y)\partial_x H_2(x,y,z) + B(x,y)\partial_y H_2(x,y,z) + C(x,y)z\partial_z H_2(x,y,z) \\
									 & = -zC(x,y) e^{-D(x,H_1(x,y))} \\
									 & -z e^{-D(x,H_1(x,y))} \partial_h D(x, H_1(x,y)) \big( A(x,y) \partial_x H_1(x,y) + B(x,y) \partial_y H_1(x,y) \big) \\ 
									 & + zC(x,y) e^{-D(x,H_1(x,y))} = 0,
\end{align*}
on hem fet servir que \( H_1 \) és integral primera i per tant \( A\partial_xH_1 + B\partial_yH_1 = 0 \). Veiem, doncs, que \( H_2 \) és efectivament integral primera. 

Finalment hem de comprovar que \( H_1 \) i \( H_2 \) són funcionalment independents. Però això és clar ja que \( \partial_z H_2 \) mai no s'anu\l.la i en canvi \( \partial_z H_1 \) és 0 a tot arreu. Els gradients de \( H_1 \) i \( H_2 \), per tant, són linealment independents i en conseqüència \( H_1 \) i \( H_2 \) són funcionalment independents. 

\parbreak

Anem a resoldre fent servir els resultats que acabem de desenvolupar l'equació
\begin{equation*}
	x \partial_x z + 2y \partial_y z = 4yx^2 z,
\end{equation*}
que és lineal. El sistema característic és 
\begin{equation*}
	\left. 	
		\begin{aligned}
			\dot{x} & = x \\
			\dot{y} & = 2y \\
			\dot{z} & = 4yx^2z
		\end{aligned}
	\right\}
\end{equation*}

Busquem en primer lloc una integral primera per a les dues primeres equacions. Tenim
\begin{equation*}
	\frac{dy}{dx} = \frac{2y}{x},
\end{equation*}
que és una equació separable. Si la resolem trobem
\begin{equation*}
	\tfrac{1}{2}\log{y} = K + \log{x}
\end{equation*}
i per tant
\begin{equation*}
	e^{2K} = \frac{y}{x^2}.
\end{equation*}
Deduïm d'aquí que \( H_1(x,y) = \frac{y}{x^2} \) és una integral primera. De l'equació \( H_1(x,y) = h \) podem aïllar-ne \( y \) com \( y = hx^2 \). Si ara substituïm això a la primera i tercera equacions arribem al sistema
\begin{equation*}
	\left. 	
		\begin{aligned}
			\dot{x} & = x \\
			\dot{z} & = 4hx^4z
		\end{aligned}
	\right\}
\end{equation*}


\addtocounter{section}{2}
\addtocounter{section}{-1}
\setcounter{equation}{0}
\section*{Problema 2}

\end{document}

